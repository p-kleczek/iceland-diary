\chapter {15.08.}

Spokojnie zwijamy sobie namioty, pakujemy sakwy, a tu widzimy, że drogą jedzie kolumna z 10 jednakowo ubranych rowerzystów z pełnym ekwipunkiem. Wyglądają tak... znajomo! Tak, mijaliśmy ich już koło Djúpivogur, tyle że wtedy oni jechali w przeciwnym kierunku. Nie daje nam spokoju, co to za impreza - rajd? kolonia? Wszystkiego dowiemy się wkrótce...

Rano kropi, potem pada, a momentami leje. Serwis pogodowy nie pozostawia złudzeń - taka aura utrzyma się do wieczora.

Zaraz po wjeździe do miejscowości Geysir, gdy tylko minąłem znak “teren zabudowany”, usłyszałem charakterystyczny syk gdzieś przy ziemi. No tak, brezentowa łatka też się przetarła (wraz z dętką) i teraz ja również przejeżdżając przez kałuże robię za gejzer...  Kiedy towarzystwo idzie oglądać tryskający gejzer, ja łatam oponę na zapleczu jakiś magazynów - i przy okazji natykam się na tych samych Niemców, którzy wczoraj dali mi łatkę :) Na wszelki wypadek tym razem już nie kombinuję z brezentem, nie dziaduję, tylko naklejam "naklejkę".

Powracająca od gejzera Trójka przynosi nowiny - owi pro-rowerzyści, to w rzeczywistości kanadyjska rodzina. Widząc zdziwione spojrzenia, kobieta wyglądająca na matkę szybko sprostowała: "Tak, jeździmy sobie z mężem i 9 moich dzieci." Wow! Brać taką chmarę dzieci, z czego najstarsze ma może 20 lat, a najmłodsze koło 10 na Islandię? Szacunek! Jeszcze gdy dodali, że ich objazdówka trwa łącznie dwa miesiące, a śpią właściwie codziennie na dziko, to nagle poczuliśmy się tacy mali. Bo my co drugi dzień nocujemy na kempingu, a i tak po 3 tygodniach mamy już dość.

Uwaga turystyczna. Nie nastawiajcie się, że gejzer to będzie jak fontanna wylewająca hektolitry wody albo hydrant z kreskówek, z którego tryska intensywny strumień wody. W rzeczywistości "erupcja" trwa 2-3 sekundy - na tyle krótko, że co po niektórzy nawet się nie orientują, że to już!

Nasz główny checkpoint na trasie to Reykholt - tam według mapy ma się znajdować spożywczak. Rzeczywistość nieco nas rozczarowała, bo zamiast Nettó albo chociaż Samkaupa zastajemy jedynie sklepik "na CPN-ie". Zgoda, kupiliśmy wszystko czego potrzebujemy po w miarę znośnych cenach, no ale nastawialiśmy się na szał zakupów... Za oknem pada, według prognozy deszcz przejdzie koło 18:00. A ponieważ w środku w miarę ciepło, więc siedzimy sobie ponad dwie godziny przy stolikach jedząc, korzystając z internetów i grając w karty. Podkręciliśmy nawet grzejnik z 1 na 3 - tak żeby szybciej się wysuszyć. Po jakiś 20 minutach przyszedł właściciel z krótkim pytaniem: Did you turn on the heat? - Yes. - That’s not a living room! On poszedł, my grzecznie skręciliśmy na 1, ale widzieliśmy że od tego momentu gość miał nas cały czas na oku.

\img{./photos/x-s-2014-08-15_16-24-58__105.jpg}{card_break}{Przymusowa przerwa w podróży}

W Kerið zatrzymaliśmy się, by oglądnąć sobie malowniczy krater. Tę atrakcję oglądamy będąc dokumentnie przemoczeni, trzęsąc się z zimna i szczając zębami, stąd w skrócie nasze nastawienie wygląda tak - "szybko zaliczyć, odfajkować"! Sam krater może i ładny, ale dlaczego za przejście dosłownie paru metrów i rzucenie okiem na jeziorko powulkaniczne każą płacić 350 kr?!

Chcieliśmy zanocować na kempingu przy krzyżówce \road{35} i \road{36} - widzianym na którejś mapie - ale oczywście kempingu brak. Mnie znów nawala przednie koło (co moment muszę dopompowywać), więc zatrzymujemy się w przydrożnym fast-food barze. Ja zmieniam dętki, a reszta zajada hot dogi i popija kawę.

\img{./photos/x-s-2014-08-14_22-19-41__266.jpg}{will_there_be_sun}{Będzie słońce?!}

Ponieważ jestem dokumentnie przemoczony, a temperatura nie rozpieszcza, więc by uniknąć zaziębienia się postanawiam pocisnąć samotnie na kemping do Þingvellir. Już, już niemal widzę zabudowania, gdy na ostatnim podjeździe (jakieś 6 km przed wioską) znów łapię kapcia. Z początku łudziłem się, że wystarczy trochę podpompować i jakoś zajadę, więc gdy przejeżdżający akurat drogą rangersi spytali "Do you need help?”, to odparłem “No, no problem”. Gdy jednak zobaczyłem jak szybko ucieka powietrze, zmieniłem zdanie i skorzystałem z zaoferowanej możliwość podwózki. Po raz pierwszy w takiej sytuacji nie mam poczucia, że poszedłem na łatwiznę, tylko że podjąłem słuszną decyzję.

Namioty rozstawiamy po nocy, koło 23:00. Mimo ciemności jakoś udało nam się znaleźć zaciszne miejsce u podnóża rosłych krzaków - istnieje cień szansy, że w nocy nie zdmuchnie nam naszych małych domków. Pozostała jeszcze kwestia gotowania obiadu. Po dramatycznych pierepałkach z palnikiem, który gasł co chwila, przyszedł czas na zrobienie herbaty. W tym momencie już skapitulowaliśmy i zostawiliśmy zapalony palnik na zewnątrz, a impreza przeniosła się pod prysznic. W oczekiwaniu na wrzątek Kasia zabawiała nas deklamując "Baśń o stalowym jeżu" :)

%% {cytaty dot. deszczu i wiatru (“jak Islandia zmienia perspektywę”) → fb}
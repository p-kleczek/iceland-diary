\chapter{21.07. - Podróż “tam” (prolog)}

%% TODO: zrobić opis

\section{Podróż pociągiem}

głowa przez okno - świst pociągu, pęd powietrza… czuć wakacje, ilekroć tak jadę!
sycę oczy widokami mijanych brzezinek, sosnownych lasów, wiosek i miasteczek - wiem, że już za parę godzin znajdę się na wyspie, gdzie przez wiele dni będę oglądał jedynie skalne rumowiska i wulkaniczne pustynie
to uczucie, że co można było zrobić, to się zrobiło, że etap przygotowań i oczekiwania dobiegł końca i że teraz już tylko pozostało realizować plan - nadeszła chwila działania!
Okęcie
ważenie na pustych stanowiskach odpraw + ogólnodostępna waga wielkogabarytowa
ok. 5 rowerów na Islandię, m.in. Greta z Australii oraz 3 Ukraińców (na początku myślałem, że Rosjan)

\section{Keflavik}

błądzenie nocą (bo skręciłem w złym miejscu), wizyta na komisariacie policji - “gdzie dalej?”, druk mapy ALE noc i dziwny układ ulic ⇒ błądzenie przez kolejną godzinę; ostatecznie w domu o 5:00, nocleg na siłowni (żeby nie budzić gospodarza i innych rowerzystów)
nie było miejsca dla Grety (choć w sumie mogłaby spać w Common Roomie)
pudła rowerowe w mieszkaniu Artura

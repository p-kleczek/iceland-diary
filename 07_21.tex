\chapter*{21.07. --- Ahoj, przygodo!}

\section*{Podróż pociągiem}

Pola uprawne, gęste lasy, wioski i miasteczka migają za oknami pociągu relacji Kraków--Warszawa. Skład pamięta lata 80-te, co w tym wypadku ma swój jeden niezaprzeczalny plus --- mogę otworzyć okno na tyle, by bez przeszkód wystawić przez nie głowę.

Pęd powietrza targa włosy, słońce przyjemnie grzeje twarz, czuć zapach sosnowego lasu\textellipsis Ilekroć tak jadę czuję, że oto zaczęły się wakacje! Tym razem szczególnie dużo czasu poświęcam na kontemplację mijanych krajobrazów, gdyż wiem jedno --- już za parę godzin znajdę się na wyspie, gdzie przez wiele kolejnych dni będę oglądał jedynie skalne rumowiska i wulkaniczne pustynie.

Wreszcie odczuwam spokój. Gdzieś zniknęło to ciągłe zastanawianie się ,,co by jeszcze przygotować?'' albo ''czy na pewno wszystko spakowałem?''. Etap przygotowań i oczekiwania dobiegł końca --- co można było zrobić, to się zrobiło, a teraz pozostało już tylko realizować plan. Nadeszła chwila działania!

\section*{Okęcie}

W Warszawie, a dokładniej na Warszawie Centralnej, czekała na mnie miła niespodzianka --- oto brat przybył specjalnie po to, by eskortować mnie z dworca na lotnisko. Przy tej ilości pakunków była to pomoc nieoceniona!

Na Okęciu czekał mnie jeszcze ostatni (przedwylotowy) stres --- ważenie bagaży i odprawa. Niby przed wyjazdem poszedłem na pocztę i sztuka po sztuce sprawdzałem wagę śpiwora, karimaty, osobno prawej i lewej sakwy, lecz czy gdzieś się nie pomyliłem? W końcu część rzeczy dopakowałem w ostatniej chwili\textellipsis Z ulgą odkryłem, że w paru nieużywanych aktualnie punktach odpraw nie zostały wyłączone taśmociągi-wagi --- mogłem w spokoju bawić się w przekładanie w tę i we w tę różnych pierdółek tak, by uzyskać regulaminowe pakunki. No i byłem pełen wdzięczności dla personelu lotniska po tym, jak pani w informacji poradziła mi, by ważyć pudło ze złożonym rowerem na ogólnodostępnej wadze dla bagaży wielkogabarytowych. Po wszystkich tych zabiegach okazało się, że do samolotu muszę wsiąść ubrany ,,na cebulkę'' --- ze skarpetkami poupychanymi w kieszeniach kurtki i~polara, w kalesonach i ciepłej czapce na sobie, z kilogramami kuskusu i innego jedzenia w~podręcznym plecaczku\textellipsis

\hint{Pamiętaj, że do wagi bagażu należy doliczyć również opakowanie. W moim przypadku płachta budowlana ważyła ponad 0,5 kg, a taśma pakowa i taśma zbrojona --- kolejne 300 g. Nie zapomnij też, by do bagażu, który będzie leciał w lukach bagażowych, przełożyć wszystkie przedmioty zakazane w kabinie dla pasażerów (scyzoryk, nożyczki itp.).}

W kolejce do odprawy naliczyłem bodaj pięć innych rowerów --- cztery były ładnie opakowane w kartony, a jeden szedł ,,luzem'' (nie był owinięty nawet folią). Właśnie ten rower ,,luzem'' nie przeszedł kontroli. Pani w okienku odmówiła przyjęcia go, więc pasażer zażądał rozmowy z kierownikiem. Jak się skończyła ta sytuacja --- nie wiem. Jednak zważywszy na to, że bagażowi na lotniskach zazwyczaj nie patyczkują się z ładunkiem (nawet z rowerami), to jednak najlepiej pójść do sklepu rowerowego, wziąć karton i poświęcić tę chwilę na lepsze zabezpieczenie swego ,,rumaka''. Sam na szczęście żadnych pierepałek z odprawą nie miałem i na dwadzieścia minut przed startem szczęśliwie znalazłem się w strefie odlotów.

Zająłem miejsce w fotelu, wyciągnąłem jakieś czasopismo z zamiarem umilenia sobie podróży, a tu nagle słyszę za sobą jakby rosyjską mowę. Jako że uczyłem się języka rosyjskiego przez ponad rok, starałem się wyłowić jak najwięcej z konwersacji trójki siedzących za mną ludzi. Jakież było moje zaskoczenie, gdy raz po raz słyszałem znajome słowo \emph{wełosiped} --- a zatem siedzę obok potencjalnych \emph{poputczików}, przygodnych towarzyszy podróży. Wmieszałem się do rozmowy, zrazu po rosyjsku, a gdy okazało się że oni są z Ukrainy, to szybko przeszedłem na angielski. Wiadomo --- konflikt na wschodzie Ukrainy, trudno wyczuć jakie jest ich nastawienie do języka Puszkina.

Efektem krótkiej konwersacji było uzyskanie wielu przydatnych informacji o stanie infrastruktury turystycznej --- w tym rowerowej --- w rejonie Zakarpacia. Współrozmówcy byli tam parokrotnie i podkreślali pozytywne zmiany, które zaszły w ostatnich latach: tworzenie bazy noclegowej, znakowanie szlaków z pomocą wolontariuszy (z Polski, Słowacji i~Czech), zmianę mentalności ludzi\textellipsis  Ziarno podróży w okolice Mukaczewa, Drohobycza i~Jaremczy zostało we mnie zasiane!

Niestety, tej sympatycznej trójki na pewno nie spotkam na trasie --- oni kierują się na północ, podczas gdy my na południe. Oni spędzą na Islandii dwa tygodnie, podczas gdy my --- miesiąc.

\section*{Keflavík}

Na lotnisko Keflavík dolecieliśmy zgodnie z planem, o północy miejscowego czasu. Nim dostaliśmy swoje bagaże, skręciliśmy rowery i założyliśmy wszystkie nasze pakunki (w tym kartony) ,,na pakę'' minęły dobre dwie godziny. W międzyczasie zdążyła już zapaść noc, a~do tego zaczęło mżyć.

Moje bagaże wyjechały trochę później i gdy reszta miała już wszystko gotowe do drogi ja wciąż jeszcze byłem ,,w proszku''. Dałem im więc zielone światło na wyjazd do Ásbrú. W tej niewielkiej mieścinie czekał na nas Artur --- kolega, u którego chcieliśmy zostawić na czas objazdu wyspy pudła na rowery i przenocować. Sam wyruszyłem dopiero w kwadrans później.

W nikłym świetle lampek rowerowych, walcząc z podmuchami wiatru i~zalewającą oczy wodą, potoczyłem się mozolnie w stronę wspomnianego Ásbrú. Jazdy w żadnym wypadku nie ułatwiało trzymane przeze mnie w prawej ręce pudło, które raz po raz odgrywało rolę płatu nośnego (czułem się jak samolot!). Cóż jednak było począć? Żadnym sposobem nie udało mi się go przytroczyć do bagażnika.

W połowie drogi zauważyłem drogowskaz ze skrętem na miejscowość Keflavík. Popatrzyłem na szkic trasy sporządzony przezornie przed wyjazdem --- wygląda na to, że mam tu skręcić. Ruszyłem w dół drogi, trzymając się tego, co zapamiętałem ze skrawka papieru, klucząc po coraz mniejszych osiedlowych dróżkach. A gdy po chwili zajechałem na podwórze jakiejś fermy, musiałem uznać się za pokonanego i przyznać przed sobą, że jednak się zgubiłem.

Gdy tak próbowałem dotrzeć do jakiegoś charakterystycznego punktu (bo przez zaparowane okulary pokryte szczelnie drobnymi kropelkami deszczu niewiele widziałem), zauważyłem nagle znajomy symbol ,,gwiazdy szeryfa''. \emph{Oho! komisariat policji!} --- pomyślałem i nie zastanawiając się długo postanowiłem złożyć stróżom prawa małą wizytę.

Posterunkowy okazał się przesympatycznym człowiekiem. Nie tylko powiedział mi gdzie się znajduję i jak dojechać tam, gdzie chciałem, ale też wydrukował mapę trasy z internetu i zaznaczył dalszą drogę. Podziękowałem i ruszyłem dalej. Niestety, nawet ta nowa mapa na niewiele się zdała --- dookoła noc, wciąż ciężko o punkty charakterystyczne, a układ drogowy miejscowości przypomina czyjąś radosną twórczość. Efekt? Błądziłem kolejną godzinę i ostatecznie dopiero o 5:00 dotarłem do akademika, w którym mieszka Artur.

O tej porze nie miałem już serca budzić śpiących domowników i reszty ekipy. Zadekowałem się w ogólnodostępnej siłowni, gdzie w kącie na podłodze rozłożyłem swój barłóg. Pierwszy dzień wyprawy --- pierwsza przygoda!
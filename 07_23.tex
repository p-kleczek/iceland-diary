\chapter*{23.07.}

Poranek przebiega spokojnie, gdyż każde z nas pragnie maksymalnie wykorzystać możliwości kempingu -- są więc tosty i herbata na śniadanie, potem długi ciepły prysznic… W trasę ruszamy o 10:30, akurat chwilę po tym gdy zaczął wiać silny wiatr ze wschodu. Jego podmuchy sprawiają, że nasza średnia prędkość spada do około 12 km/h --- no, ale przynajmniej nie pada! Ta pogoda pozwoliła nam odkryć nowe zastosowanie dla dużej polskiej flagi, którą wożę na maszcie przymocowanym do bagażnika --- pełni ona rolę wiatrowskazu i umożliwia ustawienie się w szyku.

\hint{Jazda w szyku to niezwykle istotny element wszelkich poważniejszych wypraw rowerowych. Pozwala oszczędzać siły, a w sytuacjach ekstremalnych --- szczególnie gdy wieje --- w ogóle pozwala zajechać w planowane miejsce! Szerokość szyku należy dostosować do warunków (żeby np. za nami nie tworzył się korek), warto jednak zajmować tyle pasa (i jechać na tyle daleko od jego krawędzi), by kierowcy nie wyprzedzali ,,na gazetę''. \newline  Więcej o tym zagadnieniu poczytasz tu: \url{http://bikestory.pl/jazda-rowerem-grupie/} oraz \url{http://www.agk-pruszkow.cba.pl/strona01.html}. }

Co jakiś czas napotykamy na zagubione przy drodze drogowskazy z jakimiś napisanym po islandzku i osobliwym symbolem --- herbem św. Jana (obecnie głównie kojarzonym z klawiaturami spod znaku Apple). Wtedy to zatrzymujemy się, Put wyciąga komórkę, włącza internet i sprawdza w Google Images jaką to atrakcje właśnie mijamy. Jeśli ze zdjęć wynika, że ma potencjał --- zbaczamy z trasy, jeśli nie --- odpuszczamy. Właśnie w ten sposób trafiliśmy na piękne klify Krýsuvíkbjarg, które oddalone są od \road{427} o parę kilometrów i nie sposób ich dostrzec z szosy.

\img{./photos/x-s-2014-07-23_15-03-31__12.jpg}{cliff_ford}{Czy te brody to już?}
\img{./photos/x-s-2014-07-23_15-44-30__18.jpg}{cliff_rock}{Młody człowiek i morze}

Kawałek dalej ponownie nadłożyliśmy kilometrów, by zahaczyć o gorące źródło Austurengjahver. Dotarcie do samego źródła wymaga trochę wysiłku --- \href{http://www.openstreetmap.org/way/33182596}{długi na 1,6 km spacer} skutecznie odstręcza większość ,,turystów'', więc na miejscu można w spokoju pomoczyć nogi w lekko błotnistej (lecz ciepłej i śmierdzącej jak każde inne gorące źródło) wodzie. Zdania ,,czy iść'' były w naszej ekipie mocno podzielone i niewiele brakowało, a też byśmy sobie podarowali ów spacer. Ostatecznie jednak, w demokratycznym głosowaniu (2 za, 1 przeciw, 1 wstrzymał się) projekt został ,,klepnięty''. Potem podjechaliśmy jeszcze pod Seltún, obszar geotermalny z ładnymi kolorowymi skałami.

\img{./photos/x-s-2014-07-23_18-26-49__25.jpg}{austurengjahver_legs}{Nie ma to jak okład z ciepłego błotka!}
\img{./photos/x-s-2014-07-23_18-29-35__27.jpg}{austurengjahver_panorama}{Swoiste OP-1 Puta nie chroniło go przed smrodem…}

Nocleg wypadł nam na kempingu w osadzie Strandarkirkja, około 14 km przed Þorlakshöfn. Ach, jakaż była nasza radość, gdy zobaczyliśmy znak \emph{FREE camping}, a po dojeździe na miejsce oczom naszym ukazała się olbrzymia łąka z soczyście zieloną trawą, solidne sanitariaty (z grzejnikiem w środku) oraz ,,wiatrołapem'' pełniącym rolę kuchni polowej… Prysznic był płatny (500 kr), lecz ciepła woda z umywalki w zupełności wystarczała do zachowania minimum higieny.

Krajobrazy dnia dzisiejszego: po prawej pastwiska i równina, po lewej masywne góry o niemal pionowych zboczach.

\img{./photos/x-s-2014-07-23_22-36-27__12.jpg}{strandarkirkja}{Kemping w Strandarkirkja}

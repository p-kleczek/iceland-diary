\chapter*{02.08. --- Dzień Turysty}

\hint{To bardzo ważne, by raz na czas --- po solidnym, długotrwałym wysiłku --- zrobić sobie dzień odpoczynku. Wpływa to korzystnie nie tylko na ciało, ale także na ,,ducha'' --- poprawia nastrój w drużynie.}

Cisnęliśmy bez przerwy od 11 dni, z czego przez ostatnie --- ostro pod wiatr, nocując w miejscach gdzie hulał lodowaty wicher, czasem bez możliwości odpowiedniego umycia się. Widać było, że forma i nastroje zaczynają siadać, częściej pojawiały się spięcia i konflikty, zaistniało ryzyko ,,zmęczenia materiału''. Dlatego też dzisiejszy dzień postanowiliśmy niemal w całości spędzić w Egilsstaðir, tak na luzie.

Wreszcie znalazł się czas, by na spokojnie zrobić pranie i zakupy, by pogawędzić z innymi turystami-rowerzystami (w tym z naszymi Holendrami), by napisać kartki do rodziny i znajomych, trochę nic-nie-robić i zjeść obiad bez pośpiechu. Dla chętnych znalazła się dodatkowa atrakcja w postaci basenu --- co prawda bez gorącej wody (z wyjątkiem dwóch ,,niecek''), ale za to z trzema pustymi torami. Ach\textellipsis nic tak nie rozluźnia mięśni pleców i nóg jak solidna porcja żabki i kraula!

\hint{Za wstęp na islandzki basen płaci się zazwyczaj około 500--600~kr, wejście jest bez limitu czasu (można siedzieć od rana do wieczora;) Nie ma wymogu zakładania czepka.}

Zachciało nam się przejechać dziś jeszcze takie symboliczne 30~km --- głównie po to, by nie płacić za kolejną noc na kempingu. Przed wyjazdem każde z nas wzięło jeszcze ostatni prysznic w ciepłej wodzie i o 18:30, przy dość ładnej pogodzie, ruszamy w dalszą trasę.

Pod Nettó, tuż obok kempingu, spotkaliśmy ponownie Panią Polkę. Tym razem miała ona ze sobą pełny ekwipunek --- na 29-calowym potworze zawisły wyładowane do granic możliwości sakwy (przednie i tylne), solidna torba na kierownicę (na olbrzymią lustrzankę), na bagażniku leżał sobie statyw tak solidny, że pewnie i kamerę telewizyjną by utrzymał\textellipsis Trzeba przyznać, że wzbudzało to respekt! Tylko jej opowieści nie brzmiały zbyt wiarygodnie, bo gdy ktoś twierdzi że przejeżdżał przez Öxi ledwie godzinę po tobie przy pięknej słonecznej pogodzie, a ty męczyłeś kilometry w lekkiej, niemal marznącej mżawce, to jak tu takiemu komuś wierzyć?!

\img{./photos/x-s-2014-08-02_12-14-29__66.jpg}{wool_market}{Standardowy asortyment typowego islandzkiego supermarketu}

Aż do okolic wiaduktu nad kanionem, którym płynie rzeka Jökulsá á Brú, \road{1} przypomina jazdę kolejką górską w wesołym miasteczku --- to pnie się kawałek w górę, to trochę opada, sumarycznie jednak zdobywa się wysokość. Za to ostatni zjazd, na wiadukt, jest długi i piękny. Stanowi istną nagrodę za wszystkie wcześniejsze trudy! Później droga wznosi się już bardzo łagodnie --- tak łagodnie, że Put był przekonany, iż jedzie wręcz po równym \wink

Obozowisko założyliśmy kawałek za miejscem, gdzie od \road{923} odbija jakaś nienazwana szutrówka. Dosłownie dwa metry od drogi znaleźliśmy dość równe trawiaste poletko, gdzie niby szpilki wchodzą w podłoże jak w masło, lecz równocześnie trzymają fest!

Rzut oka na licznik --- właśnie wybił nam 50. kilometr trasy od Egilsstaðir. Ot, taki sobie dzień turysty!

\chapter*{11.08.}

\curiosity{Czy wiesz, że butelkowaną islandzką kranówę\textellipsis --- o przepraszam, wodę źródlaną --- można nabyć w licznych punktach sprzedaży detalicznej w USA? (Tekst promocyjny z etykiety \href{http://icelandspring.com/}{icelandspring.com}.)}

Ostatnio powstał swoisty standardowy zestaw tematów do posiłków: ile kilometrów mamy do przejechania w poszczególnych dniach? ile jeszcze noclegów i gdzie one wypadają? ile śniadań i obiadów czeka nas przed najbliższym sklepem? co zjemy na który posiłek (zwłaszcza na obiad, bo trzeba wybrać typ zapychacza i wkładki mięsnej)? Zgoda, to ważne sprawy i należy je omówić, lecz ilość czasu którą obecnie na to poświęcamy jest olbrzymia.

Kolejny dzień, kiedy chociaż z rana świeci słońce --- nastraja to pozytywnie do pedałowania! Choć w sumie\textellipsis poranek taki piękny, a do przejechania tylko jakieś 70 km, więc wszystko toczy się iście emeryckim tempem (z leżeniem na trawie, opalaniem się oraz skakaniem na dużej dmuchanej trampolinie włącznie). Wyjeżdżamy na trasę późno jak nigdy (jeżeli nie liczyć dnia turysty :), bo około 13:00, lecz  wszyscy jesteśmy zadowoleni z tych miło spędzonych chwil.

\img{./photos/x-s-2014-08-11_15-13-41__330.jpg}{vermahlith_sun}{Miła odmiana od deszczu.}
\img{./photos/x-s-2014-08-11_17-49-02__336.jpg}{into_interior}{Ponownie wkraczamy w interior}

Zawitaliśmy jeszcze do supermarketu, by kupić ,,chleb i mleko'', co szybko przerodziło się w szał zakupów --- gdy wreszcie udało się wszystkie puszki z owocami i paczki ciastek poutykać po sakwach, te znów są konkretnie wypchane :)

Droga \road{733} oraz \road{F35} (na odcinku do Áfangi) to cud-miód-poezja: solidnie utwardzona (chyba jakimś środkiem wiążącym podłoże), łagodnie nachylona (począwszy od okolic elektrowni wodnej Blanda nie ma już stromych podjazdów) i\textellipsis tylko momentami trochę dziurawa (ale tak do przeżycia).

Po odbiciu z \road{733}, tuż za mostem, zapragnęliśmy przyrządzić kolejny polowy obiad. Doświadczenia z Askji zaprocentowały --- znów konstruujemy wiatrołap. Tym razem świeci słońce, nigdzie nam się nie spieszy, więc między innymi znalazł się czas, by Kasia przeszła pełnowymiarowy kurs otwierania konserw przy użyciu scyzoryka. Po obiedzie zgodnie stwierdzamy, że coraz ciężej uchwycić moment przejścia między uczuciem głodu, nasycenia i przejedzenia. Zazwyczaj objawia się to tym, że najpierw jemy i jemy, a po takim posiłku ledwo dajemy radę kręcić pedałami --- każdy stromszy podjazd grozi zwrotką.

\img{./photos/x-s-2014-08-11_21-47-43__230.jpg}{road}{Ech, ta droga\textellipsis}

Jazda w kompletnej ciszy to nowe, osobliwe doświadczenie. Nie mam na myśli tego, że tylko pedałujemy, nie odzywając się do siebie nawzajem. Nie, to nie tak! Samochodów brak, ludzi brak, owadów brak, flauta\textellipsis Bywało, że gdy przystawaliśmy (a więc gdy milkł chrzęst nienasmarowanych łańcuchów), aż zaczynało dzwonić w uszach!

Późniejszym popołudniem było tak upalnie i bezwietrznie, że nawet Kasia dołączyła do grona osób podróżujących w krótkim rękawku! Taki stan nie utrzymał się jednak długo, dokładniej --- urzymał się przez całe 10 minut\textellipsis A gdy później Put i Karolina polowali na szczycie wzniesienia przed Áfangi na zachód słońca, podmuchy wiatru były tak mocne i lodowate, że Kasia i ja skapitulowaliśmy (innymi słowy nie dotrwaliśmy do zachodu) i udaliśmy się do pobliskiego ,,kurortu'' w celu ogrzania zgrabiałych palców.

Áfangi nieco nas rozczarowało. Niby cena za nocleg przyzwoita, bo po 1200 kr od łebka (a prysznic i jacuzzi w cenie), lecz mimo to szału nie ma. Zdania ,,czy zostajemy'' są podzielone, zatem decyzję podejmujemy rzucając monetą. Choć z rzutu wyszło, że nie zostajemy, to i tak ostatecznym argumentem było to, że jedyny fragment trawiasty zdatny do rozbicia się znajdował się\textellipsis tuż obok stajni! Odjechaliśmy więc kilometr dalej i rozbiliśmy się parę metrów od \road{F35}, na względnie płaskim i nie-kamienistym skrawku ziemi.

\img{./photos/x-s-2014-08-11_23-25-53__350.jpg}{afangi_sunset}{Uroki jazdy o zmroku.}
\img{./photos/x-s-2014-08-12_00-52-07__354.jpg}{afangi_campsite}{Wzgardziliśmy noclegiem tuż obok stajni w Áfangi.}

\chapter*{13.08.}

Poranek pierońsko chłodny. Mimo polara i rękawiczek budzę się o 6:00 z zimna. Pozostali też, mimo wielu warstw i zawijania się w śpiwór, wielokrotnie budzili się w nocy. Kasia nawet ubierała się w półśnie, z sukcesem! Niebo zasnute jednolitą warstwą grubych chmur, ale silny północny wiatr jakby zaczynał je powoli rozganiać.

Nie chciało mi się wracać do namiotu (bo wiedziałem, że już nie zasnę), więc wybrałem się na spacer zieloną trasą turystyczną --- najkrótszą z trzech możliwych, liczącą jakieś 3~km. Przyznam jednak, że trochę się nią rozczarowałem --- właściwie żadnych nowych krajobrazów, form skalnych albo roślin\textellipsis Parę szczelin w ziemi, z których bucha para, ale poza tym szału ni ma. Według mnie największą atrakcją przyrodniczą Hveravellir jest, oprócz gorącego źródła, buchający gęstymi kłębami pary stożek. Zobaczenie go w zupełności wystarcza do zaliczenia programu zwiedzania.

\curiosity{Główny problem życia w Hveravellir? --- Brak zimnej wody! Czy jesteście w stanie uwierzyć, że są w górach na Islandii miejsca, gdzie wodę trzeba chłodzić w lodówce? Serio! Ze względu na podziemne gorące cieki wodne z kurka z ,,zimną'' wodą leci coś, co ma temperaturę koło $\text{20}^{\circ}\text{C}$!}

Dziś nawierzchnia na drodze ma standard pewnie dwie klasy niższy niż w poprzednich dniach --- bardzo dużo żwiru (na odcinku około 10~km), miejscami kamienie wielkości pięści. Dobrze że jest chociaż dość płasko, bo inaczej byłby kompletny dramat.

\img{./photos/x-s-2014-08-12_15-55-28__361.jpg}{on_the_road_again}{On the road again!}

Sprzęt zaczyna się sypać\textellipsis Wyrwał mi się fragment ,,pleców'' jednej z sakw --- akurat razem z hakiem, na którym wisi sakwa. Coś, co od biedy mogę określić mianem ,,naprawa'' zostało wykonane z użyciem parcianego pasa. Nie zabrałem sprzączki ani klamry, więc dokonuję cudów, by jakoś związać to wszystko na tyle solidnie, by sakwa nie latała więcej na wszystkie strony.

Droga od \road{35} do Asgarður okazała się być całkiem znośna. Obawialiśmy się, że czeka nas co najmniej 4~km pchania roweru, a tu: brodu brak, chamskiego żwiru (niemal) brak, tylko jeden naprawdę ostry podjazd\textellipsis Właśnie ten ostatni podjazd był naprawdę mylący. Wydawało nam się, że aby dojechać do kempingu, trzeba będzie wdrapywać się jeszcze drugie tyle, a tu po zjeździe na niewielkie wypłaszczenie okazuje się, że ,,o, to już tu!''!

W pewnym momencie na trasie straciłem możliwość zmieniania przedniej przerzutki. Była ona ustawiona na 2 i nie wchodziła ani na 1 ani na 3. ,,Ani chybi kwestia linki bądź sprężyny'' --- myślę, lecz uważniejsze oględziny wskazały faktyczną, zgoła inną i bardziej prozaiczną przyczynę. Okazało się, że do mechanizmu powpadało mi parę drobnych drobnych kamyków z drogi (łącznie trzy sztuki) i to one fizycznie blokowały wszelki ruch pantografu.

Na dziurze tuż za ostatnim mostem przed Asgarður straciłem dwie szprychy w tylnym kole. Szczęśliwie urwały się tylko łepki nypli i tym samym wiedziałem, że naprawa tych zniszczeń zajmie góra 15--20~minut. Inaczej byłaby zabawa ze zdejmowaniem kasety i przeplataniem szprychy przez tarczę hamulca (a ja, głupi, zapomniałem wziąć śrubokręt do torxów bądź wymienić te śrubki na takie pod imbusy)! Cóż\textellipsis na chwilę obecną koniec ,,rumakowania'' :)

\img{./photos/x-s-2014-08-13_16-26-43__241.jpg}{german_polish_horses}{Dwa rumaki --- niemiecki i polski}

Zaraz po dojeździe na kemping --- obiad, a potem --- w góry! Postanowiliśmy przejść się pętlą wokół doliny Hveradalir, która nosi miano ,,prawdopodobnie najpiękniejszego obszaru geotermalnego na Islandii''. Jeśli starczy nam czasu i ochoty, wdrapiemy się jeszcze na najwyższy szczyt w okolicy --- Snækollur (1477~m~n.p.m.)

W drodze do doliny napotkaliśmy liczną młodzieżową grupę turystów górskich z USA. Któryś z chłopaków spytał nas ,,skąd my?'', a gdy odparliśmy że z \emph{,,Poland''}, zaraz zawołał \emph{,,Dżoana! Here are folks from Poland!''}. Joanna okazała się być córką polskich emigrantów i tak się ucieszyła, że może rozmawiać po polsku, że przegadaliśmy dobre 10~minut\textellipsis Czego ciekawego się dowiedzieliśmy? Ano na przykład tego, że idą tydzień z buta przez Kjölur, z całym ekwipunkiem na grzbiecie. Smaczku dodaje fakt, że około 5~z~30~osób nigdy nie biwakowało (choćby przez jeden dzień)! Szaleńcy\textellipsis

\img{./photos/x-s-2014-08-13_21-18-39__253.jpg}{near_hveradalir}{W okolicach Hveradalir (w tle Snækollur)}
\img{./photos/x-s-2014-08-13_20-47-40__249.jpg}{hveradalir_moss}{Mech. Kolory oryginalne.}

Osiągnięcie rejonu Hveradalir okazało się trudnym zadaniem, szczególnie ze względu na fatalne znakowanie szlaku. Dość powiedzieć, że części tyczek, schowanych gdzieś między skałkami, w ogóle nie było widać! Ponieważ wiemy już, że na Snækollur nie zajdziemy, a ja pragnąłem popodziwiać trochę ładnych panoram, więc u podnóża szczytu Mænir rozdzielamy się. Po raz kolejny na czworaka przeprowadzam atak szczytowy po tłuczniu, ponownie podczas powrotu stosuję standardową kombinację zsuwania się i zbiegania. Czy było warto? No, tym razem mam wątpliwości. Z jednej strony owszem, widoki niczego sobie (nawet gdzieniegdzie jakby ocean był widoczny), z drugiej strony --- gdy zobaczyłem na aparacie Puta bajkowe zdjęcia z Hveradalir, to faktycznie tam było pięknie i inaczej niż w innych (dotychczas mijanych) obszarach geotermalnych: dnem snują się opary, ściany doliny porastają jaskrawozielone mchy, kolor skał to pełna paleta barw, a w ramach dodatku do widoków można jeszcze pomoczyć nogi w ciepłych strumieniach.

W ośrodku wczasowym Kerlingarfjöll już trzeci sezon pracuje Polka ze Szklarskiej Poręby. Jak mnie ,,wyhaczyła''? Nie inaczej --- dzięki fladze na rowerze! Z jej opowieści wynikało, że i ona była zadowolona z dotychczasowej współpracy, i Islandczycy byli zadowoleni, więc tym razem przyjechała ze swoim chłopakiem --- oboje do pracy. W sumie w takim miejscu to można popracować ze dwa tygodnie, w wolnych chwilach robiąc wycieczki po okolicy. Szlaków od groma, spokojnie da się z nich sklecić cztery ładne pętle.
%TODO: zdanie wprowadzania do tego akapitu
%TODO: synonim do <wyhaczyła>
%TODO: <tej współpracy> → jakiej?

\img{./photos/x-s-2014-08-13_22-13-19__396.jpg}{hveradalir_panorama}{Hveradalir w pełnej krasie.}
\img{./photos/x-s-2014-08-13_19-49-15__97.jpg}{hveradalir_feet}{Ulga dla zmęczonych stóp zapewniona.}

Krótki raport meteo: kolejny dzień, gdy właściwie przez cały czas świeciło słońce, a wiatr wiał w plecy. Zaczynam wierzyć w magię Kjöluru! Niestety, zimne podmuchy nie zawsze pozwalają na jazdę w krótkim rękawie.

Coś się dziś przy kolacji rozmarzyliśmy i stwierdziliśmy, że ,,kiedyś, gdy będziemy bogaci, przyjedziemy na Islandię solidnymi jeepami, będziemy spać po chatach, jeździć na koniach, pić piwo w cenie 25~zł za półlitrową puszkę i ogólnie korzystać z życia!'' Na chwilę obecną jeździmy rowerami, nocami marzniemy w namiotach i liczymy każdy grosz.

Chwilę po 21:00 jesteśmy już w wszyscy razem w bazie. Na króciutkiej naradzie sztabowej podejmujemy decyzję o odjechaniu paru kilometrów w stronę \road{35}, w sam raz na tyle, by nie płacić za kemping. Pech chciał, że teren na którym się rozbijamy --- koło pierwszych wodospadów, licząc od strony Kerlingarfjöll --- tylko z drogi wyglądał na przyjazny. W rzeczywistości miejscami było to raczej bagienko, skutkiem czego buty znów nadają się do suszenia.

\hint{W miejscach szczególnie narażonych na ,,ciągnięcie od ziemi'' sprawdza się patent z folią NRC --- wystarczy wyłożyć nią podłogę namiotu i od razu robi się cieplej!}

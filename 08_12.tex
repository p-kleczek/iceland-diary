\chapter*{12.08. --- Hveravellir}

Pobudka o 10:30. Właśnie ustanowiony został nowy rekord! Dziś jednak była to nawet okoliczność korzystna, gdyż koło 8:00 całe niebo zasnuwały szczelnie chmury. A tak, gdy wyruszaliśmy o 12:30, mogliśmy już podziwiać pierwsze nieśmiałe prześwity w tym całunie.

Czekająca nas trasa to czysta rekreacja --- niecałe 40~km, z czego pierwsze 30~km po takiej nawierzchni jak dzień wcześniej. Potem mało przyjemny podjazd po kamykach i jeszcze gdzieś króciutki odcinek po kamerdolcach. Można spokojnie podziwiać krajobrazy, a te są niecodzienne. Jedziemy w końcu przez płaskowyż Kjölur, zatem jak okiem sięgnąć rozciągają się gładkie łączki i niewielkie pustynie i tylko gdzieś na horyzoncie ciemnieją zęby łańcuchów górskich oraz skrzą się w słońcu czapy lodowców.

Zalewie po trzech godzinach docieramy do celu. Błyskawicznie przygotowujemy obiad, a zaraz potem  zaczynamy grzanie tyłków w ,,basenie'' z gorącą wodą. Moczenie trwało do wieczora --- siedzieliśmy tak długo, że nawet załapaliśmy się na wschód księżyca\textellipsis

W Hveravellir po raz kolejny okazało się, że flaga pełni funkcję towarzyską. To właśnie dzięki niej poznałem Piotrka, z ,,zawodu'' słuchacza Akademii Marynarki Wojennej w Gdyni, a z zamiłowania --- obieżyświata. Rozmawialiśmy o wielu interesujących rzeczach, wymienialiśmy się licznymi praktycznymi radami odnośnie wypraw, lecz nie będę ich przytaczał w tym miejscu (zebrałem je w stosownym dziale). Wspomnę tyko o tym, jak to się stało, że go tam spotkaliśmy: otóż Piotrek przemierzał stopem Islandię i akurat nocleg wypadł mu tu --- w Hveravellir. Zagadnął właściciela kempingu, czy mógłby coś pomóc ,,w obejściu'' w zamian za wikt i opierunek, a ten dał mu fuchę w postaci wymiany rur od gorącej wody. Nazajutrz, na odchodnym, spytał Piotrka, czy nie chciałby jeszcze ich zutylizować, na co ten ochoczo przystał. Po dwóch dniach Piotrek otrzymał propozycję pomocy jeszcze przez tydzień, w zamian za 500~€. W ten oto sposób nasz łazik zdobył fundusze na pokrycie niemal całej swojej islandzkiej eskapady!

\img{./photos/x-s-2014-08-13_12-13-18__234.jpg}{hveravellir_crowded}{Hveravellir okazało się być popularnym miejscem\textellipsis}
\img{./photos/x-s-2014-08-13_12-42-05__235.jpg}{hveravellir_natural}{Dwie główne atrakcje (no\textellipsis może poza ,,jacuzzi'')}

Podczas wieczornej narady klamka zapadła --- zmieniamy plany i zamiast prosto do Gullfossu, jedziemy jutro całą grupą do kempingu u podnóża Kerlingarfjöll. Przez moment rozważaliśmy jeszcze taki wariant: ja i Karolina jedziemy dziś wieczorem do Kerlingarfjöll i tam śpimy, jutro z rana idziemy w góry i spotykamy się znów w komplecie na rozdrożu. Lecz skoro wszyscy przystali na opcję z Kerlingarfjöll, to nie będzie konieczne takie kombinowanie! Jedyny problematyczny aspekt to zaopatrzenie w żywność --- nie przewidzieliśmy dodatkowego dnia w interiorze, a po drodze sklepów brak. Co najwyżej można zjeść coś w restauracjach, ale to średnio nam się uśmiecha ze względu na duuużą marżę.

Dzisiejszy dzień był najluźniejszy i według mnie najpiękniejszy podczas całego naszego wyjazdu --- mam na myśli zarówno pogodę, jak i widoki oraz atrakcje na trasie. Nie, nie przygody, a prawdziwe atrakcje --- choćby widok stada stu koni pędzonych po ,,stepie''.

Dziś każde z nas jechało na dobrą sprawę ,,na własną rękę'', swoim tempem, nierzadko w odległości kilometra od wcześniejszej osoby --- ale to też jest potrzebne! Ciężko robić przez tak długi czas wszystko kolektywnie. Podejrzewam, że każdy potrzebuje od czasu do czasu odpoczynku od innych, zwłaszcza że od trzech tygodni funkcjonujemy właściwie w tym samym ,,sosie''. Oczywiście trzeba podkreślić fakt, że warunki atmosferyczne i trudność trasy na to pozwalały --- nie istniało ryzyko, że ta ostatnia osoba ,,zaginie w akcji''.

Po basenach, tak przed spaniem, Karolina i Put szarpnęli się na piwo (1000~kr za puszkę to solidny wydatek!). Nie ma się jednak czemu dziwić, skoro byliśmy w jednym z nielicznych miejsc, w którym dało się kupić prawdziwe piwo, a nie jakieś siki\textellipsis

Po raz pierwszy idziemy spać ,,na głodnego''. Nie dopychamy się na dobranoc chlebem, bo z obliczeń wyszło nam że wtedy pojutrze zabraknie nam go w trasie. Tylko Put stwierdził, że niesamowicie go przysysa i że dlatego musi zamówić w barze coś do jedzenia oraz gorącą czekoladę.

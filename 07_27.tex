\chapter*{27.07. --- Góry, wulkany i pustkowia}

\section*{Wypad do Þorsmörk}

Cała ta eskapada odbywała się na wariackich papierach --- zaniżone odległości, optymistyczne szacowanie czasu przejazdów, założenie, że ,,wszystko przebiegnie po mojej myśli''\textellipsis Pierwsze zaskoczenie przyszło, gdy po 20~km wciąż nie widziałem żadnego skrętu na Skógar. Po 25~km --- już mocno zaniepokojony --- pytam o drogę w jakiejś ,,chacie''. Mieszkająca tam dziewczyna łamaną angielszczyzną instruuje mnie, że muszę pojechać jeszcze parę kilometrów dalej (czyli droga wydłużyła się o 10~km). Dodaje też, że wraz z siostrą też tam jutro idą --- może się spotkamy.

Jakimś psim swędem, już po nocy, dotarłem wreszcie do Skógar. Tam jednak znów zbłądziłem, więc zrobiłem jakiemuś losowemu Islandczykowi ,,backdoor'' na werandę i otrzymałem kolejną porcję wskazówek: \emph{wróć się, pojedź za stare obory, tam będzie furtka i żwirowa droga biegnąca ostro w górę --- jedź za nią}.

Sam początek drogi był tragiczny: takie kamerdolce, że nawet pchać rower było ciężko! Później zrobiło się już trochę lepiej, lecz wciąż na przemian trochę jazdy 6~km/h (bo drogę zalegają kamienie wielkości pięści) i trochę pchania. Całe szczęście, że choć niby był to środek nocy (1:00), to jednak słońce do końca nie zaszło, stąd od biedy nie musiałem nawet korzystać z czołówki --- z pewnym wysiłkiem, ale dostrzegałem wszystkie przeszkody leżące na drodze.

W połowie drogi do schroniska Baldvinsskáli spotkała mnie niemiła niespodzianka --- dalszą drogę przegradzała lodowcowa rzeka. Niby był bród, ale bardziej dla monster-truc\-ków niż dla rowerów. Usiadłem zrezygnowany, zastanawiając się co dalej --- ryzykować samotną przeprawę przez lodowaty nurt czy dać sobie spokój i wrócić do Skógar? Chcąc kupić sobie trochę czasu poszedłem za potrzebą i\textellipsis dostrzegłem nieopodal przerzuconą przez potok kładkę dla turystów pieszych! Tuż za nią stała tablica z olbrzymią mapą terenu (i rozrysowanymi na niej szlakami turystycznymi), której nie omieszkałem sfotografować. Nie dysponowałem bowiem fizycznie jakąkolwiek mapą papierową, a jedynie tym, co zapamiętałem z mapy samochodowej w skali  1:300~000.

\img{./photos/2014-07-27_03-43-15.jpg}{baldvinsskali}{Checkpoint \#1 -- schronisko Baldvinsskáli}

Równo o 4:00 dojechałem pod pierwsze schronisko, zostawiłem rower i ruszyłem niebieskim szlakiem do Þorsmörk. W plecako-worku miałem zapas żywności i wody, a do tego: kalesony, rękawiczki, opaskę, polar, kurtkę\textellipsis Choć w Skógar było dość ciepło, to 1000~m różnicy wzniesień i bliskość lodowców sprawia, że w nocy w górze temperatury oscylują koło 5~°C. Początkowo martwiłem się, czy znajdę drogę wśród tych pustkowi, gdyż droga wyznaczona była tylko rachitycznymi palikami z końcówką pomalowana odpowiednio na czerwono lub niebiesko (w zależności od koloru szlaku). Przestałem się martwić, gdy w~pewnym momencie te listewki ustapiły miejsca eleganckim, solidnym, żółtym słupom wbitym w regularnych odstępach.

Poranne widoki były obłędne --- czyste niebo, podnoszące się z dolin mgły, skrzące się w ostrym słońcu płaty śniegu, formacje z pumeksu, ,,pustynie'' z popiołu\textellipsis

Koło 6:30 zawitałem na pole biwakowe Básar,  znajdujące się u wrót Þorsmörk, gdzie mogłem wreszcie nabrać wody (i skorzystać z kulturalnej toalety;) Ponieważ 0,5~l napoju w~bidonie okazało się ilością niewystarczającą, wygrzebałem z pojemnika na butelki PET jedną ,,sprawną'' sztukę i zrobiłem z niej drugi bidon. Naprawdę, pomiędzy Básarem a Baldvinsskáli właściwie nie ma miejsca, gdzie można by uzupełnić wodę! A trzeba mieś świadomość, że to ponad 700~m podejścia na dystansie ponad 10~km.

Czasu na powrót miałem skrajnie mało, gdyż umówiłem się z pozostałymi o 11:00 pod Skógafossem, stąd przebiegał on w błyskawicznym tempie --- przypominał bardziej górski marszobieg. Znalazłem jednak chwilę, by zahaczyć o drugie schronisko, to na samym Fimmvörðuháls. Owa chatka ma swój niepowtarzalny klimat i jeśli myślisz o pieszej wędrówce po tamtej okolicy (a do tego stać cię na zapłacenie 5000 kr za noc) --- gorąco je polecam! Schroniskiem zarządza \href{http://www.utivist.is/english}{Utivist}, takie islandzkie PTTK.

Na koniec spotkało mnie jeszcze jedno rozczarowanie --- zjazd odbywał się z niewiele większą prędkością niż wyjazd! Ach\textellipsis gdyby tak mieć takie opony jak mijające mnie monstertrucki\textellipsis Albo gdybym nie wiózł na bagażniku całego dobytku i nie obawiał się o uszkodzenie bagażnika albo złapanie gumy\textellipsis

\img{./photos/2014-07-27_04-58-49.jpg}{caldera}{Poranek na stokach Eyjafjallajökull}

\img{./photos/2014-07-27_07-41-16.jpg}{brocken_spectre}{Widmo Brockenu. Widzę je po raz pierwszy w życiu, więc umrę w górach.}

\clearpage

W połowie drogi spotkałem dziewczyny, z którymi rozmawiałem w nocy. Jak się okazało, były to Polki pracujące w wakacje w Skógar, w branży turystycznej. Opowiedziały mi między innymi pewną interesującą rzecz --- ich szef regularnie funduje im atrakcje w stylu: wypad na lodowiec, bilety autobusowe z Þorsmörk (niby trasa liczy raptem 40~km, lecz bilet kosztuje ze 100~zł), rejs na Wyspy Zachodnich Ludzi\textellipsis Czy ktoś słyszał o czymś podobnym u nas, w Polsce?

\processifversion{PDF}{
\hint{Dokładna relacja z 2-dniowej wyprawy (nie, nie mojej:) szlakiem ze Skógar do Þorsmörk znajduje się na stronie \url{http://adrian-harvey.com/2012/07/21/the-fimmvorduhals-diary/}, natomiast taki opis w~wersji skróconej można przeczytać na stronie \url{http://www.volcanohuts.com/fimmvorduhals}
}

\processifversion{HTML}{
\hint{Dokładna relacja z 2-dniowej wyprawy (nie, nie mojej:) szlakiem ze Skógar do Þorsmörk znajduje się \href{http://adrian-harvey.com/2012/07/21/the-fimmvorduhals-diary/}{tutaj}, natomiast \href{http://www.volcanohuts.com/fimmvorduhals}{tu} można poczytać opis w wersji skróconej.}
}



\section*{Pustkowia po raz pierwszy}

W Vík przygotowaliśmy pierwszy podczas naszej wyprawy obiad polowy --- pierwszy, bo w poprzednich dniach po prostu nie było sprzyjających warunków do gotowania na świeżym powietrzu! Tym razem przycupnęliśmy we wnęce w ścianie marketu i tak --- osłonięci od wiatru --- spokojnie gotowaliśmy. Ludzie patrzyli się na to zaintrygowani, lecz nikt nie próbował nas stamtąd wyrzucać :)

Droga za Vík to coś niewyobrażalnie nużącego: pola, pola, aż po horyzont pola łubinu\textellipsis Żadnej drogi w bok, żadnego zabudowania, tylko od czasu do czasu miejsce postojowe.

\img{./photos/x-s-2014-07-27_18-43-10__49.jpg}{vik_fields}{Znużeni jeźdźcy gdzieś w szczerym polu za Vík}

Nocleg zaplanowaliśmy na kempingu w Hrífunes. Kiedy jeszcze jechaliśmy po \road{209}, wszystko wyglądało w porządku --- przy drodze stał elegancki znak z symbolem kempingu. Problem pojawił się w momencie, gdy 50~metrów dalej ścieżkę przegradzał sznurek, na którym zawieszono kawałek dykty z napisem ,,LOKAÐ'' (czyli po naszemu ,,zamknięte''). Wizja lokalna w zapyziałym budynku sanitariatów wykazała, że kemping przestał funkcjonować dobrych parę lat wcześniej. Nie było jednak sensu jechać gdziekolwiek indziej, więc rozbiliśmy się tuż przy wspomnianej tabliczce. Szczęśliwie dla nas akurat trafiła się tam mała polanka, w sam raz na dwa namioty.

W okolicy śpią też bracia Rosjanie, którzy podobnie jak i my nastawiali się na nocleg na kempingu, co w kontekście konfliktu na Ukrainie stało się przyczynkiem do mało wybrednych żartów: \emph{,,Pamiętajcie, żeby uważać na gaz!''}}, \emph{,,Pilnujcie dobrze rowerów\textellipsis i zegarków!''} bądź też \emph{,,Ciekawe, czy nas zaanektują\textellipsis''}\footnote{Od autora: Żeby nie było, ja do Rosjan przyjeżdżających do Europy na rowerze nic nie mam :)}
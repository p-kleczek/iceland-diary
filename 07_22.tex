\chapter*{22.07.}

\section*{Ostatnie sprawunki}

\indent Przed wyruszeniem we właściwą trasę musieliśmy jeszcze załatwić kilka sprawunków w Keflavíku. Chodzi o wszelkie rzeczy, których z różnych względów ,,formalno-prawnych'' po prostu nie dało się załatwić w Polsce.
Najpierw zawitaliśmy o stację benzynową Olís \footnote{\href{https://www.google.com/url?q=https\%3A\%2F\%2Fmaps.google.com\%2Fmaps\%3Fq\%3D63.979816\%2C-22.54672}{mapa z zaznaczoną stacją benzynową}}, by zakupić naboje do palników gazowych.

W sumie nie było takiej konieczności --- równie dobrze mogliśmy skorzystać z darmowych butli zostawianych przez odlatujących turystów na kempingach nieopodal lotniska (tj. w promieniu 50 km). Podobno na kempingu w Garður --- tuż obok lotniska --- leżą całe hałdy tych naboi i to np. w 2/3 pełne.
Następnie zahaczyliśmy o warsztat wulkanizacyjny \footnote{\href{https://www.google.com/url?q=https\%3A\%2F\%2Fmaps.google.com\%2Fmaps\%3Fq\%3D63.982619\%2C-22.546328}{mapa z zaznaczonym zakładem wulkanizacyjnym}}, gdzie sprawnie dobiliśmy opony do ponad 4 atm.

\hint{Naboje gazowe można próbować ,,upolować'' na kempingach znajdujących się w pobliżu lotniska (czyli w promieniu około 50 km, np. w Garður albo w Grindavíku). Zdarzają się nawet naboje w 2/3 pełne!}

\hint{Do jazdy po asfalcie lepiej mieć opony naprawdę twarde, gdyż dzięki temu znacznie zmniejszają się opory toczenia. Mówiąc po ludzku --- zajedziesz dalej, szybciej, mniej się męcząc :)}

Kolejnym punktem programu był supermarket sieci Bonus --- bodaj najtańszej na Islandii. Nie będę się tu rozpisywał za bardzo o tym, w jaki zachwyt wprawiły nas ceny produków i ich wybór, gdyż \href{http://www.roboppy.net/food/2009/04/iceland-day-1-part-ii-reykjavik-bonus-supermarket-skyr.html}{to zrobili już inni przed nami}. Wspomnę tylko o jednym naszym odkryciu --- chodzi o przecier ananasowy marki EuroShopper. Trzypak puszek, każda po 227 g (z czego 70\% to ananas, a reszta --- sok ananasowy, a nie syrop) kosztował… 40 kr! Czyli 60 kr --- jakieś 1,50 zł --- za kilogram. Aż żal nie kupić! Te ananasy doskonale pełniły rolę deseru --- oczywiście gdy tylko Bonus był pod ręką…

\img{./photos/x-s-2014-07-22_14-53-31__2.jpg}{keflavik_metal_guys}{Fantazja Islandczyków --- uliczni tancerze…}

%\begin{figure}[h]
%\centering
%\begin{subfigure}{.5\textwidth}
%  \centering
%  \includegraphics[width=.9\textwidth]{./photos/2014-07-22_14-53-31__2.jpg}
%  \caption{A subfigure}
%  \label{fig:keflavik_metal_guys}
%\end{subfigure}%
%\begin{subfigure}{.5\textwidth}
%  \centering
%  \includegraphics[width=.9\textwidth]{./photos/2014-07-22_14-54-31__3.jpg}
%  \caption{A subfigure}
%  \label{fig:keflavik_breakfast}
%\end{subfigure}
%\end{figure}

Posileni, pełni sił, ruszyliśmy do centrum Keflavíku, do oddziału Landsbanki \footnote{\href{https://www.google.com/url?q=https\%3A\%2F\%2Fmaps.google.com\%2Fmaps\%3Fq\%3D63.995522\%2C-22.548067}{mapa z zaznaczonym oddziałem Landsbanki}} --- takie PKO --- by wymienić trochę waluty. Niby na lotnisku był bankomat, ale nietóre karty średnio z nim współpracowały, a do tego dziewczyny chciały wymienić trochę euro w gotówce na korony…

\hint{Na Islandii naprawdę wszędzie można płacić kartą. Nawet na kempingach w środku interioru! Jedyne, co warto mieć ze sobą, to żelazną rezerwę monet o nominale 100 kr --- na wielu kempingach zainstalowane są ,,dozowniki ciepłej wody'', które przyjmują 5x 100 kr i w zamian umożliwiają wzięcie 5-minutowego ciepłego prysznica. Czasem da się rozmienić banknoty na monety na kempingu, lecz w myśl zasady ,,przezorny zawsze ubezpieczony'' znacznie lepiej zrobić to zawczasu w jakimś sklepie czy stacji benzynowej.}

Ostatnim puntem był zakup czegoś, co umożliwi nam dzwonienie z islandzkiego numeru komórkowego oraz korzystanie z internetu --- wybór padł na Vodafone (biuro tej firmy mieściło się koło polskiego marketu), który oferował 3 GB danych za około 50 zł. O ile bez problemu kupiliśmy prepaid, o tyle aktywacja internetu wymagała już więcej zachodu, bo nie dość, że operacja odbywała się poprzez infolinię Vodafone, to jeszcze niezbędne było posiadanie karty kredytowej (szczęśliwie mieliśmy w odwodzie w Polsce posiadacza takowej). A internet miał nam służyć nie tylko do zabawy facebookiem, ale o tym później…

\hint{Zawczasu, przed wyjazdem z Polski, zorientuj się kto z rodziny / przyjaciół / znajomych posiada kartę kredytową i będzie skłonny podać ci jej numer, datę ważności oraz kod CVV. Istnieje pewna grupa usług --- np. aktywacja pakietu internetowego, zakup niektórych biletów online --- których nie da się załatwić przy użyciu zwykłej karty debetowej!}

Keflavík opóściliśmy ostatecznie o 15:00, przy (wciąż jeszcze) słonecznej pogodzie.

\hint{Islandia to olbrzymi obszar i ciężko zawczasu dowiedzieć się o każdej atrakcji, każdym interesującym miejscu na trasie. Warto korzystać więc intensywnie z informacji turystycznych, zbierać (i przeglądać!) foldery reklamowe oraz pytać innych turystów (najlepiej też rowerzystów, bo ci będą w stanie np. przestrzec cię przed trudnościami terenowymi itp.). Keflavík posiada Centrum Informacji Turystycznej --- lokalizację znajdziesz na ich profilu na Google+ (\url{https://plus.google.com/111960041675886065623/about?gl=pl\&hl=en}).}

\section*{Pierwsze godziny w trasie}

Pierwsze kilometry i już mocne zderzenie z islandzką naturą --- jedziemy przez pustkowia. Po lewej, po prawej, jak okiem sięgnąć lawa i niewielkie skałki. Nic dziwnego, że obszar ten --- jako jeden z trzech na Islandii --- służył NASA do prowadzenia treningów dla astronautów przed misją Apollo.

\img{./photos/x-s-2014-07-22_18-05-39__5.jpg}{first_hours_on_iceland}{Pustkowia półwyspu Reykjanes}

\img{./photos/x-s-2014-07-22_18-25-14__8.jpg}{neptune_monument}{Pomnik planety Neptun}

Kawałek przed Hafnir przyplątał się do naszej grupy pies --- ochrzczony Posejdonem (od \href{https://www.facebook.com/120832791270880/photos/a.612815058739315.1073741825.120832791270880/612815132072641/?type=3&theater}{pomnika planety Neptun}, który stał w miejscu zdarzenia) --- który biegł za nami aż do \href{http://www.visitreykjanes.is/searchresults/attraction/bridge-between-continents}{Kładki Między Kontynentami}. Nie byłoby w tym nic aż tak specjalnego, lecz Posejdon urzekł nas swym polowaniem na samochody: gdy zauważył nadjeżdżający pojazd, zaczajał się na poboczu i potem w ostatniej chwili wypadał na jezdnię --- tuż przed maskę --- głośno ujadając. Albo niesamowicie odważny albo niesamowicie głupi ;-)

Już daje nam się we znaki wiatr oraz liczne (na szczęście krótkie) stromsze podjazdy. Walka z takim kombo jest szczególnie ciężka dla tych osób, które nie jeździły ostatnio za wiele.

\hint{Podczas jazdy na rowerze pracują odrobinę inne mięśnie niż np. podczas wycieczek górskich. Bieganie, pływanie, orbitrek --- wszystko to oczywiście zwiększa wydolność organizmu, lecz nic nie zastąpi odpowiedniej liczby przejechanych kilometrów! :)}

Słońce było tylko na zachętę --- szybko się schowało za chmury i zrobiła typowa islandzka aura --- z mniej lub bardziej intensywnymi opadami deszczu… A trzeba wiedzieć, że tu nie ma się gdzie schować --- żadnych przydrożnych sklepów, barów, nawet wiat PKS-u. Tak więc siłą rzeczy --- mimo deszczu jedziemy dalej. Zresztą… nie lało tylko kropiło, czyli nawet nie byłoby sensu przeczekiwać tego --- klimat tu bowiem taki, że możnaby się nie doczekać momentu ,,wypadania''.

\section*{Grindavík}

Będąc już na obrzeżach Grindavíku powstał problem --- szukać miejsca na nocleg już teraz, czy może jechać jeszcze kawałek dalej? Cała ekipa była dość jednomyślna: \emph{Nie, nie jedziemy już dalej --- jest za późno (była 19:00 --- red.) no i warto się nie zarżnąć od razu pierwszego dnia. Lepiej wyruszyć wcześniej nazajutrz, ale przynajmniej dziś odespać podróż!} Tu konieczne jest słowo komentarza --- Berlińczycy nie spali ani w Polskim Busie, ani koczując na lotnisku Schönefeld ani też w samolocie. Zatem byli non-stop na chodzie przez 30 godzin, ze względu na późny przylot i długie czekanie na bagaż noc w Ásbrú też była zarwana… Tak więc po przejechaniu 50 km rozbiliśmy się na kempingu w Grindavíku.

Kemping można zaliczyć do klasy ,,all inclusive'', bo posiada ,,świetlicę'' (ogrzewanie podłogowe!) z aneksem kuchennym z pełnym wyposażeniem (sztućce, garnki, płyty indukcyjne, toster…), a prysznice są w cenie. W kuchni cała jedna szafka była zajęta przez ,,free stuff'' --- nie tylko rzeczy spożywcze, ale też gospdarcze (np. są wspomniane naboje gazowe).

\img{./photos/x-s-2014-07-23_11-11-19__6.jpg}{grindavik_cantine}{Kantyna kempingu w Grindavíku}

\hint{Będąc w kuchni na kempingu warto przeglądnąć cały asortyment opatrzony napisem ,,free food'' i ,,free stuff'', gdyż często można tam natrafić na towary luksusowe --- porządną herbatę, konserwy, słodycze, płatki śniadaniowe… Należy jednak zwracać uwagę na termin przydatności do spożycia oraz organoleptycznie zbadać faktyczny stan produktu (np. otwartego mleka lepiej nie tykać ;-)}

Na obiadokolację z radością pochłaniamy spaghetti z mięsem ze słoika (polskie, domowe, zawekowane mięso --- mniam!), sosem i warzywami z puszki. Przekąszamy ciasteczkami i niezwłocznie udajemy się na spoczynek --- większość z nas nawet bez mycia się…

\hint{Kupując warzywa i sosy w puszcze zwróć uwagę na ,,gęstość'' produktu, tj. stosunek wagi netto (po odcieku) do wagi brutto --- bo po co wozić ze sobą puszkowaną wodę?! Na Islandii bezkonkurencyjna pod tym względem okazała się mieszanka warzyw Euroshopper, w której wspomniany współczynnik wynosił około 0,95.}

Ponieważ mamy internet i założone wydarzenie na fb (co by nie musieć powtarzać po parę razy tego samego --- gdzie jesteśmy i co robimy --- i rodzinie i znajomym), więc co jakiś czas Put informuje nas \emph{,,O! Mamy 5 lajków i 3 komentarze!''}. Hm.. może należało założyć profil fb? Wtedy moglibyśmy jeszcze korzystać ze statystyk…

W ogóle (w miarę) swobodny dostęp do internet wprowadza też nową jakość wieczorami, przed spaniem. W pewnym momencie rozlega się dramatyczne wołanie \emph{,,Pucie, zrobisz modem?''}, a chwilę później \emph{,,Pucie, przełożysz komórkę bliżej naszego namiotu? Bo coś słaby zasięg…''}

\vspace{16pt}

Aha, a na zakończenie --- cytat dnia:
\epigraph{,,Czuję się, jakbym była na Islandii od tygodnia…''}{--- \textup{Karolina}}


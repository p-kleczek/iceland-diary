\chapter*{16.08. --- Reykjavík!}

Podjęta z rana próba załatania dętki zakończyła się połowicznym sukcesem --- powietrze niby ucieka, lecz na tyle powoli, że wystarczy co godzinę odrobinę podpompować i jakoś można jechać. Ciężko mi zdiagnozować, co było przyczyną powstania tej nowej nieszczelności: przetarte fragmenty opony? niemiecka łatka? mój podkład z brezentu? taśma izolacyjna?

Ponieważ do Reykjavíku mamy raptem 50~km, więc niespiesznym tempem ruszamy o 14:15. A właściwie o 15:00, spod parkingu koło takiego uskoku tektonicznego. To kolejne miejsce, gdzie kręcono epizod Gry o Tron, a dokładniej scenę, w której Petyr Baelish prowadzi Sansę Stark wąwozem ku bramie do Eyrie.

Wczorajszy manewr polegający na dotarciu do Þingvellir okazał się strzałem w dziesiątkę. Dziś wieje bowiem silny, północny-północnozachodni wiatr, który momentami ,,kładzie'' nas na bok na drodze. A znów manewr ten udało nam się zaplanować tylko i wyłącznie dzięki prognozie pogody prosto od \url{vedur.is}!

Zbliżając się do Mosfellsbær przeżywamy szok, bo oto widzimy latarnie ,,Reykjavík megalopolis'' (a nawet wiaty ichniejszego MPK!) ustawione niemal w szczerym polu, parę kilometrów przed pierwszymi poważniejszymi zabudowaniami. Nie ma co ukrywać, silnie kontrastuje to z naszymi dotychczasowymi doświadczeniami z islandzką ,,architekturą drogową''.

\img{./photos/x-s-2014-08-16_17-32-37__106.jpg}{sandpit_dinner}{Gotowanie obiadu (niemal) w piaskownicy}

W Akureyri było śniadanie pod Bonusem, natomiast tym razem pod Bonusem konsumujemy obiad. No dobrze, nie pod samym Bonusem, a na pobliskim placu zabaw --- niemniej też klimatycznie! Matki z dziećmi miały duży ubaw obserwując nas spożywających posiłek w takim otoczeniu.

Wieczorem, zadekowawszy się już na kempingu w sercu Reykjavíku, wybraliśmy się na spacer nadmorską promenadą do gmachu Harpy\footnote{Sala koncertowa i centrum konferencyjne o ciekawej, ażurowej bryle.} --- to był obowiązkowy punkt programu, zwłaszcza dla studentki architektury Karoliny. Potem zawitaliśmy na piwo do knajpy --- kolejna rzecz, która chodziła za nami od wielu dni. Było to osobliwe doświadczenie, bo normalnie na piwo ze znajomymi chodzi się po to, żeby pogadać co u kogo słychać. A my co? Mamy się pytać nawzajem \emph{,,Hej! Co ciekawego robiłeś/robiłaś przez ostatni miesiąc?''} Bez sensu!

Czuć już atmosferę powrotu, dodatkowo potęgowaną przez szok cywilizacyjny. Naprawdę, na wjeździe ,,jedynką'' do Reykjaviku wręcz nie wiedzieliśmy, jak się zachować. Nagle znaleźliśmy się na dwupasmowej drodze, co chwila mijaliśmy ronda, a tuż obok nas śmigały jedne za drugimi wszelkiej maści pojazdy --- i osobówki, i tiry\textellipsis

\img{./photos/x-s-2014-08-16_21-21-32__108.jpg}{winners}{Zdobywcy.}

\pagebreak

%\img{./photos/x-s-2014-08-17_00-03-06__420.jpg}{harpa}{Harpa --- kawał ciekawej islandzkiej architektury}
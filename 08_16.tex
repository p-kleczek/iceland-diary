\chapter*{16.08.}

Podjęta z rana próba załatania dętki zakończyła się połowicznym sukcesem --- powietrze niby ucieka, lecz na tyle powoli, że wystarczy co godzinę odrobinę podpompować i jakoś można jechać. Ciężko mi zdiagnozować, co było przyczyną powstania tej nowej nieszczelności --- przetarte fragmenty opony? niemiecka łatka? mój podkład z brezentu? taśma izolacyjna?

Ponieważ do Reykjavíku mamy raptem 50 km, więc niespiesznym tempem ruszamy o 14:15. A właściwie to o 15:00, spod parkingu koło takiego uskoku tektonicznego. To kolejne miejsce, gdzie kręcono epizod Gry o Tron, a dokładniej scenę w której Petyr Baelish prowadzi Sansę Stark wąwozem do bramy do Eyrie.

Wczorajszy manewr polegający na dobiciu do Þingvellir okazał się strzałem w dziesiątkę --- dziś wieje silny, północny-północnozachodni wiatr, który momentami ,,kładzie'' nas na bok na drodze. A to znów dzięki prognozie pogody prosto od \url{vedur.is}!

Zbliżając się do Mosfellsbær przeżywamy szok, bo oto widzimy latarnie “Reykjavik megalopolis” (a nawet wiaty ichniejszego MPK!) ustawione niemal w szczerym polu, parę kilometrów przed pierwszymi poważniejszymi zabudowaniami. Nie ma co ukrywać, silnie kontrastuje to z naszymi dotychczasowymi doświadczeniami z islandzką ,,architekturą drogową''.

\img{./photos/x-s-2014-08-16_17-32-37__106.jpg}{sandpit_dinner}{Gotowanie obiadu (niemal) w piaskownicy}

W Akureyri było śniadanie pod Bonusem, a tym razem jest obiad. No dobrze, nie pod samym Bonusem, a na pobliskim placu zabaw --- niemniej też klimatycznie! Matki z dziećmi miały duży ubaw obserwując nas spożywających posiłek w takim otoczeniu.

Wieczorem wybraliśmy się na spacer nadmorską promenadą do gmachu Harpy --- to był obowiązkowy punkt programu, zwłaszcza dla studentki architektury Karoliny. Potem zawitaliśmy na piwo do knajpy --- kolejna rzecz, która chodziła za nami od wielu dni. Było to osobliwe doświadczenie, bo normalnie na piwo ze znajomymi chodzi się po to, żeby pogadać co u kogo słychać. A my co? Mamy się pytać nawzajem \emph{,,Hej! Co ciekawego robiłeś/robiaś przez ostatni miesiąc?''} Bez sensu!

Czuć już atmosferę powrotu, dodatkowo potęgowaną przez szok cywilizacyjny. Naprawdę, na wjeździe ,,jedynką'' do Reykjaviku aż nie wiedzieliśmy, jak się zachować --- nagle znaleźliśmy się na dwupasmowej drodze, gdzie co moment są ronda, a obok nas śmigają jeden za drugim samochody i tiry.

\img{./photos/x-s-2014-08-16_21-21-32__108.jpg}{winners}{Zdobywcy.}
\img{./photos/x-s-2014-08-17_00-03-06__420.jpg}{harpa}{Harpa --- kawał ciekawej islandzkiej architektury}
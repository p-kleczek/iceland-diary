\chapter*{06.08.}

Mój poranek wyglądał następująco: 6:15 --- pobudka, szybka pasza i zwijanie namiotu w lekkim deszczu; 7:15 --- wyjazd, godzina jazdy po \road{F88} (przy świetnych warunkach drogowych) i potem gonienie resztką sił do Reykjahliðu na zakupy w Samkaupie… Cel został osiągnięty o 10:00. Jedyna trudność podróży ,,tam'' to podjazd 10\% długi na około 1 km na dosłownie ostatnich kilometrach przed miasteczkiem.

Po trzech dniach spędzonych w interiorze, gdy przekroczyłem próg sklepu, to aż poczułem się przytłoczony bogactwem dóbr spożywczych, które były na wyciągnięcie ręki. Ponieważ jednak żołądek pilnie, ale to bardzo pilnie, domagał się jeść, więc na pierwszy ogień poszedł chleb i szoko. Dopiero gdy zjadłem pół bochenka chleba z połową kubka kremu czekoladowego, nabrałem sił niezbędnych do zrobienia głównych zakupów. W międzyczasie postanowiłem jeszcze trochę podsuszyć przemoczone ubrania i tu spotkała mnie miła niespodzianka --- gdy pani ekspedientka zobaczyła moje zabiegi (ściągnąłem buty buty turystyczne i starałem się ustawić je tak, by stały jak najbliżej wentylatora od lodówki), ulitowała się nad biednym rowerzystą i zaprosiła do kotłowni. A tam mogłem nie tylko suszyć rzeczy, ale też podładować komórkę i w spokoju zjeść. Po prostu tyle wygrać! Właśnie miałem ostatecznie już opuszczać sklep, gdy natknąłem się na owych Anglików z Askji. Widząc mnie z samego rana, dość rześkiego, wykazywali już więcej respektu dla osiągnięć naszej ekipy!

W drodze powrotnej przekąpałem się w wypływającym z Krafli cieplutkim strumieniu, który przepływał przez łąkę raptem parę metrów od \road{1}. Po prostu po trzech dniach bez prysznica, kiedy ręka lepi się do wszystkiego, wszelkie obiekcje w stylu ,,co sobie ludzie pomyślą?'' przestają się liczyć --- no zwyczajnie trzeba się porządnie wyszorować! Islandczyków widok mnie kąpiącego się ,,w stroju Adama'' przesadnie nie ruszał, za to przejeżdżający Niemcy byli tym wręcz wyraźnie rozbawieni i nawet przyjaźnie machali! (Pewnie z NRD-ówka przyjechali…)

Zebraliśmy się z powrotem do kupy na niewielkim kempingu w Grimmstaðir, tuż obok… namiotu Holendrów! O tak, oni rozplanowali noclegi zdecydowanie lepiej niż my: w Brú, potem w Dreki i wreszcie we wspomnianym Grimmstaðir (już koło \road{1}). Dzięki temu ani razu noc nie zastała ich w pośrodku niczego…. Niestety, samych Holendrów nie zastaliśmy --- pewnie pojechali zwiedzać okolicę.

Stojący nieopodal niewielki sanitariat umożliwiał skorzystanie z toalety jak ludzie (i za darmo, bo to było bardziej pole biwakowe ,,przy chałupie''), a że słoneczko wyszło zza chmur to pojawiła się też kusząca możliwość zrobienia prania i od razu wysuszenia go. Proces ,,konserwacji sprzętu i siebie'' trwał dobre dwie godziny i polegał między innymi na zjadaniu różnych dobrych rzeczy, świeżo przywiezionych ze sklepu. Dzięki temu potem wszyscy w dobrych nastrojach i z wysokim morale ruszamy w stronę Dettifossu. Jeszcze w ostatniej chwili, tuż przed odjazdem, zostawiliśmy Holendrom małe słodkie pozdrowienia ,,od ekipy z Polski'' --- dwie duże żelki wraz z karteczką, wszystko zapakowane w woreczek i przyklejone do ich super plastikowego stołeczka podróżnego.

Droga do Dettifossu to większy dramat niż interior --- cały czas ,,tarka'', liczne dziury i sporo podjazdów. To jakiś cud, że nasze bagażniki jeszcze żyją! W połowie dystansu do wodospadów spotykamy naszych Holendrów --- ot, taki miły akcent.

\img{./photos/x-s-2014-08-06_21-22-05__198.jpg}{dettifoss_and_cookies}{Cookies'y --- doskonale uświetniają osiągnięcie Dettifossu}

Na parkingu przy Dettifossie jakiś Kanadyjczyk stojący koło dużego jeepa zagadnął nas: \emph{,,Ej, nie macie czasem pożyczyć pompki?''} Mówię, że owszem mamy i być może nawet napompuje nią koło do 0,1 atmosfery (bardzo starałem się, by zabrzmiało to ironicznie). O on na to: \emph{,,O! Super!''} No i faktycznie, pożyczył tą pompkę, potem siedział przez pół godziny pompując na zmianę raz prawą raz lewą ręką i… napompował! W sumie potem Put wytłumaczył mi, że oponę nabija się do koło 2 atmosfer --- znacznie mniej niż w rowerze --- więc problemem nie jest ciśnienie tylko objętość powietrza, którą należy wtłoczyć. Niemniej mimo to byłem pod wrażeniem wytrwałości tego człowieka. No i że żaden z licznych turystów-samochodziarzy na parkingu mu nie pożyczył pompki?

Prognozując dalszą trasę po raz kolejny i ja i Put popełniamy ten sam błąd, przykładając polską miarę do islandzkich warunków. Otóż droga wzdłuż polskich rzek pnie się spokojnie dnem doliny. Tu natomiast rzeka płynie sobie głębokim kanionem, a droga idzie dokładnie przez sam środek okolicznych szczytów. Tak! Niby znajdujemy się 25 km od morza, a te pagórki mają po 300 metrów wysokości!

Szczęśliwie złożyło się, że od Detiffosu do Ásbyrgi nie ma już tarki. Miejscami drogę pokrywa za to solidna warstwa drobnego szutru, co na zjazdach mocno daje się we znaki --- rower zwyczajnie ,,pływa''.

\img{./photos/x-s-2014-08-06_23-46-58__308.jpg}{gravel}{Nagroda za wytrwałość --- szkoda, że droga szutrowa.}

Początkowo chcemy zanocować na kempingu na końcu doliny Ásbyrgi --- wydaje nam się, że tam będzie naprawdę urokliwie. Błąd! Kemping ten nie istnieje, niemiecka mapa znów nas oszukała! Wróciliśmy się więc na kemping u wylotu doliny i o 23:00 zabraliśmy się za rozbijanie namiotów, a potem --- za przygotowanie i pałaszowanie ciepłej kolacji (czyli kus-kusu z warzywami z puszki, poprawionego kromkami z szoko)!

Ten kemping jest bodaj najdroższym dotychczas napotkanym --- opłata za osobę wynosi 1400 kr, lecz za prysznic należy zapłacić dodatkowe 500 kr. Nie ma ani kuchni ani \emph{common roomu}, dobrze że jest chociaż dość sprawnie działająca suszarnia (coś w rodzaju szafy, w podłodze której zamieszczono dmuchawy zimnego powietrza --- na rano pranie suche). Słowem --- to miejsce przeznaczone jest typowo dla turystów w kamperach (podobnie jak \href{http://www.vatnajokulsthjodgardur.is/english/plan-your-visit/camping/}{pozostałe kempingi} na terenie parku narodowego Vatnajökull). Z całej naszej czwórki tylko Kasia wskoczyła pod prysznic, stwierdzając: \emph{,,Po interiorze mi się należy!''}. Karolina stwierdziła, że już się myła wczoraj  w Dreki, ja też dałem sobie spokój --- w kocu rano się kąpałem --- za to Put dokonywał ekwilibrystyki nad umywalką w ubikacji dla niepełnosprawnych ;)

\hint{Toaleta dla niepełnosprawnych to wymarzone miejsce dla rowerzysty --- ma i muszlę i umywalkę oraz jest zamykana, a co za tym idzie można się spokojnie podmyć (a czasem też podsuszyć ubrania pod suszarką do rąk… --- bywało, że właśnie w toalecie dla niepełnosprawnych znajdował się jedyny działający egzemplarz).}

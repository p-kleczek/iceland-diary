\chapter*{30.07. --- Znów w cywilizacji!}

Po raz pierwszy budzi nas piękne słońce! Do tego na dobrą sprawę nie ma wiatru! Cóż, nie na długo\textellipsis

\epigraph{,,Co ci Islandia da przed zakrętem, zaraz za zakrętem odbierze.''}{--- \textup{Put}}

Powyższe powiedzenie, ukute przez Puta, jak ulał pasuje do islandzkich warunków. Gdy tylko wyjechaliśmy za najbliższy cypel, w twarz uderzył nas wiatr tak silny, że raz Karolina niemal wylądowała w rowie. Nasza prędkość --- na prostym, równym odcinku --- spadła do ledwie 8~km/h, a miejscami ledwo byliśmy w stanie wyjechać na choćby niewielkie wzniesienia! Za to jakieś 15~km przed Höfn wreszcie skręciliśmy tak, że zaczęło wiać w plecy i dzięki temu kawałek przed wjazdem do miasta mogliśmy spokojnie rozpędzić się na prostych do oszałamiającej prędkości 50~km/h. Stojący na poboczu policjanci z drogówki mieli spory ubaw obserwując nasze wyczyny --- szczególnie, że jechaliśmy szybciej niż niektóre samochody (których kierowcy przestrzegali ograniczeń prędkości w terenie zabudowanym)!

\img{./photos/x-s-2014-07-30_08-48-02__59.jpg}{moss_camping}{Słońce po raz drugi!}

\hint{Przy silnym wietrze należy szczególnie uważać na mostach (zarówno przy wjeździe jak i zjeździe), przy przejeżdżaniu przez niewielkie otwarte przestrzenie, np. wiadukt, oraz gdy mijają cię większe pojazdy (choćby duży jeep). Dlaczego? Początkowo walczysz z wiatrem. Potem zaczyna mijać cię tir --- zasłania ci wiatr, a ty zjeżdżasz w stronę środka jezdni. Potem kończy cię mijać i~wtedy ponownie uderza w ciebie wiatr --- lądujesz w rowie. Na moście bywa śmieszniej, bo odbijający się od wysokich barierek i~murków wiatr lubi tworzyć swoisty tunel aerodynamiczny\textellipsis}

W Nettó w Höfn dokonujemy po raz pierwszy ,,królewskich zakupów'' --- nie tylko dlatego, że potrzebujemy zapasów jedzenia na 3 dni, ale też ponieważ wreszcie sobie nie krzywdujemy. W koszyku ląduje fura ciastek, owoce w puszce w dużych ilościach, sok pomarańczowy\textellipsis Do pełni szczęścia brakuje tylko zacisznego miejsca na zrobienie obiadu. Po chwili znajdujemy takowe --- tuż obok tylnego wejścia do supermarketu stoi solidny drewniany ogrodowy ławo-stół. Na drugim daniu się nie skończyło, w ramach deseru konsumujemy Marijki z Szoko-szoko!

\img{./photos/x-s-2014-07-30_16-10-44__60.jpg}{xtra_dinner}{Produkty marki X-tra podstawą zdrowej diety :)}

Kawałek za Höfn niespodzianka: tunel drogowy, jeden z trzech czy czterech na całej Islandii. Rozpoczyna się nerwowe szukanie kamizelek odblaskowych, które co po niektórzy wrzucili na samo dno sakwy. Generalnie tunel jest krótki i dość dobrze oświetlony, lecz wiadomo --- przezorny zawsze ubezpieczony.

\epigraph{,,Co się może zdarzyć na ostatnich kilometrach przed kempingiem?''}{--- \textup{Kasia}}

Hm\textellipsis ostatnie kilometry\textellipsis Już niemal widać miejsce noclegu, już każdy myśli o jedzeniu i myciu --- cóż takiego może się jeszcze wydarzyć? Otóż na ostatnich kilometrach może znowu powiać prosto w twarz z prędkością dochodzącą do 15~m/s --- zjeżdżamy z dość stromej górki ledwo dokręcając (na przełożeniach 1--3 bądź 1--4) do 8~km/h.

Zaraz za mostem prowadzącym do Stafafell widzimy znak ,,kemping 7~km'' i strzałkę na drogę szutrową. Ponieważ nie ogarnęliśmy, że jeszcze jeden kemping znajduje się raptem za 2~km (tuż przy \road{1}), więc chcąc nie chcąc ruszamy w podróż po żwirze. Po jednym kilometrze jazdy mamy już serdecznie dość wertepów i po prostu rozbijamy się na minipolance przy drodze.

Wieczorem wreszcie mieliśmy chwilę czasu i ochotę, by pograć w karty. Nic ambitnego, prosta gra karciana --- makao. Ale ile emocji wzbudza samo ustalanie wspólnej wersji zasad!

\img{./photos/x-s-2014-07-30_23-04-25__117.jpg}{camping_stafafell}{\textellipsis i po raz kolejny kemping ,,w tak pięknych okolicznościach przyrody''!}

\chapter*{30.07.}

Po raz pierwszy budzi nas piękne słońce (a do tego właściwie brak wiatru)! Nie na długo…

“Co ci Islandia da przed zakrętem, zaraz za zakrętem odbierze.” --- oto nowe powiedzenie ukute przez Puta, które jak ulał pasuje do islandzkch warunków. Gdy tylko wyjechaliśmy za najbliższy cypel, w twarz uderzył nas wiatr tak silny, że raz Karolina autentycznie niemal wylądowała w rowie. Nasza prędkość --- na prostym, równym odcinku --- spadła do 8 km/h, a miejscami ledwo byliśmy w stanie wyjechać na choćby niewielkie wzniesienia! Za to jakieś 15 km przed Höfn wreszcie skręciliśmy tak, że zaczęło wiać w plecy i dzięki temu kawałek przed wjazdem do miasta mogliśmy spokojnie rozpędzić się na prostych do 50 km/h. (Policjanci z drogówki mieli spory ubaw obserwując nasze wyczyny :)

\img{./photos/x-s-2014-07-30_08-48-02__59.jpg}{moss_camping}{Słońce po raz drugi!}

\hint{Przy silnym wietrze należy szczególnie uważać na mostach (zarówno przy wjeździe jak i zjeździe), przy przejeżdżaniu przez niewielkie otwarte przestrzenie w stylu wiadukt oraz gdy mijają cię większe pojazdy (choćby duży jeep). Dlaczego? Początkowo walczysz z wiatrem. Potem zaczyna mijać cię tir --- zasłania ci wiatr, a ty zjeżdżasz w stronę środka jezdni. Potem kończy cię mijać i wtedy ponownie uderza w ciebie wiatr --- lądujesz w rowie. Na moście bywa śmieszniej, bo odbijający się od wysokich barierek i murków wiatr lubi tworzyć swoisty tunel aerodynamiczny…}

W Nettó w Höfn dokonujemy po raz pierwszy “królewskich zakupów” --- potrzebujemy bowiem jedzenia na 3 dni, a przy okazji nie krzywdujemy sobie --- w koszyku ląduje fura ciastek, owoce w puszce w dużych ilościach, sok pomarańczowy… Do pełni szczęścia brakuje tylko zacisznego miejsca na zrobienie obiadu. Po chwili znajdujemy takowe --- tuż obok tylnego wejścia do supermarketu stoi solidny drewniany ogrodowy ławo-stół. Na drugim daniu się nie skończyło, w ramach deseru konsumujemy Marijki z Szoko-szoko!

\img{./photos/x-s-2014-07-30_16-10-44__60.jpg}{xtra_dinner}{Produkty marki X-tra podstawą zdrowej diety :)}

Kawałek za Höfn niespodzianka --- tunel, jeden z 3 czy 4 na całej Islandii. Rozpoczyna się nerwowe szukanie kamizelek, które co po niektórzy wrzucili na samo dno sakwy. Generalnie tunel jest krótki i dość dobrze oświetlony, lecz wiadomo --- przezorny zawsze ubezpieczony.

Kasia: “Co się może zdarzyć na ostatnich kilometrach przed kempingiem?” Otóż na ostatnich kilometrach może znowu powiać w mordę z prędkością dochodzącą do 15 m/s --- zjeżdżamy z dość stromej górki ledwo dokręcając (na przełożeniach 1-3 bądź 1-4) do 8 km/h.

Zaraz za mostem prowadzącym do Stafafell widzimy znak “kemping 7 km” i strzałkę na drogę szutrową. Ponieważ nie ogarnęliśmy, że jeszcze jeden kemping znajduje się za 2 km (tuż przy \road{1}), więc chcąc niechcąc ruszamy w podróż po żwirze. Po 1 km jazdy mamy już serdecznie dość wertepów i po prostu rozbijamy się na minipolance przy drodze.

Wieczorem wreszcie znaleźliśmy chwilę czasu i ochoty, by pograć w karty. Nic ambitnego --- makao. Ale ile emocji wzbudza samo ustalanie wspólnej wersji zasad!

\img{./photos/x-s-2014-07-30_23-04-25__117.jpg}{camping_stafafell}{…i po raz kolejny kemping w pięknych okolicznościach przyrody!}

\chapter*{14.08.}

Wczoraj pewien Niemiec na motorze, spotkany na skrzyżowaniu przed Asgarður, solidnie nastraszył nas: \emph{,,Pierwsze 10-15 km od skrzyżowania to będzie jakiś dramat --- żwir i kamienie przez cały czas. Ja na motorze jakoś dałem radę przejechać, ale nie wiem, czy wam też się uda.''} Każde z nas wyobraża już sobie mozolne brnięcie przez morze kamyków i to ile godzin będziemy w związku z tym ,,w plecy''. Jedziemy jednak, jedziemy, kilometry mijają, a tu nic! Jedzie się całkiem znośnie, żadnego zsiadania z roweru, może trafiły się ze dwa krótkie stromsze podjazdy, na których trzeba by trochę się skoncentrować na szukaniu dogodnego przejazdu. Kawałek przed mostem przy Hvítárvatn zaczyna się dobra, twarda nawierzchnia, a typowy asfalt --- na 13-15 km przed Gullfossem.

\img{./photos/x-s-2014-08-14_15-06-56__260.jpg}{rock_break}{Postój przy skałce-wiatrochronie --- jedynej w promieniu wielu kilometrów}
\img{./photos/x-s-2014-08-14_15-08-38__407.jpg}{lone_rider}{Kasia jako Samotny Jeździec}

Trapią nas dziś jednak liczne usterki: Karolinie pękł trzeci już wspornik bagażnika oraz zrobiła się dziura w sakwie (z przetarcia), podobną dziurę w sakwie zyskał i Put. Mnie natomiast przetarta się tylna opona --- przetarła całkiem konkretnie, można palec włożyć w dziurę. Oczywiście od razu wiązało się to ze złapaniem kapcia. Od mijających nas Niemców wyżebrałem łatki na oponę, lecz wcześniej już “załatałem” dziurę kawałkiem brezentu (a potem jeszcze zmieniłem oponę tylną z przednią, bo przednia się aż tak bardzo nie starła) --- wierzę, że jadąc wystarczająco ostrożnie jakoś doturlikam się do Keflaviku…

\hint{Jeśli nie chcesz wozić ze sobą zapasowej opony, zainwestuj w łatki na oponę --- na przykład \href{http://www.competitivecyclist.com/park-tool-emergency-tire-boot-set-tb-2c}{TB-2C} firmy Park Tool. Taka łatka przypomina dużą, grubą, powlekaną naklejkę i doskonale sprawdza się na tymczasowe (!) naprawy.}

Gdy tylko przejechaliśmy po pierwszych metrach nawierzchnii bitumiecznej, z radości aż urządziliśmy “powitanie asfaltu”. Po czterech dniach jazdy po bezdrożach nagły brak ciągłych podskoków i ,,uciekania'' tylnego koła był czymś wprost niesamowitym!

Powrót do cywilizacji to nie tylko asfalt, ale i bliskość sklepów. W związku z tym mogliśmy sobie pozwolić na autentyczne świętowanie i tak nasza kolacja okazała się być trzydaniowa: jako główne --- parówki z ryżem, potem --- kus-kus z warzywami, na deser --- herbatniki imbirowe z herbatą oraz daktyle. W ten sposób pozbyliśmy się właściwie dokumentnie całej żywności.

\img{./photos/x-s-2014-08-14_16-12-19__98.jpg}{tyre_repair}{Na ostatnich kilomterach zawsze się coś spieprzy…}
\img{./photos/x-s-2014-08-14_18-12-22__100.jpg}{welcome_asphalt}{Powitanie asfaltu.}
\img{./photos/x-s-2014-08-14_19-28-50__101.jpg}{icelandic_bus}{Autobus turystyczny --- wersja islandzka}
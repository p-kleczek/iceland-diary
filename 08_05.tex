\chapter*{05.08. --- Byle uciec stąd\textellipsis}

Pierwsze, co dziś o poranku słyszymy z namiotu dziewczyn, to soczysta wiązanka. Cytować nie będę, ale przytoczę sens, parafrazując za prof. Miodkiem: \emph{,,Bardzo mnie irytują te powstałe na podłodze namiotu kałuże, które sprawiły, iż moja mata samopompująca oraz śpiwór nadmiernie zawilgły. Nie nastraja mnie to pozytywnie do dalszej podróży''}. Co spowodowało ten kataklizm? Właśnie to, że nieumiejętnie rozłożyliśmy folię i cała woda z tropiku spływała pod namiot (jak czytelnik może się domyślać, porada z dnia 3.~sierpnia powstała już \textit{post factum}).

To --- w połączeniu z faktem odkrycia przez Karolinę, że wczoraj rozwalił jej się bagażnik (od jazdy po wertepach nastąpiło zmęczenie materiału i dwa wsporniki ,,poszły w~pizdu'') --- wprowadziło nas w nastrój ogólnej rezygnacji, który trwał dobre dwie godziny.

Wreszcie jakoś wzięliśmy się w garść i przystąpiliśmy do naprawy zniszczeń. Bagażnik udało się złożyć do kupy z pomocą --- a jakże --- taśmy izolacyjnej i opasek zaciskowych\footnote{Po dłuższej dyskusji udało nam się ustalić, że opaska zaciskowa to po śląsku ,,trytytka'', natomiast mieszkańcy małopolski mówią na nią ,,zip''.}. Niestety, z mokrymi matami i śpiworami nie możemy chwilowo zrobić czegokolwiek konstruktywnego, bo na polu mży, a w budynkach nie ma ogrzewania.

Holendrzy po raz kolejny nas wyprzedzili --- oni wyruszyli o 9:30, a my o 12:00. Cicha rywalizacja wciąż trwa\textellipsis

Po wczorajszym dniu wiedzieliśmy już, że pierwsze 13~km drogi (do skrzyżowania) to będzie poezja. Szkoda, że ta autostrada kończyła się tuż za nim. Dalej, aż do Herðubreiðarlindir, wytrzęsła nam tyłki swoista ,,tarka'' (wiecie, takie żwirowe garby w regularnych odstępach paru centymetrów). Właśnie ta fatalna nawierzchnia spowodowała straty w dwóch kolejnych bagażnikach: ja zgubiłem nakrętkę od wspornika, a Put śrubkę (również od wspornika). Dobrze, że mieliśmy po jednej nadprogramowej, bo inaczej bylibyśmy w przysłowiowej ,,czarnej dupie''\textellipsis

\hint{Zadbaj, by w rowerze mieć standardowe śruby 5 mm z łebkiem na imbus oraz nakrętki samozaciskowe (nie odkręcą się same). Weź też ze sobą na zapas po parę sztuk zarówno śrub jak i nakrętek.}

\pagebreak

\img{./photos/x-s-2014-08-05_20-57-16__190.jpg}{rasper}{Przed nami tarka\textellipsis}
\img{./photos/x-s-2014-08-05_20-57-32__192.jpg}{askja_void}{\textellipsis a wokół nas --- jak zwykle --- nic.}

\pagebreak

Nie tylko droga, ale też i samo Herðubreiðarlindir zostało zalane --- w szczególności na drodze, na wjeździe i wyjeździe, pojawiły się dwa dodatkowe ,,brody'' do przekroczenia. Te niespodziewane przeszkody terenowe sprawiły, że nie w głowie było nam brać dalszą drogę ,,z marszu''. Zapragnęliśmy najpierw zjeść godziwy obiad. Udałem się zatem do pani dozorczyni spytać się o dalszą drogę i o to, \emph{,,w którym miejscu na kempingu mieści się kuchnia''}. W odpowiedzi usłyszałem między innymi, że \emph{,,skorzystanie z kuchni kosztuje 500~kr.}'' No ale jak to, pytam, za sam fakt posiedzenia w ciepłej kuchni?! Mówiłem wyraźnie, że mamy własne palniki i własne naczynia! W odpowiedzi słyszą: \emph{,,Właśnie tak!} Skwitowałem to, że spytam się reszty, co oni na to. A reszta mówi: \emph{Trudno. Zapłacimy. Byleby obiad był\textellipsis''}. Wracam więc do dozorczyni, wręczam banknot 1000~kr i czekam na resztę. A pani mówi: ,,Jeszcze 1000~kr''. Oniemiałem. No bo jak to, będą nas kasować za każdą osobę? Wyjąłem zatem z ręki dozorczyni wręczony uprzednio banknot i wygłosiłem oświadczenie o tym, jak to polskich studentów nie stać na płacenie 2000~kr za luksus zrobienia sobie obiadu. Wtedy dozorczyni stwierdziła, że w porządku --- 1000~kr za całą grupę starczy. Przystałem na jej propozycję, lecz niesmak pozostał\textellipsis

Po obiedzie, całkiem przypadkiem, załapaliśmy się na autobus do Mývatn. Niby wiedzieliśmy, że takowy kursuje, ale jakoś pomyliły nam się godziny odjazdu z Herðubreiðarlindir --- sądziliśmy, że przyjedzie dopiero za godzinę. Do środka zmieściły się tylko dziewczyny (z rowerami i swoimi sakwami) oraz część bagaży moich i część bagaży Puta. W~środku siedzieli znajomi z Askji: Anglicy ,,od nerek'' (ich wyprawa miała charakter charytatywny, zbierali bodaj na przeszczep nerki dla kogoś) i Pan Rusek. \emph{,,Leszcze!''} --- pomyśleliśmy. Plan był taki, że autobus przewiezie dziewczyny przez najtrudniejszy odcinek (czyli zalaną drogę i dwa brody), a zaraz potem wysadzi. Ta operacja nie powinna kosztować ani korony! Tak też się stało, wszystko poszło zgodnie z planem: bezproblemowo i za frajer. Osobliwie zachował się starszy Anglik, który wysiadł potem na chwilę razem z dziewczynami --- nachylił się nad Karoliną i szepnął jej do ucha: \emph{,,Remember, it’s warmer inside the bus!''}

\img{./photos/x-s-2014-08-05_20-01-34__79.jpg}{last_ford}{Ostatni bród w tej edycji interioru.}

\pagebreak

W międzyczasie ja i Put ruszyliśmy na autonogach w dalszą drogę. Wbrew apokaliptycznym wizjom roztaczanym przez ludzi z Dreki zalana droga była łatwa do obejścia --- wystarczyło odbić 20~metrów i przejść bokiem po lekko namokniętym gruncie. Podobnie było z brodami --- ich sforsowanie nie wymagało większego wysiłku niż pokonywanie tych na \road{F910}. Dziewczyny czekały na nas w sumie może ze 20~minut, lecz było to dla nich wystarczająco długo, by rozbić namiot \tongue

Ciśniemy dalej, w lekkim deszczu i chłodzie, do oporu. Po drodze pokonujemy jeszcze jeden solidny bród. W sumie jednak końcówkę jechało się całkiem znośnie, bo ani razu nie było piasku. Droga w większości dość ubita --- może to przez padający deszcz, który ,,związał'' ten piasek?

\img{./photos/x-s-2014-08-06_10-44-32__81.jpg}{}{Jak się nie ma, co się lubi\textellipsis}

Nocleg wypadł znów --- nie inaczej --- pośród pustyni (do złączenia się z \road{1} pozostało jeszcze 19~km). Największym problem okazało się wybranie miejsca na rozbicie namiotów. Chodziło o to, żeby były one osłonięte od wiatru oraz żeby na ziemi leżało jak najmniej kamyków i kamyczków. Aż chce się powiedzieć: \emph{,,Wszędzie tyle samo! Czy to takie ważne?!''} Hm\textellipsis to po prostu wychodzi z nas zmęczenie --- 77~km po wertepach, w ciężkich warunkach robi swoje.

Powiem za siebie: Wieczorem byłem tak wytyrany, że w 5~minut po rozłożeniu namiotu i przebraniu się --- zasnąłem. Nie udało mi się wytrwać do rozpoczęcia fazy przygotowania obiado-kolacji, miałem za to plan by zaraz z rana pocisnąć do Mývatn i dokonać solidnych zakupów w tamtejszym Samkaupie. Problem polegał na tym, że na obiado-kolację miał być kuskus, który to ja posiadałem w swoich zapasach. Put wołał mnie później z namiotu dziewczyn (podobno głośno) --- nie dawałem zna\-ku życia, szukał torebek z kaszą --- lecz nie znalazł. Reszta nie doczekała się więc kuskusu i ostatecznie towarzystwo dojadało dwoma chlebami z mielonką.

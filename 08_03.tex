\chapter*{03.08. --- Interior}

Droga do Brú jest jeszcze całkiem znośna --- owszem, sporo jeżdżenia góra-dół, ale nawierznia niczego sobie (nawet asfalt mają po wioskach!)

Gdy zajechaliśmy do ,,centrum'' Brú i stanęliśmy pod drogowskazami, aż przetarliśmy oczy ze zdumienia --- jak byk stoi ,,Askja 88''. Czyżby mapa nas okłamała? Łudzimy się jeszcze, że może to odległość na szczyt Askji --- a ponieważ my jedziemy na kemping Dreki, u jej podnóża, to akurat wyjdzie 78 km.

Zaraz na pierwszym szutrowym podjeździe dopadają nas deszcze konwekcyjne --- tym razem wiemy już, co robić. Znów plandeka idzie w ruch, znów wykorzystujemy tą chwilę na przygotowanie i konsumpcję kanapek.

Po paru kilometrach napotkaliśmy pierwszy bród. Przy jego przekraczaniu mieliśmy jeszcze sporo radości --- zwłaszcza że był na tyle płytki, że od biedy można by przejechać przez niego rowerem.

Zaraz potem wjechaliśmy w strefę permanentnych opadów. Podobno właśnie tu, na ciągnącym się na prawo i na lewo łańcuchu górskim spada większość deszczu z ciągnących znad wschodniego wybrzeża chmur i dalej (nad interior) napływa już w miarę suche powietrze. Niby kropi, ale równocześnie promienie słońca bajecznie rozświetlają niewielkie kamyczki zalegające przydrożne tereny --- tworząc wrażenie jechania przez srebrzyste morze! Szkoda, że strome podjazdy utrudniają spokojną kontemplację tych widoków ;-)

\img{./photos/x-s-2014-08-03_16-23-04__68.jpg}{rainy_moon_rocks}{Słońce, deszczyk, srebrzyste kamyczki…}

Jeden z brodów, na które natrafiliśmy na \road{F910} po zjeździe z przełęczy, nie był zaznaczony na naszej --- dość dokładnej --- niemieckiej mapie. Na mapie rowerowej ,,Cycling Iceland'' wręcz żaden z trzech brodów nie figurował! Akurat trafiliśmy na taki stan wody, że wystarczyło założyć klapki i jako-tako dało się przeprowadzić rower, ale przy odrobinie wyższym stanie wody istniało ryzyko zalania piasty. Natomiast to, co zastaliśmy na odcinku po skrzyżowaniu z \road{F905}, to już były brody pełną gębą --- woda po kolana, dość silny nurt, w obu przypadkach nie obyło się bez demontażu bagażu.

\img{./photos/x-s-2014-08-03_18-53-50__165.jpg}{ford_riding}{Bród czasem można pokonać "w bród"…}
\img{./photos/x-s-2014-08-03_21-01-14__259.jpg}{ford_walking}{lecz bywa też, że trzeba się "spieszyć" ;)}

\hint{Bród najlepiej przekraczać jego dolnym skrajem (czyli po tej stronie, w dół której płynie nurt), gdyż tam jest najpłycej. Przy większości brodów stoją tabliczki ilustrujące to w odpowiedni sposób.}

W pewnym momencie widzimy, że Kasia zostałą wyraźnie z tyłu, że rozmawia z jakimiś losowymi ludźmi jadącymi samochodem i że coś sobie podają z ręki do ręki. O co chodzi?! Historia wyglądała tak: Kasia jedzie sobie spokojnie, w pewnym momencie czuje że jakoś tak lżej się jej jedzie, ale się tym nie przejmuje --- a wręcz cieszy! Jadący za nią samochód trąbi, lecz Kasia tego nie słyszy. Po chwili orientuje się, skąd to nagłe przyspieszenie --- zgubiła rzeczy z paki! Samochód za nią nadal trąbi, Kasia myśli sobie --- \emph{,,O wy ch*** głupie!''} Gdy wreszcie samochód dogonił Kasię, okazało się że jej zgubione rzeczy są w środku! Ci podróżni specjalnie specjalnie zawrócili, żeby dowieźć zguby, a jakby tego było mało --- zaproponowali, że śmieci wezmą ze sobą :)

Niemal co parę kilometrów powtarza się mniej więcej taka wymiana zdań --- ktoś z ekipy pyta \emph{Dojedziemy w jeden dzień do Askji?}, na co Paweł pewnym głosem odpowiada \tak{Tak!}.

\img{./photos/x-s-2014-08-03_17-01-48__71.jpg}{road_to_askja}{Droga do Askji}

Gdy słońce chyliło się już ku zachodowi, a my wciąż przemierzaliśmy pustynię i nijak nie widać było końca naszej dzisiejszej wędrówki, zapragnęliśmy wreszcie zjeść coś konkretnego. Tylko gdzie tu gotować, kiedy wokół nas tylko piasek? Tu ponownie swą użyteczność wykazała płachta, którą obłożyliśmy oparte o duży kamień rowery tak, że całość tworzyła coś w rodzaju plażowego parawanu z okapem. Były to warunki partyzanckie i Put-kucharz wiele się natrudził, by wciąż na nowo podpalać gasnący palnik, lecz obiad przyszedł w samą porę (bo jak wiadomo --- głodny Polak to zły Polak;) Po takim posiłku znów mieliśmy siły (i odrobinę więcej ochoty), by jechać dalej.

Olbrzymie wrażenie zrobiła na nas księżycowa sceneria po przekroczeniu mostu na rzece Kreppa --- spękane płyty zastygłej magmy, dużo pumeksu, żwirek i wijąca się wśród wulkanicznych skałek droga. Nic dziwnego, że teren ten został objęty ochroną. A do tego jeszcze widok słońca zachodzącego za Herðubreið, którego szczyt nakryty był jakby kapeluszem chmur --- no po prostu magiczne!

Właśnie --- na terenie rezerwatu nie wolno się rozbijać. Czy jednak mieliśmy wybór, gdy o 23:00 wciąż znajdowaliśmy się w przysłowiowym (!) lesie? Nie zważając na zakazy, z duszą na ramieniu, założyliśmy obozowisko za dużymi skałami --- tak, by patrolujący okolicę rangersi (tj. strażnicy parku) nie dostrzegli nas zbyt łatwo. Właśnie dlatego przed pójściem spać zatarłem jeszcze nasze ślady klapkiem Puta…

\img{./photos/x-s-2014-08-04_00-51-09__268.jpg}{herthubreith}{Herðubreið o zmroku}
\img{./photos/x-s-2014-08-04_08-37-36__75.jpg}{camping_national_park}{Breaking the law! (niestety)}

\hint{Płachta budowlana może służyć jako warstwa izolacyjno-ochronna od podłoża (m.in. żeby drobne kamyczki nie podziurawiły podłogi namiotu). Należy jednak zwrócić uwagę, by była ona w całości schowana pod namiotem! (Niezastosowanie się do tego grozi --- w przypadku deszczu --- powodzią w namiocie!)}

\chapter*{24.07.}

Po krótkim, lecz intensywnym porannym deszczu przed Þorlakshöfn kolejne godziny były niemal bezwietrznie (bądź wręcz wiało w plecy!) i świeciło piękne słońce (!). Tak… byliśmy tym faktem tak zaskoczeni, że nie przygotowaliśmy zawczasu kremu z filtrem UV - był na dnie sakwy - i w efekcie co poniektórzy mieli z tego dnia pamiątki w postaci bąbli na uszach i łuszczącej się skóry na nosie…

\img{./photos/x-s-2014-07-24_14-38-32__16.jpg}{fodder_thorlakshofn}{Pasza w Þorlakshöfn}

Pierwsze co zrobiliśmy po przybyciu do Hveragerði - kolejnego dużego miasteczka na naszej dzisiejszej trasie - to zajechaliśmy w okolicę Bonusa. Szybka sonda wykazała jednak, że nikt jeszcze nie jest głodny! Dominował pogląd: “A, zjemy obiad jak będziemy wracać po kąpieli w gorących źródłach!”

Zajechaliśmy więc do Rjúpnabrekkur, gdzie zostawiliśmy rowery, i dalej - już pieszo - udaliśmy się górską ścieżką w stronę \href{http://www.vulkaner.no/t/isl2004/hot.html}{Klambragil}. Celem było znalezienie “niebieskiego jeziorka”, w którym można by się pokąpać - informacje o jego istnieniu Karolina otrzymała od swojego znajomego. Cóż… poszukiwania zakończyły się fiaskiem i koniec końców moczyliśmy się w “zwykłej” rzeczce płynącej środkiem doliny (ok, też była bardzo ciepła, a momentami - gorąca).

\img{./photos/x-s-2014-07-24_17-17-04__18.jpg}{klambragil_on_the_way}{W drodze do wód…}
\img{./photos/x-s-2014-07-24_19-24-40__23.jpg}{klambragil_stream}{Moczymy tyłki}

Gdy wróciliśmy pod Bonus, spotkała nas niemiła niespodzianka - sklep był już zamknięty. Zamknęli go równo 15 minut przed naszym przyjazdem, o 18:30. Tak więc póki co nici z obiadu, bo średnio nam się uśmiechało kupować - po czarnorynkowych cenach - jedzenie na znajdującej się vis-a-vis Bonusa[16.08.]
Podjęta z rana próba załatania dętki zakończyła się połowicznym sukcesem - niby powietrze ucieka, lecz na tyle powoli, że wystarczy co godzinę odrobinę podpompować i jakoś można jechać. Ciężko mi zdiagnozować, co było przyczyną powstania tej nowej nieszczelności - przetarte fragmenty opony? niemiecka łatka? mój podkład z brezentu? taśma izolacyjna?
Ponieważ do Reykjavíku mamy raptem 50 km, więc niespiesznym tempem ruszamy o 14:15. A właściwie to o 15:00, spod parkingu koło takiego uskoku tektonicznego. To kolejne miejsce, gdzie kręcono epizod Gry o Tron (a dokładniej scenę gdzie Petyr Baelish prowadzi Sansę Stark wąwozem do bramy do Eyrie.
Wczorajszy manewr polegający na dobiciu do Þingvellir okazał się strzałem w dziesiątkę - dziś wieje silny, północny-północnozachodni wiatr, który momentami "kładzie" nas na bok na drodze. A to znów dzięki prognozie pogody prosto od vedur.is!
Zbliżając się do Mosfellsbær przeżywamy szok, bo oto widzimy latarnie “Reykjavik megalopolis” (oraz wiaty MPK!) ustawione niemal w szczerym polu, parę kilometrów przed pierwszymi poważniejszymi zabudowaniami. Nie ma co ukrywać, silnie kontrastuje to z naszymi dotychczasowymi doświadczeniami z islandzką "architekturą drogową".
W Akureyri było śniadanie pod Bonusem, a tym razem jest obiad. OK, nie pod samym Bonusem, a na pobliskim placu zabaw - niemniej też klimatycznie! Matki z dziećmi miały duży ubaw obserwując nas spożywających posiłek w takim otoczeniu.
Wieczorem wybraliśmy się na spacer nadmorską promenadą do gmachu Harpy - to był obowiązkowy punkt programu, zwłaszcza dla studentki architektury Karoliny. Potem poszliśmy na piwo do knajpy - kolejna rzecz, która chodziła za nami od wielu dni. Było to osobliwe doświadczenie, bo normalnie na piwo ze znajomymi chodzi się po to, żeby pogadać co u kogo słychać. A my co? Mamy się pytać nawzajem "Hej! Co ciekawego robiłeś/robiaś przez ostatni miesiąc?" Bez sensu!
Czuć już atmosferę powrotu, dodatkowo potęgowaną przez szok cywilizacyjny. Naprawdę, na wjeździe "jedynką" do Reykjaviku aż nie wiedzieliśmy, jak się zachować - nagle znaleźliśmy się na dwupasmowej drodze, gdzie co moment są ronda, a obok nas śmigają jeden za drugim samochody i tiry. stacji benzynowej. Chcąc nie chcąc, bez zbędnego guzdrania się, pojechaliśmy do Selfossu - miasta oddalonego o ledwie 12 km. Tam postanowiliśmy zanocować i poszukać jakiegoś innego marketu.

Kemping w Selfossie ma wszystko to, co każdy szanujący się kemping mieć powinien - przestronną, ciepłą świetlicą z aneksem kuchennym, specjalną trawiastą polanę (niemal tuż przy budynku gospodarczym) wyłącznie dla turystów z namiotami oraz ciepły prysznic w cenie. Aha, no i niedaleko od kempingu jest Samkaup czynny w dni powszednie od 7:30 do 23:30! Udało nam się więc zrobić zakupy na śniadanie - jutro nie będziemy musieli zaczynać dnia od spacerów wśród półek sklepowych.

Jednym z milszych akcentów dnia dzisiejszego był widok całych stad koni islandzkich galopujących po ciągnących się wzdłóż drogi pastwiskach.
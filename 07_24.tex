\chapter*{24.07.}

Po krótkim, lecz intensywnym porannym deszczu przed Þorlakshöfn kolejne godziny były niemal bezwietrznie (bądź wręcz wiało w plecy!) i świeciło piękne słońce (!). Tak… byliśmy tym faktem tak zaskoczeni, że nie przygotowaliśmy zawczasu kremu z filtrem UV --- był na dnie sakwy --- i w efekcie co poniektórzy mieli z tego dnia pamiątki w postaci bąbli na uszach i łuszczącej się skóry na nosie…

\img{./photos/x-s-2014-07-24_14-38-32__16.jpg}{fodder_thorlakshofn}{Pasza w Þorlakshöfn}

Pierwsze co zrobiliśmy po przybyciu do Hveragerði --- kolejnego dużego miasteczka na naszej dzisiejszej trasie --- to zajechaliśmy w okolicę Bonusa. Szybka sonda wykazała jednak, że nikt jeszcze nie jest głodny! Dominował pogląd \emph{,,A, zjemy obiad jak będziemy wracać po kąpieli w gorących źródłach!''}

Zajechaliśmy więc do Rjúpnabrekkur, gdzie zostawiliśmy rowery, i dalej --- już pieszo --- udaliśmy się górską ścieżką w stronę \href{http://www.vulkaner.no/t/isl2004/hot.html}{Klambragil}. Celem było znalezienie ,,niebieskiego jeziorka'', w którym można by się pokąpać --- informacje o jego istnieniu Karolina otrzymała od swojego znajomego. Cóż… poszukiwania zakończyły się fiaskiem i koniec końców moczyliśmy się w ,,zwykłej'' rzeczce płynącej środkiem doliny (ok, też była bardzo ciepła, a momentami --- gorąca).

\img{./photos/x-s-2014-07-24_17-17-04__18.jpg}{klambragil_on_the_way}{W drodze do wód…}
\img{./photos/x-s-2014-07-24_19-24-40__23.jpg}{klambragil_stream}{Moczymy tyłki w Klambragil}

Gdy wróciliśmy pod Bonus, spotkała nas niemiła niespodzianka --- sklep był już zamknięty. Zamknęli go równo 15 minut przed naszym przyjazdem, o 18:30. Tak więc póki co nici z obiadu, bo średnio nam się uśmiechało kupować --- po czarnorynkowych cenach --- jedzenie na znajdującej się vis-a-vis Bonusa stacji benzynowej. Chcąc nie chcąc, bez zbędnego guzdrania się, pojechaliśmy do Selfossu --- miasta oddalonego o ledwie 12 km. Tam postanowiliśmy zanocować i poszukać jakiegoś innego marketu.

Kemping w Selfossie ma wszystko to, co każdy szanujący się kemping mieć powinien: przestronną, ciepłą świetlicą z aneksem kuchennym, specjalną trawiastą polanę (niemal tuż przy budynku gospodarczym) wyłącznie dla turystów z namiotami oraz ciepły prysznic w cenie. Aha, no i niedaleko od kempingu jest Samkaup czynny w dni powszednie od 7:30 do 23:30! Udało nam się więc zrobić zakupy na śniadanie --- jutro nie będziemy musieli zaczynać dnia od spacerów wśród półek sklepowych.

Jednym z milszych akcentów dnia dzisiejszego był widok całych stad koni islandzkich galopujących po ciągnących się wzdłóż drogi pastwiskach.
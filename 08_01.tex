\chapter*{01.08.}

Ta noc była zimna jak nigdy --- nie dość, że zawiewało z gór, to jeszcze ciągnęła wilgoć od zatoki, a momentami temperatura spadała poniżej zera. Skutkiem tego o poranku, wbrew zapowiedziom z dnia poprzedniego, większość wycieczki nie umyła się w (czystej!) górskiej rzece;)

\hint{Choćby pogoda była paskudna, minimum ,,higieny rowerzysty'' trzeba zachować --- umyć pachwiny, pachy, stopy… Owszem, nie zawsze to przyjemne, lecz wierzcie mi --- zapocone odparzenia i czyraki są znacznie gorsze! Aha, zęby można myć w namiocie i pluć przez wejście ;-)}

Coś zawiodło, bo zaplanowana na 8:00 pobudka nie doszła do skutku z powodu ,,oporu materii'' --- wyruszyliśmy dopiero koło 10:30. Na szczęście nic to, bo do przejechania mamy ledwie 65 km.

Wyjazd na Öxi --- oto nasze główne dzisiejsze zadanie bojowe. Trasa liczy sobie 17 km po ,,szutrze'', a po drodze trzeba zaliczyć parę podjazdów 17\% (i do tego wiele mniejszych). Z początku wyglądało to groźnie: ,,17\%?! Toż to prawie pionowa ściana!'' Nic bardziej mylnego. Owszem, na jednej ściance trzeba było podprowadzić rower, lecz generalnie bardzo dobry stan nawierzchni (niemal jak asfalt) umożliwiał sprawne pokonanie większości podjazdów.

\img{./photos/x-s-2014-08-01_12-46-26__149.jpg}{oxi_17_perc}{Podjazd 17\%? Pff…}
\img{./photos/x-s-2014-08-01_14-36-57__233.jpg}{oxi_kasia}{Kasia dzielnie walczy ze stokiem}

W związku z nadchodzącą ulewą zaraz na krzyżówce z \road{1} urządzamy sobie piknik pod płachtą --- tak tak, ta wożona dotychczas bez celu płachta budowlana wreszcie znajduje praktyczne zastosowanie! Cztery osoby bez problemu mogą się pod nią wygodnie schować, co jeszcze nie raz będziemy wykorzystywać…

Gdzieś w którejś z zamieszczonych w internecie relacji można było wyczytać, że ,,zjazd Öxi do Egilsstaðir to sama poezja --- 50 km zjazdu''. Błąd. Po drodze liczne podjazdy, momentami konkretne (choć może nie 15 czy 17\%), a że akurat wieje wiatr --- standardowo --- w twarz, to trzeba dokręcać aby utrzymać prędkość. Od czasu do czasu trochę pada. Mimo to nie narzekamy na pogodę, bo w porównaniu z warunkami na przełęczy było cudnie. Öxi niby ma marne 532 m n.p.m., lecz tak pizgało i było tak chłodno, że smarki niemal zamarzały. Nagroda czekała dopiero tuż przed miastem --- słońce zaświeciło tak intensywnie, że można się było opalać. Bez zaciskania zębów z zimna.

\img{./photos/x-s-2014-08-01_16-07-31__155.jpg}{sheet_lunch}{Pierwszy podczas wyjazdu posiłek pod płachtą}

Na kemping zajeżdżamy dopiero o 17:50 i niezwłocznie przystępujemy do realizacji planu pod tytułem ,,dzień turysty'': ciepły prysznic, pranie, gotowanie na spokojnie, wieczorne piwo (Viking Light --- czy to wciąż jeszcze można nazwać piwem? Smakuje jak lekko gazowana woda z delikatną nutką chmielu!) przegryzane orzeszkami… o prądzie (a więc i wifi) nie wspominając.

Gdy idziemy spać (tj. koło północy) Kasia staje przed namiotem, przygląda się mu uważnie i zdziwiona pyta: \emph{,,Ej, dlaczego on taki sztywny?!''} Podchodzi bliżej, maca, patrzy --- a na jej palcach roztapiają się płatki lodu. Z powodu przymrozka jakaś para porzuciła swój namiot i umościła się na materacu  w łazience, tuż przy grzejniku.

\hint{Na Islandii nigdy nie należy czynić żadnych założeń odnośnie pogody. Szczególnie nie próbować przewidywać kierunku wiatru --- lepiej przyjąć, że zawsze wieje w twarz. Co za tym idzie, wszelkie planowanie bądź ekstrapolacja czasu przejazdu mija się z celem…}

(Luźna impresja.) Krajobrazy islandzkie są skrajnie różne od tego, co znamy z ,,Europy kontynentalnej'' --- to, że łatwiej tu o spotkanie z owcą niż z człowiekiem to żadna nowość (patrz: liczne górskie regiony Walii czy holenderskie wybrzeże Morza Północnego), ale brak miejscowości przy drodze (a w zasadzie --- czegokolwiek!) to już zupełnie inna bajka! Autentycznie, wszelkie zabudowania na islandzkiej prowincji oddalone są od \road{1} co najmniej o 1 km, co dodatkowo potęguje wrażenie jazdy przez pustkowia.

\hint{Na islandzkich kempingach w zasadzie nie trzeba pilnować swoich rzeczy. Podobnie nie trzeba przypinać rowerów idąc do sklepu bądź w góry na wycieczkę. Istne szaleństwo! Według mnie wytłumaczenie tego fenomenu może być tylko jedno --- miejscowych prawie nie ma (a nawet jak są to nie kradną --- taka kultura), a przyjezdni to albo nadziani emeryci z Niemiec, Francji czy USA albo wagabundzi, którzy mają swój kodeks honorowy i nie rabują innych podróżnych :)}

\img{./photos/x-s-2014-08-01_21-44-01__65.jpg}{laundry_party}{Na zakończenie dnia --- impreza w pralni.}
\chapter*{18.08.}

\textellipsis i tak oto nasz ,,tabor cygański'' wyrusza z Ásbrú --- z pieśnią ,,My, cyganie'' na ustach i z Putem, robiącym za samolot, na czele\textellipsis Szczęśliwie podróż spod akademika na lotnisko odbyła się bez niespodzianek i tak oto pętla została definitywnie zamknięta. Teraz ostatnie nerwy: czy uda się wszystko spakować? czy rowery wejdą do pudła? czy nie odwołają lotu z powodu erupcji wulkanu?

Gdy tak siedzieliśmy na lotnisku przepakowując się, przed wejście zajechała karetka pogotowia. Pierwsza myśl --- oho, wypadek! Ale nie, po chwili drzwi od przedziału transportowego otwarły się i stanął w nich ratownik z walizką w ręku. Druga myśl --- może szef jakiegoś szpitala jedzie na wczasy i karetka robi za taksówkę? W tej chwili za ratownikiem powoli zaczął gramolić się\textellipsis nasz znajomy Anglik! Coś nam się jednak nie zgadzało. Dopiero po chwili spostrzegliśmy, że nasz druh paraduje w kołnierzu ortopedycznym. Konsternacja. Co się stało, że gość którego widzieliśmy niecałe dwa tygodnie temu w dobrej formie, nagle wylądował w takim stanie? Ano wybrał się pewnego razu na wycieczkę bez sakw, gwałtownie zahamował i przeleciał przez kierownicę, nadwyrężając kręgosłup. Mówił, że z początku leżał parę minut wręcz bez czucia w rękach i nogach\textellipsis Teraz wraca do Wielkiej Brytanii, gdzie przejdzie półroczną rehabilitację. Rokowania są dobre.

Berlińczycy mieli niezłą łamigłówkę, jak spakować cały swój bagaż do odprawy, a wszystko to z chęci zaoszczędzenia pieniędzy. Na drogę powrotną wykupili tylko dwa bagaże rejestrowe, a jest ich trójka. Ileż było prób ważenia, ile przymiarek z cyklu ,,co wziąć do bagażu podręcznego''! Te kombinacje zajęły tyle czasu, że w kolejce do odprawy towarzystwo ustawiło się dopiero pół godziny po rozpoczęciu odprawy.

\img{./photos/x-s-2014-08-18_10-14-20__114.jpg}{my_cyganie}{My, Cyganie\textellipsis}
\img{./photos/x-s-2014-08-19_06-18-13__116.jpg}{on_the_airport}{\textellipsis i my koczownicy.}

Gdy zarówno bagaże jak i rowery przeszły kontrolę bezpieczeństwa odetchnęliśmy wszyscy z ulgą --- ilość spraw, które mogą ,,pójść nie tak'' została ograniczona do minimum. Zdążyliśmy jeszcze zjeść pożegnalne drugie śniadanie na posadzce terminalu, zakupić zapas trunków w sklepie wolnocłowym i\textellipsis nadszedł czas pożegnań! Berlińczycy w swoją stronę, ja w swoją.

Przez pierwsze pół godziny lotu wprost nie mogłem odkleić nosa od szyby. Widzieć w mgnieniu oka trasę, którą mozolnie pokonywaliśmy przez cały pierwszy tydzień --- bezcenne! Co więcej, obserwowane z nieba sandury czy lodowce robią jeszcze większe wrażenie!

Na warszawskim lotnisku Okęcie od razu poczułem, że wróciłem do Polski --- na odbiór bagażu nasz rejs czekał blisko półtorej godziny! Dramat. Gdy już wreszcie zajechałem na Dworzec Centralny, znów miałem sytuację rodem z \emph{Misia}\footnote{Kultowa komedia w reżyserii Stanisława Barei, ukazująca absurdy PRL-u.}. Pani w kasie IC mówi, że nie może mi sprzedać biletu dla osoby z rowerem, bo już nie ma miejscówek. Pytam się, czy jest w stanie sprzedać mi bilet tylko dla jednej osoby (bez roweru) --- tak, oczywiście, żaden problem. Tylko że na miejsca siedzące też nie ma miejscówek. Czyli ostatecznie w kasie kupiłem coś uprawniającego mnie do zajęcia miejsca w pociągu, a potem u konduktora zakupiłem bilet na rower (bez dopłaty, yey!)

Miło znów podziwiać polskie krajobrazy. Mimo chłodu nocy i mimo padającego momentami deszczu, stoję przy otwartym oknie i wpatruję się w gęste lasy, obserwuję mgły ścielące się po szuwarach\textellipsis

O 5:20 pociąg zatrzymał się na dworcu Kraków Główny. Jeszcze tylko po raz ostatni ,,okulbaczyć'' mego konia mechanicznego, przejechać ostatnie kilometry i\textellipsis \emph{Home, sweet home}! Wszędzie dobrze, ale w domu najlepiej! Dla mnie islandzka eskapada dobiegła końca.

%TODO zdjęcie z pociągu

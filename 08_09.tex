\chapter*{09.08. --- Wokół Myvatn}

Rano budzi nas słońce --- tak nas zaskoczyło, że aż nie wiemy co z tym fantem zrobić! ;-) Masowo suszymy ubrania z wczoraj i konserujemy przerdzewiałe łańcuchy.

Zawitaliśmy do informacji turystycznej koło Samkaupa, by zasięgnąć języka ,,co warto zobaczyć''. Część atrakcji odpadła w przedbiegach: ze względu na aktywność tektoniczną temperatura wody w grocie Grjótagjá (kręcono w niej fragment jednego z epizodów Gry o Tron) znów podniosła się do 60 °C (kąpiel grozi nie tyle ,,ugotowaniem jajek'', co poważnym poparzeniem bądź wręcz śmiercią), Krafla znajduje się za daleko (i za wysoko…), rzeczy podobne do Hverarönd już widzieliśmy… Ostatecznie postanawiamy zwiedzić tylko to, co jest łatwo osiągalne z drogi naokoło jeziora.

Pierwszy punkt programu: formacje skalne w Dimmuborgir. Jest przyzwoicie, choć bez szału. W sumie natrafiliśmy na dwa ładne miejsca --- ,,oczko'' w skale i tzw. Kościół. Jak to dobrze, że za wstęp nie trzeba płacić, bo wtedy byłoby nam żal wydanych pieniędzy ;)

\img{./photos/x-s-2014-08-09_12-33-43__88.jpg}{dimmuborgir}{Dimmuborgir}

Kolejną atrakcję stanowi rezerwat Höfði --- ze znajdującego się w jego sercu pagórka roztaczają się piękne widoki na wysepki na Mývatn. Choć w sumie… dla Islandczyków największą frajdą musi być możliwość oglądania takiego skupiska prawdziwych, wysokich drzew!

Ostatnie miejsce, które zwiedzamy, to pseudo-kratery w Skútustaðir. Moim skromnym zdaniem o wiele, wiele lepiej prezentują się one na zdjęciach z lotu ptaka, no ale skoro już tu zajechaliśmy… Podczas tego spaceru spotkaliśmy dwie intrygujące osoby. Pierwszą była starsza Amerykanka (prawdopodobnie z prowincji), która słysząc naszą mowę zawołała do męża: \emph{,,Bob, look! Polish people!''} Bob ani na chwilę nie przerwał robienia zdjęć, nie skwitował tego choćby jednym słowem, więc Amerykanka zaczęła tyradę (do siebie samej) o tym, jak to chciała by już być w domu i że lot na Islandię taki długi… Drugą osobą był Polak z Mazur mieszkający od 7 lat nad Mývatn. Rozbroiło nas jego stwierdzenie: \emph{,,Mieszkam tu już tyle lat, a codziennie zauważam coś nowego!''} Co takiego można tu nowego zobaczyć --- nowy kamyk przy drodze?

\img{./photos/x-s-2014-08-09_16-03-32__215.jpg}{myvatn}{Myvatn}
\img{./photos/x-s-2014-08-09_16-10-39__318.jpg}{forest}{Na spacerze w bodaj jedynym (!) lesie na Islandii.}

Karolina ma problemy z tylną przerzutką --- właściwie nie może wrzucić normalnie 6-8, jedyne co pozostaje to ,,ręczna skrzynia biegów'', czyli ciągnięcie za linkę na rurze od ramy. Interesujące, że objawy nasiliły się, gdy po interiorze Karolina usunęła brud z przerzutki. Wcześniej, mimo zapiaszczenia, działały względnie dobrze. Aż ciśnie się na usta znana mądrość ludowa: \emph{Częste mycie skraca życie!}

Tuż za pierwszym podjazdem, nad jeziorem Másvatn --- zdrada! Dopadła nas krótka, lecz tak intensywna ulewa, że nim zdążyliśmy się oporządzić --- już mamy wszystko mokre: buty, spodnie, polary… Oczywiście jak zwykle Put przezornie założył co trzeba na samiutkim początku, gdy tylko poczuł pierwsze krople. Nasza trójka spróbowała ucieczki przed deszczem na drugi brzeg jeziora, co okazało się  fatalne w skutkach. Zrezygnowani, ponownie puściliśmy w ruch płachtę i --- siadłszy w przydrożnym rowie --- zaczęliśmy konsumować ciastka w oczekiwaniu na dalszy rozwój wypadków (pogodowych). Tym razem prognoza pogody się nie sprawdziła --- miało padać dopiero pod wieczór, a tu 16:00 i już woda!

Na drodze do Akureyri naliczyliśmy łącznie trzy podjazdy, wszystkie długie, równo nachylone, asfaltowe… Jazda po czymś takim to sama przyjemność, bo wiadomo że na szczycie każdego podjazdu czeka nagroda w postaci pięknego, równie długiego zjazdu! Podobnie jazda doliną koło Goðafossu nie wymęczyła nas ani trochę, gdyż droga biegła w zasadzie po równym.

Lecz cóż może być większą nagrodą dla utrudzonych rowerzystów niż możliwość podziwiania pięknego zachodu słońca nad fiordem Eyjafjörður? Załapaliśmy się na niego dosłownie w ostatniej chwili --- gdyby wspinaczka na przełęcz zajęła nam choćby 5 minut więcej, to nic byśmy nie zobaczyli! Daliśmy przykład kierowcom samochodów --- najpierw zatrzymała się niemiecka furgonetka, potem jacyś Francuzi, dalej dwa auta z Polski… Wszyscy z opuszczonymi szybami i wystawionymi aparatami fotograficznymi, uwieczniają opuszczającą się za chmury i wierzchołki gór pomarańczową kulę.

\img{./photos/x-s-2014-08-09_19-12-20__89.jpg}{gothafoss_dinner}{Obiad przy Goðafossie}
\img{./photos/x-s-2014-08-10_00-02-44__327.jpg}{eyjafjorthur_sunset}{Zachód słońca nad Eyjafjörður}
%TODO: zadbać, by te zdjęcia faktycznie pojawiły się pod tym akapitem

Niezbyt chce nam się jechać dziś do samego Akureyri, więc rozbijamy się na miedzy, gdzieś jakieś 15 km od miasta. Sądząc po śladach koło strumienia nie byliśmy pierwszymi, którzy wpadli na ten pomysł :-D

Rozbijając namioty żartujemy sobie z Putem: ,,A może by rozbić namiot tak, jak Pan Rusek polecał koło źródeł --- tuż nad strumieniem, żeby rano było łatwo załatwić ,,dwójkę''? Dziewczyny mogłyby sobie tak ustawić namiot, by mieć tam wejście ,,od kuchni''…''
%TODO: wytłumaczyć, o co chodzi z <dwójką>

\hint{Żarty żartami, ale z wiadomych względów naprawdę warto zawczasu jasno ustalić podział strumienia na odcinki: toaleta,  łazienka i kuchnia.}

Ogólne spostrzeżenie po dniu dzisiejszym: stajemy się zblazowani, jeżeli chodzi o atrakcje turystyczne Islandii. Po blisko trzech tygodniach jeżdżenia po tej wyspie coraz ciężej o coś naprawdę nowego…

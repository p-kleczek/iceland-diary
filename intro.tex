\chapter*{Tytułem wstępu}

Skąd wziął się pomysł, by wyruszyć na miesięczną wyprawę rowerową w tak niegościnne rejony? Wszystko zaczęło się rok wcześniej, w wakacje 2013 r. Wtedy to Put, Karolina i mój brat odbyli swoją pierwszą poważną eskapadę z sakwami --- do Norwegii. Tam też zachwycili się skandynawskimi krajobrazami i gościnnością ludu norweskiego, a do tego należy dodać, że także pogoda im dopisała. Stąd gdy zaczął zbliżać się ten moment, kiedy należało rozpocząć obmyślanie planów na lato roku 2014, ktoś z tej trójcy rzucił hasło ,,Na Islandię!''.

W żadnym wypadku nie była to decyzja przemyślana ni choć odrobinę racjonalna. Po prostu ktoś rzucił hasło, reszta przez aklamację podchwyciła i tak to się dalej niejako samo potoczyło. Żadnych szczegółowych przygotowań, trasa ,,na oko'' wyznaczona na Google Maps i dużo optymizmu --- taka miała być nasza recepta na sukces.

Czy warto było? Przeczytaj tę opowieść o walce z własnymi słabościami i przeciwnościami losu, jaką czwórka wędrowców musiała stroczyć na dalekiej islandzkiej ziemi. Być może ona da Ci odpowiedź, na to jedno kluczowe pytanie.

\newpage

\chapter*{Dramatis Personae}

Aktorami w tym dramacie jest czwórka młodych ludzi. Koleje losu każdego z nich są diametralnie różne, lecz to, co ich połączyło, to miłość do dwóch kółek i sakw --- pasja podróżowania po Europie rowerem, z całym dobytkiem ,,na pace''.

Oto skład naszej ekipy (wraz z hm\textellipsis ksywkami):
\begin{itemize}
	\item[-] Karolina
	\item[-] Kasia (Cybulka)
	\item[-] Tomek (Put)
	\item[-] Paweł (Kłeczi)
\end{itemize}

{\small PS. Ponieważ imiona i ksywki będą stosowane wymiennie, dlatego polecam Tobie, drogi Czytelniku, poświęcić teraz chwilę czasu na ich zapamiętanie.}

%TODO: opisać poszczególne osoby

\vspace{1cm}

Zdjęcia, którymi okraszona została ta opowieść, zostały wykonane przez Karolinę i Tomka. Chcę im gorąco podziękować za trud, który wkładali podczas trwania wyprawy, by udokumentować wszystkie godne zapamiętania momenty!



\chapter*{31.07.}

\section*{Na grani}

Jeszcze będąc w Höfn zawitałem do tamtejszej informacji turystycznej z pytaniem: co warto zobaczyć w okolicy? Pan chwilę poskrobał się w głowę, mój pomysł wyprawy w górę \road{F980} skwitował ,,szaleństwo, droga zalana'', w końcu zasugerował spacer do wąwozu Hvannagil (ten region Islandii nazywa się Lónsöræfi). Był nawet na tyle miły, że od razu wydrukował mi mapę szlaku i instrukcje jak odnaleźć jego początek.

Pierwszą próbę zobaczenia wspomnianego wąwozu podjąłem wczoraj wieczorem, lecz trochę pobłądziłem i ostatecznie dałem sobie spokój. Dziś skoro świt postanowiłem spróbować raz jeszcze, tym razem idąc najpierw w górę rzeki Jökulsá í Lóni.

Pierwszą niespodzianką było to, że spacer odbywał się dnem kanionu. Fakt, niby płynęło tam parę strumieni, lecz żaden nie był na tyle głęboki czy rwący by nie dało się go przekroczyć idąc po kamieniach lub przeskakując z jednego brzegu na drugi. Późniejsza wspinaczka na ścianę kanionu również nie nastręczała większych problemów i dlatego w ledwie 1-1,5 godziny byłem już za połową trasy. Popatrzyłem na okoliczne szczyty --- piękne, wyraźnie górujące nad okolicą, aż proszą się by z ich wierzchołków podziwiać ziemię u stóp. Spojrzałem na zegarek --- była 7:30. Reszta towarzystwa obudzi się pewnie dopiero za ponad dwie godziny. Niewiele myśląc powziąłem decyzję: \emph{,,A, zdobędę ten najbliższy!''} Dopiero w Krakowie udało mi się gdzieś wyszperać informację, że ta kupa kamieni, na którą przez blisko 40 minut wdrapywałem się na czworaka, ma swoją nazwę --- Grakinnartindar. Wędrówka była wyczerpująca, bo każdy krok w górę wiązał się z równoczesnym zsunięciem pół kroku w dół --- takie to było rumowisko. Gdzieniegdzie trafił się kamień większy niż pięść i wtedy można było na chwilę przystanąć i odsapnąć, póki i on nie zaczynał się osuwać razem ze mną. Niemniej gdy wreszcie osiągnąłem wierzchołek, to rozległa panorama na ośnieżone łańcuchy górskie, na rozczapierzone ,,palce'' u ujścia Jökulsá í Lóni i na wąwóz Hvannagil zrekompensowały mi wszelkie trudy (z dwukrotnym przekraczaniem kanionu ,,na dziko'' włącznie). Miłym akcentem była możliwość --- po raz pierwszy w życiu --- naprawdę długiego zbiegania ze szczytu ,,na azymut'' i to zupełnie bez patrzenia pod nogi. Te kamyki (właściwie tłuczeń wysypywany pod tory kolejowe) były tak luźno związane z podłożem, że bieg przypominał ,,płynięcie''. Ryzyko skręcenia kostki: znikome.

\img{./photos/x-s-2014-07-31_10-44-48__131.jpg}{winner_triumph}{Tryumf zwycięzcy}

\curiosity{To, że ten szczyt nazywał się akurat Grakinnartindar dowiedziałem się z portalu \href{http://www.islamicfinder.org/prayerDetail.php?country=iceland\&city=Grakinnartindar\&id=4471\&lang=}{Islamic Finder}, który podaje godziny modlitw w danym miejscu na ziemi. Porównałem współrzędne tam podane z tym co pokazywały Google Mapsy --- zgadza się!}

\hint{
Krótki opis trasy z ,,przewodnika'' otrzymanego w Höfn: Standardowa trasa trwa ok. 4 godz. Zaczyna się w Stafafell, koło pensjonatu i kościoła --- żwirowa droga prowadzi w górę wzgórza do małego kurnika. Potem należy podążać za ścieżką wydeptaną przez owce na szczyt wzgórza, przekroczyć płot i znów iść po owczej ścieżce --- aż do kanionu. Powrót brzegiem rzeki Jökulsá í Lóni. \newline \processifversion{PDF}{Mapę okolic Stafafell znajdziesz tutaj: \url{http://www.stafafell.is/uploads/8/3/3/1/8331287/5674623_orig.jpg}}
\processifversion{HTML}{Mapę okolic Stafafell znajdziesz \href{http://www.stafafell.is/uploads/8/3/3/1/8331287/5674623_orig.jpg}{tutaj}}
}.

%TODO: tytuł sekcji
\section*{xxx}

Dziewczyny były na tyle odważne (albo zdesperowane), że umyły się w tej mętnej polodowcowej rzece. Niby to nie żadne ścieki tylko drobinki piasku i ziemi, ale i tak moja reakcja była dość jednoznaczna --- \emph{Yy… fuuuj…} ;-)

Od pewnego Polaka, przypadkowo spotkanego w Nettó w Höfn, dowiedzieliśmy się o istnieniu ,,sklepu na CPN-ie w Djúpivogurze''. Zależało nam na zakupach, więc dokładaliśmy starań, by zdążyć tam na jakąś rozsądną porę. Jakież było nasze zdziwienie, gdy skoro tylko zajechaliśmy tam na 15:40, pani ekspedientka powitała nas od razu tekstem: \emph{,,Róbcie proszę zakupy szybko, bo zaraz zamykamy.''} Że co proszę? Miejscowość liczy 350 dusz, a sieciowy supermarket Samkaup zamykają w dzień powszedni o 16:00? Naprawdę?

\img{./photos/x-s-2014-07-31_19-57-52__224.jpg}{dead_deer}{Hm… rzeźba ogrodowa w islandzkim stylu?}

\img{./photos/x-s-2014-07-31_21-33-03__148.jpg}{berufjorthur}{Majestatyczne szczyty okalające Berufjörður}

No nic, zakupy zrobione, zaczęliśmy szukać miejsca na przygotowanie obiadu. Po dłuższej naradzie wybraliśmy obiad ,,na krzywy ryj'' w kuchni na lokalnym kempingu. Zachęciła nas do tego informacja, że jego właściciele rezydują w hotelu oddalonym o pół kilometra, zatem ryzyko że ktoś nas wyrzuci jest znikome. Tam właśnie, w tej przytulnej kuchni, poznaliśmy naszych późniejszych nieodłącznych towarzyszy na rowerowej trasie: Polkę z Poznania (też wraca do Berlina w dniu 18. sierpnia) oraz Krisa i Nelly --- parę Holendrów których już jutro powinniśmy mijać w drodze do Egilsstaðir .

Ostatnie 10 km drogi to znów wygwizdów, a nam --- po sutym obiedzie --- niezbyt chce się pedałować… A, no i jeszcze w pewnym momencie skończył się asfalt na \road{1}, co dodatkowo psuło nam przyjemność z jazdy z pełnym żołądkiem.

\img{./photos/x-s-2014-07-31_22-48-21__226.jpg}{malbik_endar}{Cóż, nawet na \road{1} czasem ,,Malbik Endar''…}
%TODO: wytłumaczyć w stopce - koniec asfaltu

Wieczorem doświadczamy standardowego problemu z kempingami zaznaczonymi na mapie --- mianowicie ich braku. Niby na niemieckiej mapie mamy przy skręcie na Öxi jak byk zaznaczony czerwony namiocik, lecz zagadnięty o to tubylec w sile wieku stwierdził: \emph{,,Kemping? Mieszkam tu od dziecka i nigdy żadnego kempingu nie było!''} Potem poradził albo jechać dalej 10 km po \road{1} albo cofnąć się z 5 km. My oczywiście zrobiliśmy po swojemu i zanocowaliśmy na dziko.
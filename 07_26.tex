\chapter*{26.07. - Wyspy Zachodnich Ludzi}

Rano deszcz i pochmurne niebo - standard. Szczęśliwie koło 9:30 coś się poprawia, gdzieniegdzie przeziera błękit i promienie słońca. W oczekiwaniu na pełne wypogodzenie robimy pranie, reperujemy sprzęt…

\img{./photos/x-s-2014-07-26_14-00-40__54.jpg}{heimaey_sunshine}{Istna rzadkość na Islandii - słońce!}

W południe ruszamy na południowy kraniec wyspy (Stórhöfði), by pooglądać maskonury (ang. puffins, nie mylić z muffins). Zmęczyliśmy się odrobinę pokonując strome serpentyny na ostatnim odcinku, lecz wysiłek definitywnie się opłacił. Z klifów maskonury widać było jak na dłoni - i to nie pojedyncze sztuki, a całą kolonię! Urządzamy więc swoiste safari fotograficzne, a potem - jako że niebo zrobiło się bezchmurne - leżymy plackiem wsłuchując się w szum fal…

Wracaliśmy już na kemping, gdy nagle Cebulka stwierdziła, że “chyba coś dziwnego dzieje się jej z bagażnikiem”. Faktycznie, wypadnięcie paru śrubek i skrzywienie jednego ze wsporników można określić mianem “czegoś dziwnego” ;) Nie ma co ukrywać, nastroje zrobiły się minorowe - bez sprawnego bagażnika nie ma co myśleć o kontynuowaniu wyprawy! Lecz wtedy przypomniałem sobie, jak to podczas jednej ze swoich wypraw ratowałem swój bagażnik… Pojechaliśmy więc do “wioski” i  w pierwszym lepszym podmu poprosiliśmy o młotek - wystarczyło parę klepnięć i… bagażnik był jak nowy. (To właśnie jedna z zalet bagażników aluminiowych.) Śrubki pożyczyliśmy z roweru Karoliny, która miała zamontowany tylko jeden kosz na bidon - dwie sztuki śrubek były więc zbędne. Uff… możemy kontynuować podróż!

Ponieważ czasu do odpłynięcia promu było jeszcze sporo, wyprawiliśmy się na okoliczne szczyty celem oglądania jeszcze większej liczby maskonu[11.08.]

Ciekawostka. Czy wiesz, że butelkowaną islandzką kranówę… - o przepraszam, wodę źródlaną - można nabyć w licznych punktach sprzedaży detalicznej w w USA? (Tekst promocyjny z etykiety icelandspring.com.)

Ostatnio powstał swoisty standardowy zestaw tematów do posiłków: ile kilometrów mamy do przejechania którego dnia? ile jeszcze noclegów i gdzie one wypadają? ile śniadań i obiadów czeka nas przed najbliższym sklepem? co zjemy na który posiłek (zwłaszcza na obiad, bo trzeba wybrać typ zapychacza i wkładki mięsnej)? Zgoda, to ważne sprowany i należy je omówić, lecz ilość czasu którą obecnie na to poświęcamy jest olbrzymia.

Kolejny dzień, kiedy chociaż z rana świeci słońce - pozytywnie nastawia do pedałowania! Choć w sumie… poranek taki piękny, a do przejechania tylko jakieś 70 km, więc wszystko toczy się iście emeryckim tempem (z leżeniem na trawie opalaniem się oraz skakaniem na dużej dmuchanej trampolinie włącznie). Wyjeżdżamy na trasę późno jak nigdy (za wyjątkiem dnia turysty:), bo około 13:00, ale wszyscy zadowoleni z miło spędzonych chwil.

Zawitaliśmy jeszcze o supermarket, by kupić “chleb i mleko”, co szybko przerodziło się w szał zakupów - gdy wreszcie udało się wszystkie puszki z owocami i paczki ciastek poutykać po sakwach, te znó są konkretnie wypchane :)

Droga [733] oraz [F35] (na odcinku do Áfangi) to cud-miód-poezja: solidnie utwardzona (chyba jakimś środkiem wążącym podłoże), łagodnie nachylona (począwszy od okolic elektrowni wodnej Blanda nie ma już stromych podjazdów) i… tylko momentami trochę dziurawa, ale tak do przeżycia.

Po odbiciu z [733], tuż za mostem, zapragnęliśmy przyrządzić kolejny polowy obiad. Doświadczenia z Askji zaprocentowały, znów konstruujemy wiatrołap. Tym razem świeci słońce, nigdzie nam się nie spieszy, więc między innymi znalazł się czas, by Kasia przeszła pełnowymiarowy kurs otwierania konserw przy użyciu scyzoryka.

Po obiedzie zgodnie stwierdzamy, że coraz ciężej uchwycić moment przejścia między uczuciem głodu, nasycenia i przejedzenia. Zazwyczaj objawia się to tym, że najpierw jemy i jemy, a po takim posiłku ledwo dajemy radę kręcić pedałami - każdy stromszy podjazd grozi zwrotem.

Jazda w kompletnej ciszy to nowe, osobliwe doświadczenie. Nie mam na myśli, że tylko pedałujemy nie odzywając się do siebie nawzajem. Nie, to nie tak! Samochodów brak, ludzi brak, owadów brak, flauta… Autentycznie bywało że gdy przystawaliśmy (i milkł chrzęst nienasmarowanych łańcuchów), aż zaczynało dzwonić w uszach!

Późniejszym popołudniem było tak upalnie i bezwietrznie, że nawet Kasia dołączyła do grona osób podróżujących w krótkim rękawku! Taki stan nie utrzymał się jednak długo, dokładniej urzymał się przez 10 minut ;)

A gdy później Put i Karolina polowali na szczycie wzniesienia przed Áfangi na zachód słońca, wiało tak mocno i takim zimnym wiatrem, Kasia i ja skapitulowaliśmy (innymi słowy nie dotrwaliśmy do zachodu) i udaliśmy się do pobliskiego “kurortu” w celu ogrzania zgrabiałych palców.

Áfangi niecno nas rozczarowało. Niby cena za nocleg przyzwoita, bo po 1200 kr od łebka (a prysznic i jazkuzzi w cenie), lecz mimo to szału nie ma. Zdania “czy zostajemy” są podzielone, zatem decyzję podejmujemy rzucając monetą. Choć z rzutu wyszło, że nie zostajemy, to i tak ostatecznym argumentem było to, że jedyny fragment trawiasty zdatny do rozbicia się znajdował się… tuż obok stajni! Odjechaliśmy więc kilometr dalej i rozbiliśmy się parę metrów od [F35], na względnie płaskim i nie-kamienistym skrawku ziemi.
rów. Te ptaszki w ogóle się nas nie boją - z początku skradamy się ostrożnie, zamierając po każdym kroku, by żadnego nie spłoszyć, lecz nawet po podejściu na odległość 1 metra one wciąż siedzą jak siedziały! Co nas zszokowało, to fakt że na kempingu widzieliśmy tylu “full-pro” turystów - w porządnych butach górskich, z kijkami i z plecakami - a na tych górskich ścieżkach (raptem 20-30 min od kempingu) nie spotykamy ani jednego. Choć w sumie… tym lepiej dla nas!

\img{./photos/x-s-2014-07-26_17-50-56__60.jpg}{puffin_hunter}{Karolina i maskonur}

Postanowiliśmy opuścić wyspę ostatnim promem. Zajeżdżamy więc do kasy na jakieś 20 minut przed jego odpłynięciem, chcemy kupić bilety, a tu… lipa! Nie ma wolnych miejsc! Możemy co najwyżej zapisać się na listę oczekujących (i liczyć na to, że ktoś z osób z rezerwacją się nie pojawi)… Przeżywamy chwile grozy - zwłaszcza, że nie jesteśmy pierwszymi na tejże liście oczekujących - i zastanawiamy się już: Co to będzie? Co to będzie? No bo jak się nie uda zaokrętować, to czeka nas kolejny nocleg na wyspie… W końcu jednak miejsce dla nas się znalazło - hura!

\hint{Jeśli to możliwe - zawczasu zarezerwuj online miejsce na promie.
Od spotkanego w drodze do portu polskiego inżyniera-stoczniowca (“Znają mnie w każej tawernie i stawiają kawę czy herbatę!”) dowiedzieliśmy się ciekawostki: Na Heimaey po erupcji z 1973 r. na nowopowstałe pola lawy zrzucano ze specjalnie przystosowanego samolotu łubin - jedną z nielicznych roślin odpornych na islandzki klimat i tak surową “glebę”. Tenże łubin po latach miał dać zaczątek faktycznej glebie. Dziś podłoże jest już na tyle żyzne, że gdzieniegdzie rosną nawet niewielkie drzewa iglaste. Więcej o wysiłkach Islandczyków podejmowanych w celu rekultywacji (nie tylko Wysp Zachodnich Ludzi) można poczytać \href{http://www.land.is/english/images/pdf-documents/healing_the_landL.pdf}{tutaj}.}

\img{./photos/x-s-2014-07-26_23-05-25__115.jpg}{ferry_to_mainland}{Heimaey w promieniach zachodzącego słońca}

Wieczorem podjechaliśmy jeszcze wspólnie pod Seljalandsfoss, a potem rozdzieliliśmy się - Put, Kasia i Karolina pojechali na nocleg, a ja uparłem się, że chcę wybrać się na (nocną) wycieczkę ze Skógar do Þorsmörk przez przełęcz Fimmvörðuháls.
\chapter*{26.07. --- Wyspy Zachodnich Ludzi}

Rano deszcz i pochmurne niebo --- standard. Szczęśliwie koło 9:30 coś się poprawia, gdzieniegdzie przeziera błękit i promienie słońca. W oczekiwaniu na pełne wypogodzenie robimy pranie, reperujemy sprzęt…

\img{./photos/x-s-2014-07-26_14-00-40__54.jpg}{heimaey_sunshine}{Istna rzadkość na Islandii --- słońce!}

W południe ruszamy na południowy kraniec wyspy (Stórhöfði), by pooglądać maskonury (ang. \emph{puffins} --- nie mylić z \emph{muffins}). Zmęczyliśmy się odrobinę pokonując strome serpentyny na ostatnim odcinku, lecz wysiłek definitywnie się opłacił. Z klifów maskonury widać było jak na dłoni --- i to nie pojedyncze sztuki, a całą kolonię! Urządzamy więc swoiste safari fotograficzne, a potem --- jako że niebo zrobiło się bezchmurne --- leżymy plackiem wsłuchując się w szum fal…

Wracaliśmy już na kemping, gdy nagle Cebulka stwierdziła, że ,,chyba coś dziwnego dzieje się jej z bagażnikiem''. Faktycznie, wypadnięcie paru śrubek i skrzywienie jednego ze wsporników można określić mianem ,,czegoś dziwnego'' ;) Nie ma co ukrywać, nastroje zrobiły się minorowe --- bez sprawnego bagażnika nie ma co myśleć o kontynuowaniu wyprawy! Lecz wtedy przypomniałem sobie, jak to podczas jednej ze swoich wypraw ratowałem swój bagażnik… Pojechaliśmy więc do ,,wioski'' i  w pierwszym lepszym domu poprosiliśmy o młotek --- wystarczyło parę klepnięć i… bagażnik był jak nowy. Tak, to zdecydowanie jedna z zalet bagażników aluminiowych. Śrubki pożyczyliśmy z roweru Karoliny, która miała zamontowany tylko jeden kosz na bidon --- dwie sztuki śrubek były więc zbędne. Uff… możemy kontynuować podróż!

Ponieważ wciąż było sporo czasu do odpłynięcia promu, wyprawiliśmy się na okoliczne szczyty celem oglądania jeszcze większej liczby maskonurów. Te ptaszki w ogóle się nas nie boją --- z początku skradamy się ostrożnie, zamierając po każdym kroku, by żadnego nie spłoszyć, lecz nawet po podejściu na odległość 1 metra one wciąż siedzą jak siedziały! Co nas zszokowało, to fakt że na kempingu widzieliśmy tylu ,,full-pro'' turystów --- w porządnych butach górskich, z kijkami i z plecakami --- a na tych górskich ścieżkach (raptem 20-30 min od kempingu) nie spotykamy ani jednego. Choć w sumie… tym lepiej dla nas!

\img{./photos/x-s-2014-07-26_17-50-56__60.jpg}{puffin_hunter}{Karolina i maskonur}

Postanowiliśmy opuścić wyspę ostatnim promem. Zajeżdżamy więc do kasy na jakieś 20 minut przed jego odpłynięciem, chcemy kupić bilety, a tu… lipa! Nie ma wolnych miejsc! Możemy co najwyżej zapisać się na listę oczekujących (i liczyć na to, że ktoś z osób z rezerwacją się nie pojawi)… Przeżywamy chwile grozy --- zwłaszcza, że nie jesteśmy pierwszymi na tejże liście oczekujących --- i zastanawiamy się już: Co to będzie? Co to będzie? No bo jak się nie uda zaokrętować, to czeka nas kolejny nocleg na wyspie… W końcu jednak miejsce dla nas się znalazło --- hura!

\hint{Jeśli to możliwe --- zawczasu zarezerwuj online miejsce na promie. Przychodząc do terminalu nawet na 30 minut przed odpłynięciem promu może okazać się, że wszystkie zostały już wyprzedane (bądź właśnie zarezerwowane) wcześniej.}

\curiosity{Od spotkanego w drodze do portu polskiego inżyniera-stoczniowca (\emph{,,Znają mnie w każdej tawernie na Islandii i stawiają kawę czy herbatę!''}) dowiedzieliśmy się ciekawostki: Na Heimaey po erupcji z 1973 r. na nowopowstałe pola lawy zrzucano ze specjalnie przystosowanego samolotu łubin --- jedną z nielicznych roślin odpornych na islandzki klimat i tak surową ,,glebę''. Tenże łubin po latach miał dać zaczątek faktycznej glebie. Dziś podłoże jest już na tyle żyzne, że gdzieniegdzie rosną nawet niewielkie drzewa iglaste. \newline \processifversion{PDF}{Więcej o wysiłkach Islandczyków podejmowanych w celu rekultywacji (nie tylko Wysp Zachodnich Ludzi) można poczytać w dokumencie pod adresem  \url{http://www.land.is/english/images/pdf-documents/healing_the_landL.pdf}} \processifversion{HTML}{Więcej o wysiłkach Islandczyków podejmowanych w celu rekultywacji (nie tylko Wysp Zachodnich Ludzi) można poczytać \href{http://www.land.is/english/images/pdf-documents/healing_the_landL.pdf}{tutaj}}}.

\img{./photos/x-s-2014-07-26_23-05-25__115.jpg}{ferry_to_mainland}{Heimaey w promieniach zachodzącego słońca}

Wieczorem podjechaliśmy jeszcze wspólnie pod Seljalandsfoss, a potem rozdzieliliśmy się --- Put, Kasia i Karolina pojechali na nocleg, a ja uparłem się, że chcę wybrać się na (nocną) wycieczkę ze Skógar do Þorsmörk przez przełęcz Fimmvörðuháls.
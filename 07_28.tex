\chapter*{28.07.}

\section*{Obiad w Kirkjubæfarklaustur}

Od rana pogoda fatalna --- permanentna mżawka, a momentami faktycznie pada. Gdy jest tak chłodno i wilgotno, znacznie więcej energii człowiek zużywa więcej enrgii na ogrzanie siebie i --- siłą rzeczy --- szybciej robi się głodny. Trudno się więc dziwić, że gdy tylko zrobiliśmy zakupy w superkarkecie w Kirkjubæfarklaustur zapragnęliśmy zjeść ciepły obiad. Z początku projekt zdawał się nierealny, bo żadne miejsce w pobliżu nie miało potencjału na stanie się naszą kuchnią polową. I tu jak z nieba spadła nam pani z informacji turystycznej, która stwierdziła że “nie ma problemu, byście się rozgościli w jednym z naszych pokoi biurowych na piętrze”. Przyznam, że w takich warunkach --- wśród stosów segregatorów i drukarek --- jeszcze nigdy nie zdarzyło mi się gotować obiadu!

\img{./photos/x-s-2014-07-28_15-03-53__50.jpg}{ti_dinner}{Któż by się spodziewał takich luksusów?}

\section*{Pustkowia po raz drugi}

Obiad zjedzony --- dzień zaliczony. Pora ruszać dalej w drogę. Ta ponownie nie była przesadnie urozmaicona --- po raz kolejny już dziś jedziemy prostymi jak strzała odcinkami, tak że gdy po 10 czy 15 km nagle pojawia się zakręt, to prowadzący grupę niczym GPS komunikuje reszcie “Za 300 metrów lekko skręć w lewo.” Z mijanych osobliwości --- gość w koparce, który gładzi łyżką żwir wokół swojej maszyny. Komentarz Puta: “To taki ich program walki z bezrobociem?” --- bezcenny.

\img{./photos/x-s-2014-07-28_22-06-42__95.jpg}{sandur}{Sandury (ławka i… DRZEWO?)}

Droga może nie byłaby taka zła, gdyby nie mgła i silny frontowy wiatr. Zgoda, zdążyliśmy się już przyzwyczaić, że “na Islandii wieje zawsze przeciwnie do kierunku jazdy” {obrazek}, lecz że wieje w twarz nawetr po skręcie o niemal 180° (gdy zjechaliśmy z \road{1} w stronę kempingu)?! To już zakrawa na ponury żart.

Kemping w Skatafell można podsumować krótko --- zdzierstwo! Za nocleg liczą sobie 1000 kr, prysznic to wydatek kolejnych 500 kr, ładowanie komórki --- 200 kr, a do tego nie ma nawet gdzie wysuszyć rzeczy (bo czy drewniana buda ze sznurkami, lecz bez żadnego ogrzewania, liczy się jako suszarnia?).

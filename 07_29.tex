\chapter{29.07.}

Z rana wybraliśmy się oglądać sztandarową atrakcję Skaftafell - wodospad Svartifoss, którego ściany przypominają olbrzymie bazaltowe organy. Potem podeszliśmy jeszcze kawałek wyżej, na punkt widokowy Sjónarsker, z którego widać było sandury w całej ich okazałości.

Kasia opowiedziała nam o ciekawych zajęciach, w jakich uczestniczyła w Oslo podczas Erasmusa - cyklowi wykładów poświęconych zjawisku imigracji w Europie od strony społecznej. Między innymi poruszana była na nich kwestia praworządności. Otóż nacje Europy dzielą się na “czerwone” i “niebieskie”. Niebiescy generalnie ufają władzy, przestrzegają prawa, są przykładnymi obywatelami, czerwoni - przeciwnie. O ile niebiescy bardzo dobrze prosperują w okresie “pokoju”, o tyle w sytuacjach kryzysowych zwyczajnie głupieją - bo nie mają szczegółowych instrukcji co robić. Czerwoni wprost przeciwnie - na codzień ich krętactwo psuje system, lecz w trudnych chwilach ten brak poszanowania dla reguł i kombinatorstwo sprawia, że doskonale odnajdują się w trudnych chwilach. Polacy oczywiście są “czerwoni” :)

Tak więc - jako przedstawiciele “czerwonej nacji” - dokonaliśmy aneksji większości pralni na cele suszarni oraz - wbrew zakazowi - podładowaliśmy komórki w tamtejszych gniazdkach. Aż dziw, że nikt inny nie wpadł na taki pomysł! (Może to właśnie ta różnica w mentalności?) Mniejsza z tym, ważne że wreszcie rzeczy są suche!

\img{./photos/x-s-2014-07-29_13-44-46__150.jpg}{sandurs_from_above}{Sandury, sandury... aż po horyzont sandury! }

Z kempingu wyruszamy w dobrych nastrojach, przy w miarę stabilnej pogodzie. Kręcimy, kręcimy kilometry aż zajechaliśmy do Fjallsárlón. Była to pierwsza widziana przez nas laguna lodowcowa i zgodnie stwierdziliśmy, że to dość urokliwe miejsce. Szybko jednak pojawiły się głosy, że to jednak nie jest “ta laguna”, którą oryginalnie planowaliśmy zobaczyć. Woda jakaś taka mętna, fok brak i coś mało ludzi…

\img{./photos/x-s-2014-07-29_20-02-07__176.jpg}{lesser_lagune}{Mała laguna lodowcowa...}
\img{./photos/x-s-2014-07-29_21-09-00__114.jpg}{greater_lagune}{...i jej większa krewna!}

I faktycznie, do słynnej laguny Jökulsárlón należało jeszcze jechać przez kolejne 10 km. Tam, w promieniach zachodzącego słońca, mogliśmy podziwiać wszystko to, co pokazywały foldery reklamowe: sunące majestatycznie górki lodowe, polujące foki i mewy, tłumy turystów...

Pod wieczór wiedzieliśmy już, że że nie dojedziemy na zaplanowany kemping nieopodal Vagnsstaðir. Szybka kontrola zapasów wykazała, że brakuje nam kluczowego składnika porannej paszy - mleka - stąd dokonaliśmy jego interwencyjnego zakupu w przydrożnej restauracji (Hali Country Hotel). Choć podyktowana przez szefową restauracji cena była czarnorynkowa - 400 kr za litr - to każdy z nas w duchu cieszył się, że jutro z rana nie zajdzie konieczność konsumowania owsianej mamałygi na wodzie.

Ostatecznie zanocowaliśmy na pięknym, miękkim “mechowisku” 4 km za wspomnianą restauracją. Gdyby tylko tak nie wiało… Bo znów przy silniejszych podmuchach nasz męski namiot przypomina czapkę smerfa!
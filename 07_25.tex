\chapter*{25.07.}

W nocy kilkukrotnie budziło nas bębnienie deszczu o tropik namiotu --- tak silna była ulewa. Nad ranem przeszła, więc zarządziliśmy sprawne śniadanie i zbiórkę --- przy wciąż jeszcze w miarę stabilnej pogodzie. Niestety, nic co piękne nie trwa wiecznie… W chwilę po załadowaniu sakw “na pakę” zaczyna najpierw mżyć, potem kropić, a wreszcie --- lać. I tak właśnie --- w mniejszym lub (częściej) większym deszczu, przy wietrze “w mordę” i przy dużym natężeniu ruchu na \road{1} --- upływa nam podróż do miejscowości Hella.

W Helli czeka na nas oaza --- budynek z supermarketem sieci Kjarval. Oprócz niego znajduje się tam jeszcze mała piekarnio-cukiernia i… parę stolików dla odwiedzających! Anektujemy jeden z nich i urządzamy popijającym kawę w cukierni Islandczykom pokaz “Jak przygotować drugie śniadanie w warunkach turystycznych?” --- wyciągamy chleb i konserwy, ser żółty, nóż-kosę, ciastka… Oczywiście wszyscy kompletnie przemoczeni, tak że buty można wykręcać, pod stołkami tworzą się kałuże… Po chwili, ośmieleni naszym przykładem, pomysł pikniku podchwycili siedzący przy sąsiednim stoliku Francuzi --- też zaczęli robić kanapki…

Na jedzeniu i trzęsieniu się z zimna upływa nam pierwsze 1,5 godziny --- wtedy właśnie poszedłem do znajdującej się piętro niżej toalety i zauważyłem wiszący na ścianie… olbrzymi, gorący grzejnik! Radość nasza była przeogromna i kolejne 20 minut spędziliśmy rozwieszając na nim wszystko co tylko się dało, a następnie uprawiając swoistą gimnastykę, by znaleźć się jak najbliżej ciepła (i wysuszyć mokre wdzianka rowerowe, skarpetki na stopach…). W sumie siedzielibyśmy dłużej --- i wysuszyli nasze ubrania na pieprz --- lecz wygonił nas Put stwierdzając, że jak tak dalej pójdzie, to nie zdążymy na odpływający o 19:00 prom na Wyspy Zachodnich Ludzi. Ech…

\img{./photos/x-s-2014-07-25_17-15-16__25.jpg}{hella_heater}{Jest grzejnik --- jest impreza.}

Już mieliśmy ruszać spod sklepu, gdy naszą uwagę przykuła naklejona na jego witrynę mapa Islandii. Mapa jak mapa, ale po co na normalnej mapie ktoś miałby zaznaczać ostre podjazdy (>8\%) albo natężenie ruchu pojazdów? Okazało się, że natrafiliśmy na mapę \href{http://www.vegagerdin.is/media/upplysingar-og-utgafa/Cycling-map.pdf}{“Cycling Iceland --- Summer 2014”}, stworzoną --- jak sama nazwa wzkazuje --- z myślą o rowerzystach. Korzystaliśmy z niej intensywnie m.in. w kwestii pól namiotowych oraz lokalizacji sklepów “poza wsiami”.

\processifversion{PDF}{
\hint{Na stronie internetowej \emph{Vegagerdin} (taki ichniejszy Zarząd Dróg) możesz nie tylko pobrać aktualną mapę rowerową, ale także sprawdzić aktualne warunki pogodowe na podstawie pomiarów z przydrożnych stacji meteo. \newline Link: \url{http://www.vegagerdin.is/english}}
}

\processifversion{HTML}{
\hint{Na stronie \href{http://www.vegagerdin.is/english}{Vegagerdin} (taki ichniejszy Zarząd Dróg) możesz nie tylko pobrać aktualną mapę rowerową, ale także sprawdzić aktualne warunki pogodowe na podstawie pomiarów z przydrożnych stacji meteo.}
}

Odcinek Hella-Landeyjahöfn był w miarę znośny, jeśli chodzi o deszcz, bo dolało nas jeszcze tylko dwa razy. Na dobrą sprawę upodabniamy się do Islandczyków, którym ,,wisi'', czy mży bądź pada oraz czy wieje --- normalnie chodzą z dziećmi (malutkimi) albo grają w piłkę nożną czy też… golfa (!) przy takiej pogodzie. Bo w sumie --- co innego można robić, skoro tak wygląda większość dni? (Gdy historię z wyletnionym dzieckiem w wózku zabranym na spacer przy paskudnej pogodzie usłyszał znajomy mieszkający parę lat na Islandii, stwierdził że czasem w takich sytuacjach jacyś nadgorliwi mieszkańcy potrafią wezwać urząd ds. dzieci).

Tak więc skoro bardziej kropiło niż padało, Kasia założyła stuptuty dopiero wtedy, gdy woda zaczęła atakować od wnętrza buta jej skarpetkę… Potem testowała jeszcze patent z ubieraniem reklamówki na gołą stopę i dopiero potem skarpetki, lecz chyba bez spektakularnych sukcesów (oryginalnie chciała założyć worek na suchą skarpetkę, lecz znalezienie takowej okazało się przerastać naszą wolędziałania w chwili suszenia się w Helli). Przypomniała mi się więc historia z gotowaniem żaby: \emph{Gdy wrzucisz do wrzątku --- wyskoczy, lecz gdy wrzucisz do ciepłej wody i zaczniesz ją stopniowo podgrzewać --- ugotuje się}. I podobnie jest z niezakładaniem stuptutów: gdy nagle się rozleje --- zakładamy momentalnie, ale gdy najpierw mży, a dopiero potem zaczyna padać coraz bardziej i bardziej, to na początku myślimy “a, zaraz przejdzie”, a gdy już mamy całe przemoczone buty, to nawet nie chce nam się wyjmować ochraniaczy…

Na prom zdążyliśmy --- z powodu wiatru i deszczu --- dosłownie w ostatnich minutach. Mylące były wskazy, które najpierw --- po skręcie z \road{1} --- podawały odległość 11 km, by po 11 km poinformować nas, że do przejechania pozostało jednak jeszcze 3 km.

\img{./photos/x-s-2014-07-25_21-22-31__45.jpg}{ferry_to_island}{Rejs na Wyspy Zachodnich Ludzi}

Rejs promem obfitował w piękne widoki: zarówno samotnych, niezamieszkałych wysepek przed Heimaey (największą wyspą), jak też klifów i skalnych ścian przy samym wejściu do portu. Podobnie byliśmy oczarowani \href{http://www.tjalda.is/en/herjolfsdalur/}{kempingiem} --- jest on położony w niecce u podnóża gór (otaczają go z trzech stron, więc właściwie nie wieje), a namioty rozbija się pośród porośniętych mchem skałek. Kemping ten posiada wspólny salon z aneksem kuchennym, który --- jak zwykle --- jest schludny i funkcjonalny (kuchenka mikrofalowa, grzejniki). Z początku ucieszyliśmy się, bo nie było nigdzie widać recepcji --- czyżby nocleg w tak pięknych okolicznościach przyrody miał się odbyć “za frajer”? Nie, po dłuższej chwili pojawiła się dziewczyna robiąca za poborcę podatków, która zainkasowałą za ten luksus po 1300 kr od łebka, ech…

\img{./photos/x-s-2014-07-25_22-44-06__52.jpg}{heimaey_camping}{Kemping --- śródgórska, malownicza oaza spokoju}

Znów wszyscy są “wycięci” po intensywnym, pełnym wrażeń dniu --- Put pisze list do domu, Kasia bije rekordy w 2048, Karolina stara się zaplanować operację prania (wbrew pozorom, pod koniec dnia, to rzecz wymagająca!), a Paweł… Paweł spisuję wszystkie te małe i duże wydarzenia dnia dzisiejszego “ku pamięci potomnych” :)

Nocą wybrałem się na mini-wycieczkę górską. Wystarczy podejść jakieś 20 minut, by z grani Eggjar móc podziwiać leżące u stóp miasto. Maskonury chyba poszły już spać, szkoda…
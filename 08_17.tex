\chapter*{17.08.}

Tak ciężko jak dziś, to nam się jeszcze nigdy nie zbierało. Dolegujemy, przewracamy się z boku na bok, wymyślamy wymówki dlaczego nie chcemy wypełznąć z namiotów. Wszystko to z bardzo prostego powodu --- mentalnie wyjazd skończył się wczoraj, z chwilą wjechania na \road{1} przed Reykjavikiem. Skończyło się obcowanie z naturą i ,,przygoda'', biwak na kempingu gdzie oprócz nas obozuje z pół tysiąca osób nie ma klimatu, miejsce otoczenie również nie nastraja optymistycznie. Zresztą --- znów do przejechania mamy zaledwie 50 km, więc \emph{,,po co się spieszyć? co będziemy robić w Asbru?!''}

Wczesne popołudnie upływa leniwie na długich zakupach pamiątek w sklepach przy ulicy Laugavegur (to coś jak reykjawickie Krupówki). W sumie to dziewczyny kupowały, a ja z Putem siedzieliśmy na murku przed sklepem i ,,kontemplowaliśmy'' życie miejskie. Ktoś, kto by nas obserwował, mógłby z całą pewnością stwierdzić: ci dwaj goście mają \href{http://en.wikipedia.org/wiki/Thousand-yard_stare}{,,spojrzenie na 1000 jardów''} --- tym objawia się nagły kontakt ze zbyt dużą ilością bodźców i ludzi…

\img{./photos/x-s-2014-08-17_16-12-32__421.jpg}{is_this_the_end}{A więc… to już koniec?!}

%TODO dialog
Tak sobie siedzimy na murku, gdy nagle zauważamy trzech starszych panów kręcących się koło nas i co chwila rzucających nerwowe spojrzenia w naszą stronę --- widać naradzają się, ale co takiego knują? Wreszcie jeden z nich podchodzi i pyta: Where are you from? I don’t see any country for this flag! Odpowiadamy, że ,,from Poland'', a oni na to: ,,Ale to powinna być flaga niebiesko-biało-czerwona, prawda?'' Jego kompan protestuje --- kolory na odwrót! (Pewnie zrozumieli, standardowo, ,,Holland'' zamiast ,,Poland''.) To gdzie teraz? --- ,,Do domu, do Berlina, do Wa-wy…'' --- Aha, a ile kilometrów przejechaliście? --- 2000 --- z Warszawy?! Ogólnie to straszne tępaki, aż zachodziliśmy w głowę gdzie takie się rodzą, lecz na do widzenia sami się przyznali: ,,Greetings from Canada!''

Po zakupach w butikach zajrzeliśmy na moment do katedry, potem na drugie śniadanie do Bonusa i… znów ruszamy w trasę.

W połowie drogi do Ásbrú rozegrał się istny dramat. Zatrzymaliśmy się na postój, oparłem rower o stojącą na poboczu latarnię, a tu nagle --- potężny huk! Ewidentnie coś eksplodowało i to eksplodowało w moim rowerze! Podejrzanych nie było wielu --- to znów feralna dętka w przednim kole. Najwyraźniej niemiecka łatka przetarła się (w końcu przejechała już blisko 200 km!) i zalegający pobocze ostry żwirek przeciął dętkę. Mniejsza o przyczyny, najgorsze że tego nie ma jak normalnie załatać! Zrezygnowany zaczynam iść dalej pieszo, lecz to bez sensu. Próbuję więc jednak coś zaradzić i zaklejam dziurę kawałkiem innej dętki, a całość dodatkowo owijam dla wzmocnienia taśmą izolacyjną --- znów klapa, powietrze uszło po stu metrach. Tym razem podkopało to już moje morale na tyle, że na dobre zaczynam pieszą wędrówkę w stronę Ásbrú. Z resztą jestem umówiony tak, że zostawią mi coś do jedzenia, a ja im dam rzeczy obiadowe. Po 1,5 km poddaję się --- co innego prowadzić rower po mieście, a co innego wyładowany do granic, ważący ze 30 kilo grzmot! Kombinuję zatem, co począć. Wygrzebuję odrobinę nieszczelną dętkę (ostatnią z zapasu), kroję łatkę do opon na mniejsze kawałki i podejmuję ostatnią rozpaczliwą próbę naprawy opony. Tym razem próba ta zakończyła się (chyba) sukcesem i w miarę sprawnie docieram w okolice Keflavík, gdzie spotykam (a właściwie --- doganiam) resztę grupy.

Podsumowując, dziś jazda dała nam się solidnie we znaki. Nie chodzi już o awarie, tylko bardziej o ruch ,,jak na Alejach'', wiatr w twarz, permanentny podjazd (na szczęście niewiele nachylony), a przede wszystkim --- o uczucie ,,już wszystko się skończyło… co my tu jeszcze robimy?!''

Znów gości nas Artur, co stanowi dla nas naprawdę duże ułatwienie w logistyce. Wieczorem robimy małe pogaduchy z nim i jego żoną-Iranką. Główną osią były wspomnienia z objazdu, lecz momentami rozmowa schodziła na inne tory --- na przykład na religię. Najbardziej rozbroiła mnie ta oto kwestia, wygłoszona przez żonę Artura: \emph{,,Most of the people in Europe say they believe in some Jesus, but hopefully I’ve met some who said they believe in God''.}

\img{./photos/x-s-2014-08-17_23-52-35__112.jpg}{exhaustion}{Wieczór. Nocleg. Ciepło i miękko.}
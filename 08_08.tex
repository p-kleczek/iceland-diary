\chapter*{08.08.}

Od rana pada. A może tylko mży? Albo kropi? Ostatatnimi czasy dyskusja o poprawnej terminologii w odniesieniu do stopnia nasilenia opadów atmosferycznych pochłania naz bez reszty --- lansowane przez większość uszeregowanie wg stopnia intensywności to mży < kropi < pada. Nasze spory wynikały z niechęci do sięgnięcia po \href{http://sjp.pwn.pl/}{SJP PWN} --- wtedy wiedzielibyśmy, że “mżyć” to “o deszczu: padać gęsto drobniutkimi kropelkami”, “kropić” to “o deszczu: padać [po prostu] drobnymi kroplami”, a “padać” oznacza zwyczajnie “o zjawiskach atmosferycznych: spadać na ziemię w postaci wody, śniegu lub lodu”. Tak czy inaczej opad wody ma --- z różną intensywnością --- trwać przez cały dzień.

Pierwszym punktem programu na dziś jest sprzedaż biletów na oglądanie wielorybów. Put i Kasia dokładają starań, by wcisnąć je napotkanym w centrum Húsavíku turystom, lecz idzie do opornie. Może to kwestia różnic kulturowych? Weźmy na przykład takiego Hiszpana --- już chce kupić, ale jeszcze coś nie daje mu spokoju. Idzie więc do kasy i pyta się “Przepraszam, czy to będzie w porządku, jeśli kupię niewykorzystane bilety od tej dwójki, co stoi przed biurem?”. No i co mu ma kasjerka odpowiedzieć?! “Tak, oczywiście --- to nasi sprawdzeni partnerzy handlowi.”? Dobrze, że wkrótce przyplątali się bracia Słowacy, którzy doskonale orientowali się w istocie trasakcji. Po krótkim targu (poprosili o zniżkę 1000 kr) stali się szczęśliwymi posiadaczami naszych biletów.

\hint{Bilety na oglądanie wielorybów zostają “skasowane” w systemie dopiero wtedy, gry dany rejs faktycznie zobaczy wieloryby. Teoretycznie można próbować oglądać do skutku…}

Jeszcze w Húsavíku rozdzieliliśmy się, gdyż moją idée fixe była chęć odwiedzenia \href{http://www.visithusavik.com/attractions/the-turf-house-museum/}{muzeum w Grenjaðarstaður} --- od dwóch tygodni głowiłem się: jak właściwie ludzie funkcjonowali dawniej (przed erą elektryczności i geotermii) w islandzkim klimacie? Inni nie podzielali tej fascynacji (bądź też pęd do wiedzy przegrał z deszczem), stąd eskapada była samotna. W Grenjaðarstaður moje serce od razu podbił kustusz muzeum, który widać że był kustoszem z powołania (rzucił garścią ciekawostek, wyczerpująco odpowiadał na różne moje pytania, pozwolił się wysuszyć…). Sama ekspozycja niby typowa --- np. stoją sobie rzędem meble albo leżą na stole jedne obok drugich narzędzia stolarskie --- lecz jej sercem były porządnie zrobione książeczki-przewodniki, w których wytłumaczono przeznaczenie poszczególnych sprzętów (i to używając poprawnej angielszczyzny!). Po zwiedzeniu tego domu-muzeum mogę powiedzieć krótko: życie Islandczyków diametralnie różniło się od tego, co znamy z Europy Środkowo-Wschodniej (i ogólnie Europy kontynentalnej), dlatego gorąco polecam każdemu zawitanie do Grenjaðarstaður i za marne 600 kr poznanie odrobiny islandzkiej historii. PS. Dobrej jakości herbata w cenie biletu!

Kawałek za skrzyżowaniem \road{87} z \road{853} kończy się asfalt, a oprócz znaku “malbik endar” dodatkowo stoi “Blindhæð” i tabliczka 16 km. To tak żeby w zawoalowany sposób powiedzieć rowerzyście: “Słuchaj, przez najbliższe 16 km czeka cię jeżdżenie góra-dół --- wjazd na każdy możliwy pagórek!” Pieprzony rollercoaster… Pokonywanie tego w deszczu --- masakra.

\img{./photos/x-s-2014-08-08_15-56-24__85.jpg}{rollercoaster}{Deszcz, dziury, pod górę. Swojsko.}

Trochę zabalowałem w muzeum i potem całą dalszą drogę gnałem na złamanie karku --- na tyle, na ile pozwalała marna jakość nawierzchni i fatalna pogoda. Z resztą grupy zrównałem się dosłownie na przedmieściach Reykjahliðu. Pierwszym miejscem, do którego skierowaliśmy się po przybyciu do miasteczka był sklep spożywczy. Tam podjęliśmy (damsko-)męską decyzję: “Pierdolimy, dziś nie zwiedzamy, nie jedziemy dalej --- idziemy na najbliższy kemping z kuchnią i common roomem i tam czekamy jutra. Ewentualnie pomoczymy jeszcze tyłki w basenie, ale tylko jeśli ma gorącą wodę!

Pierwszy kemping, na który się udaliśmy, to ten naprzeciw Samkaupa  (tu rozbili się Holendrzy ;) Nie wzbudził naszej sympatii --- kuchnio-namiot (= zimno!) i 1500 kr/os. (ok, prysznic w cenie, lecz to wciąż drogo) --- więc szukamy dalej. Ostatecznie trafiliśmy nie lepiej --- na kemping Hlíð. Tu za kuchnio-common room służy konstrukcja wykonana z plastiku falistego rozpostartego na drewnianym stelażu, przykryta dachem od namiotu piwnego --- całość ustawiona jest po prostu na asfaltowym parkingu. Środkiem tej “kuchni” płynie rzeka, a pod “sufitem” gwiżdże wiatr. Nie mamy już jednak ani siły ani motywacji, by jechać gdziekolwiek indziej --- ociekamy wodą, nasz ubiór można wykręcać (no, może z wyjątkiem Puta;). Aha, na kempingu brak suszarni (tzn. są suszarki bębnowe, ale to nie dla nas), żadnych grzejników, a suszarki do rąk nie działają…

Na obiad Kasia zaszalała i przyrządziła spaghetti z twarożkiem i szpinakiem --- świetna odmiana od konserw mięsnych / tunfisków z makaronem/ryżem i warzywami z puszki! Ten lekkostrwany, zdrowy posiłek kończymy dopychaniem się solidną porcją chleba z szynką oraz dżemem. Dopiero pojedzeni w ten sposób --- zabieramy się za rozbijanie namiotów.

% {pogoda taka, że przypomniał mi się serial “Pacyfik” (który odcinek? + link :> | 1x4, 14:40 albo 22:35 albo 28:40-29:30)}

Podczas obiadu padła z ust Puta ważka deklaracja: “Jeśli za rok powiem, że chcę jechać rowerem gdzieś na północ od Krakowa… nie, od Warszawy… albo nie --- po prostu na północ od Polski, to przypomnijcie mi, że jednak nie chcę!” Wtórowała mu Kasia: “Islandia? Chętnie jeszcze raz, ale na pewno nie rowerem!” Oczywiście zdanie moje i Karoliny na te kwestie było podobne.

\img{./photos/x-s-2014-08-08_22-33-14__86.jpg}{tent_party}{Miało być wyjście na basen, wyszło --- piżama party.}

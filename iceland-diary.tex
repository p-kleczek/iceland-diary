% http://en.wikibooks.org/wiki/LaTeX/Floats,_Figures_and_Captions
% for file in *.png; do convert $file -rotate 90 rotated-$file; done

\documentclass[12pt,a4paper,leqno]{book}

\usepackage{polski}
\usepackage[utf8]{inputenc}
\usepackage[T1]{fontenc}
\usepackage[polish]{babel}

%\usepackage{amsmath}
\usepackage{amsfonts}
%\usepackage{amssymb}
\usepackage{graphicx}
%\usepackage[tight,footnotesize]{subfigure}
\usepackage{caption}
\usepackage{subcaption}
\usepackage{hyperref}

\usepackage{indentfirst}
%\setlength{\parindent}{6pt}

\usepackage{textcomp}


%% TODO: symbol żarówki
\newcommand{\hint}[1] {
	\vspace{6pt}
	[PORADA] #1
	\vspace{6pt}
}


%% TODO: symbol znaku zapytania
\newcommand{\curiosity}[1] {
	\vspace{6pt}
	[CIEKAWOSTKA] #1
	\vspace{6pt}
}

\newcommand{\img}[3] {
	\begin{figure}[h]
		\centering
		\includegraphics[width=.9\linewidth]{#1}
		\caption*{#3}
		\label{img:#2}
	\end{figure}%
}

\newcommand{\road}[1] {\fbox{#1}}

\DeclareUnicodeCharacter{B0}{\textdegree}

%% TODO (dla wszystkich plików):
% - zamienić trzy kropki na trzykropki
% - dodać * do \chapter i \section

%% for %i in (*.txt) do sed -i "s/1 setlinewidth/10 setlinewidth/g" %i

\begin{document}

	Dramatis Personae

Karolina,
Kasia (Cybulka)
Tomek (Put)
Paweł (Kłeczi)

Zdjęcia:
- Karolina
- Tomek

	\chapter*{22.07.}

\section*{Ostatnie sprawunki}

\indent Przed wyruszeniem we właściwą trasę musieliśmy jeszcze załatwić kilka sprawunków w Keflavíku. Mam tu na myśli wszelkie rzeczy, których z różnych względów ,,formalno-prawnych'' po prostu nie dało się załatwić w Polsce.
Najpierw zawitaliśmy na stację benzynową Olís \footnote{\href{https://www.google.com/url?q=https\%3A\%2F\%2Fmaps.google.com\%2Fmaps\%3Fq\%3D63.979816\%2C-22.54672}{mapa z zaznaczoną stacją benzynową}}, by zakupić naboje do palników gazowych.

W sumie nie było takiej konieczności --- równie dobrze mogliśmy skorzystać z darmowych butli, które turyści już odlatujący z wyspy masowo porzucają na kempingach położonych w promieniu ok. 50 km od stołecznego lotniska. Podobno na kempingu w Garður --- tuż obok lotniska --- leżą całe hałdy tych naboi i to np. w 2/3 pełne.
Następnie zahaczyliśmy o warsztat wulkanizacyjny \footnote{\href{https://www.google.com/url?q=https\%3A\%2F\%2Fmaps.google.com\%2Fmaps\%3Fq\%3D63.982619\%2C-22.546328}{mapa z zaznaczonym zakładem wulkanizacyjnym}}, gdzie sprawnie dobiliśmy opony do ponad 4 atmosfer.

\hint{Naboje gazowe można próbować ,,upolować'' na kempingach znajdujących się w pobliżu lotniska (czyli w promieniu około 50 km, np. w Garður albo w Grindavíku). Zdarzają się nawet naboje w 2/3 pełne!}

\hint{Do jazdy po asfalcie lepiej mieć opony naprawdę twarde, gdyż dzięki temu znacznie zmniejszają się opory toczenia. Mówiąc po ludzku --- zajedziesz dalej, szybciej, mniej się męcząc :)}

Kolejnym punktem programu był supermarket sieci Bonus --- bodaj najtańszej na Islandii. Nie będę się tu rozpisywał za bardzo o tym, w jaki zachwyt wprawiły nas ceny produktów i ich wybór, gdyż \href{http://www.roboppy.net/food/2009/04/iceland-day-1-part-ii-reykjavik-bonus-supermarket-skyr.html}{to zrobili już inni przed nami}. Wspomnę tylko o jednym naszym odkryciu --- chodzi o przecier ananasowy marki EuroShopper. Trzypak puszek, każda po 227 g (z czego 70\% to ananas, a reszta --- sok ananasowy, a nie syrop) kosztował… 40 kr! Czyli 60 kr --- jakieś 1,50 zł --- za kilogram. Aż żal nie kupić! Te ananasy doskonale pełniły rolę deseru --- oczywiście gdy tylko Bonus był pod ręką…

\img{./photos/x-s-2014-07-22_14-53-31__2.jpg}{keflavik_metal_guys}{Fantazja Islandczyków --- uliczni tancerze…}

%\begin{figure}[h]
%\centering
%\begin{subfigure}{.5\textwidth}
%  \centering
%  \includegraphics[width=.9\textwidth]{./photos/2014-07-22_14-53-31__2.jpg}
%  \caption{A subfigure}
%  \label{fig:keflavik_metal_guys}
%\end{subfigure}%
%\begin{subfigure}{.5\textwidth}
%  \centering
%  \includegraphics[width=.9\textwidth]{./photos/2014-07-22_14-54-31__3.jpg}
%  \caption{A subfigure}
%  \label{fig:keflavik_breakfast}
%\end{subfigure}
%\end{figure}

Posileni, pełni sił, ruszyliśmy do centrum Keflavíku, do oddziału Landsbanki \footnote{\href{https://www.google.com/url?q=https\%3A\%2F\%2Fmaps.google.com\%2Fmaps\%3Fq\%3D63.995522\%2C-22.548067}{mapa z zaznaczonym oddziałem Landsbanki}} --- takiego tutejszego PKO --- by wymienić trochę waluty. Niby na lotnisku był bankomat, ale niektóre karty średnio z nim współpracowały, a do tego dziewczyny chciały wymienić trochę euro w gotówce na korony…

\hint{Na Islandii naprawdę wszędzie można płacić kartą. Nawet na kempingach w środku interioru! Jedyne, co warto mieć ze sobą, to żelazną rezerwę monet o nominale 100 kr --- na wielu kempingach zainstalowane są ,,dozowniki ciepłej wody'', które przyjmują 5x 100 kr i w zamian umożliwiają wzięcie 5-minutowego ciepłego prysznica. Czasem da się rozmienić banknoty na monety na kempingu, lecz w myśl zasady ,,przezorny zawsze ubezpieczony'' znacznie lepiej zrobić to zawczasu w jakimś sklepie czy na stacji benzynowej.}

Ostatnim puntem był zakup czegoś, co umożliwi nam dzwonienie z islandzkiego numeru komórkowego oraz korzystanie z internetu --- wybór padł na Vodafone (biuro tej firmy mieściło się koło polskiego marketu), który oferował 3 GB danych za około 50 zł. O ile bez problemu kupiliśmy prepaid, o tyle aktywacja internetu wymagała już więcej zachodu, bo nie dość, że operacja odbywała się poprzez infolinię Vodafone, to jeszcze niezbędne było posiadanie karty kredytowej (szczęśliwie mieliśmy w odwodzie w Polsce posiadacza takowej). A internet miał nam służyć nie tylko do zabawy facebookiem, ale o tym później…

\hint{Zawczasu, przed wyjazdem z Polski, zorientuj się kto z rodziny, przyjaciół lub znajomych posiada kartę kredytową i będzie skłonny podać ci jej numer, datę ważności oraz kod CVV. Dodatkowo zapisz imię i nazwisko posiadacza karty w takim brzmieniu, jak na karcie (czyli z uwzględnieniem polskich liter lub ich braku). Istnieje pewna grupa usług --- np. aktywacja pakietu internetowego, zakup niektórych biletów online --- których nie da się załatwić przy użyciu zwykłej karty debetowej!}

Keflavík opuściliśmy ostatecznie o 15:00, przy (wciąż jeszcze) słonecznej pogodzie.

\hint{Islandia to olbrzymi obszar i ciężko zawczasu dowiedzieć się o każdej atrakcji i o każdym interesującym miejscu na trasie. Warto więc korzystać intensywnie z informacji turystycznych, zbierać (i przeglądać!) foldery reklamowe oraz pytać innych turystów (najlepiej też rowerzystów, bo ci będą w stanie np. przestrzec cię przed trudnościami terenowymi). W Keflavíku znajduje się Centrum Informacji Turystycznej --- lokalizację znajdziesz na ich profilu na Google+ (\url{https://plus.google.com/111960041675886065623/about?gl=pl\&hl=en}).}

\section*{Pierwsze godziny w trasie}

Pierwsze kilometry i już mocne zderzenie z islandzką naturą --- jedziemy przez pustkowia. Po lewej, po prawej, jak okiem sięgnąć lawa i niewielkie skałki. Nic dziwnego, że obszar ten --- jako jeden z trzech na Islandii --- służył NASA do prowadzenia treningów dla astronautów przed misją Apollo.

\img{./photos/x-s-2014-07-22_18-05-39__5.jpg}{first_hours_on_iceland}{Pustkowia półwyspu Reykjanes}

\img{./photos/x-s-2014-07-22_18-25-14__8.jpg}{neptune_monument}{Pomnik planety Neptun}

Kawałek przed Hafnir przyplątał się do naszej grupy pies --- ochrzczony imieniem Posejdon (od \href{https://www.facebook.com/120832791270880/photos/a.612815058739315.1073741825.120832791270880/612815132072641/?type=3&theater}{pomnika planety Neptun}, który stał w miejscu zdarzenia). Pies ten biegł za nami aż do \href{http://www.visitreykjanes.is/searchresults/attraction/bridge-between-continents}{Kładki Między Kontynentami}. Nie byłoby w tym nic aż tak specjalnego gdyby nie to, że nasz nowy towarzysz urzekł nas swym polowaniem na samochody: gdy zauważył nadjeżdżający pojazd, zaczajał się na poboczu i potem w ostatniej chwili wypadał na jezdnię --- tuż przed maskę --- głośno ujadając. Albo niesamowicie odważny albo niesamowicie głupi ;-)

Już daje nam się we znaki wiatr oraz liczne (na szczęście krótkie) stromsze podjazdy. Walka z takim kombo jest szczególnie ciężka dla tych osób, które nie jeździły ostatnio za wiele.

\hint{Podczas jazdy na rowerze pracują nieco inne mięśnie niż np. podczas wycieczek górskich. Bieganie, pływanie, orbitrek --- wszystko to oczywiście zwiększa wydolność organizmu, lecz nic nie zastąpi odpowiedniej liczby przejechanych kilometrów! :)}

Słońce było tylko na zachętę --- szybko schowało się za chmury i zrobiła się typowa islandzka aura --- z mniej lub bardziej intensywnymi opadami deszczu… A trzeba wiedzieć, że tu nie ma się gdzie schować --- żadnych przydrożnych sklepów, barów, nawet wiat PKS-u. Tak więc, siłą rzeczy, mimo deszczu jedziemy dalej. Zresztą… nie lało tylko kropiło, czyli nawet nie byłoby sensu przeczekiwać tego --- klimat tu bowiem taki, że można by się nie doczekać momentu ,,wypadania''.
%TODO: "tego" - przydało by się coś dodać, żeby to zdanie lepiej brzmiało

\section*{Grindavík}

Gdy byliśmy już na obrzeżach Grindavíku powstał problem --- szukać miejsca na nocleg już teraz, czy może jechać jeszcze kawałek dalej? Cała ekipa była dość jednomyślna: \emph{Nie, nie jedziemy już dalej --- jest za późno (była 19:00 --- red.) no i warto byłoby się nie zarżnąć od razu pierwszego dnia. Lepiej wyruszyć wcześniej nazajutrz, ale przynajmniej dziś odespać podróż!} Tu konieczne jest słowo komentarza --- berlińczycy nie spali ani w Polskim Busie, ani koczując na lotnisku Schönefeld, ani też w samolocie. Zatem byli non-stop na chodzie przez 30 godzin, a ze względu na późny przylot i długie czekanie na bagaż noc w Ásbrú też była zarwana… Tak więc po przejechaniu 50 km rozbiliśmy się na kempingu w Grindavíku.
%TODO: wytłumaczyć, o co chodzi z "Berlińczykami" (tj. reszta grupy, bo lecieli przez Berlin)

Kemping można zaliczyć do klasy ,,all inclusive'', bo posiada ,,świetlicę'' (ogrzewanie podłogowe!) z w pełni wyposażonym aneksem kuchennym (sztućce, garnki, płyty indukcyjne, toster…), a prysznice są w cenie. W kuchni cała jedna szafka była zajęta przez ,,free stuff'' --- nie tylko rzeczy spożywcze, ale też gospodarcze (np. są wspomniane naboje gazowe).

\img{./photos/x-s-2014-07-23_11-11-19__6.jpg}{grindavik_cantine}{Jadalnia kempingu w Grindavíku}

\hint{Będąc w kuchni na kempingu warto przeglądnąć cały asortyment opatrzony napisem ,,free food'' i ,,free stuff'', gdyż często można tam natrafić na towary luksusowe --- porządną herbatę, konserwy, słodycze, płatki śniadaniowe… Należy jednak zwracać uwagę na termin przydatności do spożycia oraz organoleptycznie zbadać faktyczny stan produktu (np. otwartego mleka lepiej nie tykać ;-)}

Na obiadokolację z radością pochłaniamy spaghetti z mięsem ze słoika (polskie, domowe, zawekowane mięso --- mniam!), sosem i warzywami z puszki. Przekąszamy ciasteczkami i niezwłocznie udajemy się na spoczynek --- większość z nas nawet bez mycia się…

\hint{Kupując warzywa i sosy w puszcze zwróć uwagę na ,,gęstość'' produktu, tj. stosunek wagi netto (po odcieku) do wagi brutto --- bo po co wozić ze sobą puszkowaną wodę?! Na Islandii bezkonkurencyjna pod tym względem okazała się mieszanka warzyw Euroshopper, w której wspomniany współczynnik wynosił około 0,95.}

Ponieważ mamy internet i założone wydarzenie na fb (co by nie musieć powtarzać po parę razy tego samego --- gdzie jesteśmy i co robimy --- a to rodzinie, a to wszystkim znajomym), więc co jakiś czas Put informuje nas \emph{,,O! Mamy 5 lajków i 3 komentarze!''}. Hm.. może należało założyć profil fb? Wtedy moglibyśmy jeszcze korzystać ze statystyk…

W ogóle (w miarę) swobodny dostęp do internet wprowadza też nową jakość wieczorami, przed spaniem. W pewnym momencie rozlega się dramatyczne wołanie \emph{,,Pucie, zrobisz modem?''}, a chwilę później \emph{,,Pucie, przełożysz komórkę bliżej naszego namiotu? Bo coś słaby zasięg…''}

\vspace{16pt}

Aha, a na zakończenie --- cytat dnia:
\epigraph{,,Czuję się, jakbym była na Islandii od tygodnia…''}{--- \textup{Karolina}}


	\chapter*{23.07.}

Poranek przebiega spokojnie, gdyż każde z nas pragnie maksymalnie wykorzystać możliwości kempingu -- są więc tosty i herbata na śniadanie, potem długi ciepły prysznic… W trasę ruszamy o 10:30, akurat chwilę po tym gdy zaczął wiać silny wiatr ze wschodu. Jego podmuchy sprawiają, że nasza średnia prędkość spada do około 12 km/h --- no, ale przynajmniej nie pada! Ta pogoda pozwoliła nam odkryć nowe zastosowanie dla dużej polskiej flagi, którą wożę na maszcie przymocowanym do bagażnika --- pełni ona rolę wiatrowskazu i umożliwia ustawienie się w szyku.

\hint{Jazda w szyku to niezwykle istotny element wszelkich poważniejszych wypraw rowerowych. Pozwala oszczędzać siły, a w sytuacjach ekstremalnych --- szczególnie gdy wieje --- w ogóle pozwala zajechać w planowane miejsce! Szerokość szyku należy dostosować do warunków (żeby np. za nami nie tworzył się korek), warto jednak zajmować tyle pasa (i jechać na tyle daleko od jego krawędzi), by kierowcy nie wyprzedzali ,,na gazetę''. \newline  Więcej o tym zagadnieniu poczytasz tu: \url{http://bikestory.pl/jazda-rowerem-grupie/} oraz \url{http://www.agk-pruszkow.cba.pl/strona01.html}. }

Co jakiś czas napotykamy na zagubione przy drodze drogowskazy z jakimiś napisanym po islandzku i osobliwym symbolem --- herbem św. Jana (obecnie głównie kojarzonym z klawiaturami spod znaku Apple). Wtedy to zatrzymujemy się, Put wyciąga komórkę, włącza internet i sprawdza w Google Images jaką to atrakcje właśnie mijamy. Jeśli ze zdjęć wynika, że ma potencjał --- zbaczamy z trasy, jeśli nie --- odpuszczamy. Właśnie w ten sposób trafiliśmy na piękne klify Krýsuvíkbjarg, które oddalone są od \road{427} o parę kilometrów i nie sposób ich dostrzec z szosy.

\img{./photos/x-s-2014-07-23_15-03-31__12.jpg}{cliff_ford}{Czy te brody to już?}
\img{./photos/x-s-2014-07-23_15-44-30__18.jpg}{cliff_rock}{Młody człowiek i morze}

Kawałek dalej ponownie nadłożyliśmy kilometrów, by zahaczyć o gorące źródło Austurengjahver. Dotarcie do samego źródła wymaga trochę wysiłku --- \href{http://www.openstreetmap.org/way/33182596}{długi na 1,6 km spacer} skutecznie odstręcza większość ,,turystów'', więc na miejscu można w spokoju pomoczyć nogi w lekko błotnistej (lecz ciepłej i śmierdzącej jak każde inne gorące źródło) wodzie. Zdania ,,czy iść'' były w naszej ekipie mocno podzielone i niewiele brakowało, a też byśmy sobie podarowali ów spacer. Ostatecznie jednak, w demokratycznym głosowaniu (2 za, 1 przeciw, 1 wstrzymał się) projekt został ,,klepnięty''. Potem podjechaliśmy jeszcze pod Seltún, obszar geotermalny z ładnymi kolorowymi skałami.

\img{./photos/x-s-2014-07-23_18-26-49__25.jpg}{austurengjahver_legs}{Nie ma to jak okład z ciepłego błotka!}
\img{./photos/x-s-2014-07-23_18-29-35__27.jpg}{austurengjahver_panorama}{Swoiste OP-1 Puta nie chroniło go przed smrodem…}

Nocleg wypadł nam na kempingu w osadzie Strandarkirkja, około 14 km przed Þorlakshöfn. Ach, jakaż była nasza radość, gdy zobaczyliśmy znak \emph{FREE camping}, a po dojeździe na miejsce oczom naszym ukazała się olbrzymia łąka z soczyście zieloną trawą, solidne sanitariaty (z grzejnikiem w środku) oraz ,,wiatrołapem'' pełniącym rolę kuchni polowej… Prysznic był płatny (500 kr), lecz ciepła woda z umywalki w zupełności wystarczała do zachowania minimum higieny.

Krajobrazy dnia dzisiejszego: po prawej pastwiska i równina, po lewej masywne góry o niemal pionowych zboczach.

\img{./photos/x-s-2014-07-23_22-36-27__12.jpg}{strandarkirkja}{Kemping w Strandarkirkja}

	\chapter*{24.07.}

Po krótkim, lecz intensywnym porannym deszczu przed Þorlakshöfn kolejne godziny były niemal bezwietrznie (bądź wręcz wiało w plecy!) i świeciło piękne słońce (!). Tak… byliśmy tym faktem tak zaskoczeni, że nie przygotowaliśmy zawczasu kremu z filtrem UV --- był na dnie sakwy --- i w efekcie co poniektórzy mieli z tego dnia pamiątki w postaci bąbli na uszach i łuszczącej się skóry na nosie…

\img{./photos/x-s-2014-07-24_14-38-32__16.jpg}{fodder_thorlakshofn}{Pasza w Þorlakshöfn}

Pierwsze co zrobiliśmy po przybyciu do Hveragerði --- kolejnego dużego miasteczka na naszej dzisiejszej trasie --- to zajechaliśmy w okolicę Bonusa. Szybka sonda wykazała jednak, że nikt jeszcze nie jest głodny! Dominował pogląd \emph{,,A, zjemy obiad jak będziemy wracać po kąpieli w gorących źródłach!''}

Zajechaliśmy więc do Rjúpnabrekkur, gdzie zostawiliśmy rowery, i dalej --- już pieszo --- udaliśmy się górską ścieżką w stronę \href{http://www.vulkaner.no/t/isl2004/hot.html}{Klambragil}. Celem było znalezienie ,,niebieskiego jeziorka'', w którym można by się pokąpać --- informacje o jego istnieniu Karolina otrzymała od swojego znajomego. Cóż… poszukiwania zakończyły się fiaskiem i koniec końców moczyliśmy się w ,,zwykłej'' rzeczce płynącej środkiem doliny (ok, też była bardzo ciepła, a momentami --- gorąca).

\img{./photos/x-s-2014-07-24_17-17-04__18.jpg}{klambragil_on_the_way}{W drodze do wód…}
\img{./photos/x-s-2014-07-24_19-24-40__23.jpg}{klambragil_stream}{Moczymy tyłki w Klambragil}

Gdy wróciliśmy pod Bonus, spotkała nas niemiła niespodzianka --- sklep był już zamknięty. Zamknęli go równo 15 minut przed naszym przyjazdem, o 18:30. Tak więc póki co nici z obiadu, bo średnio nam się uśmiechało kupować --- po czarnorynkowych cenach --- jedzenie na znajdującej się vis-a-vis Bonusa stacji benzynowej. Chcąc nie chcąc, bez zbędnego guzdrania się, pojechaliśmy do Selfossu --- miasta oddalonego o ledwie 12 km. Tam postanowiliśmy zanocować i poszukać jakiegoś innego marketu.

Kemping w Selfossie ma wszystko to, co każdy szanujący się kemping mieć powinien: przestronną, ciepłą świetlicą z aneksem kuchennym, specjalną trawiastą polanę (niemal tuż przy budynku gospodarczym) wyłącznie dla turystów z namiotami oraz ciepły prysznic w cenie. Aha, no i niedaleko od kempingu jest Samkaup czynny w dni powszednie od 7:30 do 23:30! Udało nam się więc zrobić zakupy na śniadanie --- jutro nie będziemy musieli zaczynać dnia od spacerów wśród półek sklepowych.

Jednym z milszych akcentów dnia dzisiejszego był widok całych stad koni islandzkich galopujących po ciągnących się wzdłóż drogi pastwiskach.
	\chapter*{25.07. --- W drodze do maskonurów}

W nocy kilkukrotnie budziło nas bębnienie deszczu o tropik namiotu --- tak silna była ulewa! Nad ranem przeszła, więc zarządziliśmy sprawne śniadanie i zbiórkę --- przy wciąż jeszcze w miarę stabilnej pogodzie. Niestety, nic co piękne nie trwa wiecznie\textellipsis W chwilę po założeniu sakw ,,na pakę'' zaczyna najpierw mżyć, potem kropić, a wreszcie --- lać. I tak właśnie --- w mniejszym lub (częściej) większym deszczu, przy wietrze ,,w mordę'' i przy dużym natężeniu ruchu na \road{1} --- upływa nam podróż do miejscowości Hella.

W Helli czeka na nas oaza --- budynek z supermarketem sieci Kjarval. Oprócz niego znajduje się tam jeszcze mała piekarnio-cukiernia i\textellipsis parę stolików dla odwiedzających! Anektujemy jeden z nich i urządzamy popijającym kawę w cukierni Islandczykom pokaz ,,Jak przygotować drugie śniadanie w warunkach turystycznych?'' --- wyciągamy chleb i~konserwy, ser żółty, nóż-kosę, ciastka\textellipsis Oczywiście kompletnie przemoczone, tak że nawet buty można wykręcać, stąd w niedługim czasie pod stołkami utworzyły się potężne kałuże\textellipsis Po chwili, ośmieleni naszym przykładem, pomysł pikniku podchwycili siedzący przy sąsiednim stoliku Francuzi --- też zaczęli robić kanapki!

\img{./photos/x-s-2014-07-25_17-15-16__25.jpg}{hella_heater}{Jest grzejnik --- jest impreza.}

Na jedzeniu i trzęsieniu się z zimna upływa nam pierwsze 1,5 godziny. Wtedy to poszedłem do znajdującej się piętro niżej toalety i zauważyłem wiszący na ścianie\textellipsis olbrzymi, gorący grzejnik! Radość nasza była przeogromna i kolejne 20 minut spędziliśmy rozwieszając na nim wszystko co tylko się dało, a następnie uprawiając swoistą gimnastykę, by znaleźć się jak najbliżej ciepła i wysuszyć mokre wdzianka rowerowe tudzież skarpetki na stopach. W sumie siedzielibyśmy dłużej (i wysuszyli nasze ubrania na pieprz!), lecz wygonił nas Put stwierdzeniem: \emph{Słuchajcie! Jak tak dalej pójdzie, to nie zdążymy na odpływający o 19:00 prom na Wyspy Zachodnich Ludzi!} Ech\textellipsis

Już mieliśmy ruszać spod sklepu, gdy naszą uwagę przykuła naklejona na jego witrynę mapa Islandii. Mapa jak mapa, ale po co na normalnej mapie ktoś miałby zaznaczać ostre podjazdy (>8\%) albo natężenie ruchu pojazdów? Okazało się, że natrafiliśmy na mapę \href{http://www.vegagerdin.is/media/upplysingar-og-utgafa/Cycling-map.pdf}{,,Cycling Iceland -- Summer 2014''}, stworzoną --- jak sama nazwa wskazuje --- z myślą o rowerzystach. Korzystaliśmy z niej intensywnie m.in. w kwestii pól namiotowych oraz lokalizacji sklepów ,,poza wsiami''.

\processifversion{PDF}{
\hint{Na stronie internetowej \emph{Vegagerdin} (taki ichniejszy Zarząd Dróg) możesz nie tylko pobrać aktualną mapę rowerową, ale także sprawdzić aktualne warunki pogodowe na podstawie pomiarów z przydrożnych stacji meteo. \newline Link: \url{http://www.vegagerdin.is/english}}
}

\processifversion{HTML}{
\hint{Na stronie \href{http://www.vegagerdin.is/english}{Vegagerdin} (taki ichniejszy Zarząd Dróg) możesz nie tylko pobrać aktualną mapę rowerową, ale także sprawdzić aktualne warunki pogodowe na podstawie pomiarów z przydrożnych stacji meteo.}
}

Odcinek Hella-Landeyjahöfn był (jeśli chodzi o deszcz) w miarę znośny, gdyż dolało nas jeszcze tylko dwa razy. Na dobrą sprawę upodabniamy się do Islandczyków, którym ,,wisi'' to, czy mży bądź pada oraz czy wieje --- nawet przy takiej pogodzie chodzą normalnie z~dziećmi (malutkimi) albo grają w piłkę nożną czy też\textellipsis golfa (!).

Bo w sumie --- co innego można robić, skoro tak wygląda większość dni? (Gdy historię z wyletnionym dzieckiem w wózku zabranym na spacer przy paskudnej pogodzie usłyszał znajomy mieszkający parę lat na Islandii, stwierdził że czasem w takich sytuacjach jacyś nadgorliwi mieszkańcy potrafią wezwać urząd ds. dzieci).

Tak więc skoro bardziej kropiło niż padało, Kasia założyła stuptuty dopiero wtedy, gdy woda zaczęła atakować od wnętrza buta jej skarpetkę\textellipsis Następnie testowała jeszcze patent z ubieraniem reklamówki na gołą stopę i dopiero potem skarpetki, lecz chyba bez spektakularnych sukcesów (pierwotnie chciała założyć worek na suchą skarpetkę, lecz znalezienie takowej okazało się przerastać naszą wolę działania w chwili suszenia się w Helli).

Przypomniała mi się więc historia z gotowaniem żaby: \emph{Gdy wrzucisz do wrzątku --- wyskoczy, lecz gdy wrzucisz do ciepłej wody i zaczniesz ją stopniowo podgrzewać --- ugotuje się}. I~podobnie jest z niezakładaniem stuptutów: gdy nagle się rozleje --- zakładamy momentalnie, ale gdy najpierw mży, a dopiero potem zaczyna padać coraz bardziej i bardziej, to na początku myślimy ,,a, zaraz przejdzie'', a gdy już mamy całe przemoczone buty, to nawet nie chce nam się wyjmować ochraniaczy\textellipsis

Z powodu wiatru i deszczu na prom zdążyliśmy, lecz dosłownie w ostatnich minutach. W punktualnym dotarciu nie pomagały też wskazy. Pierwszy, tuż po skręcie z \road{1}, podawał odległość 11 km. Drugi, ustawiony równo 11 km dalej, informował że ,,do przejechania pozostało jednak jeszcze 3 km''.

\pagebreak

\img{./photos/x-s-2014-07-25_21-22-31__45.jpg}{ferry_to_island}{Rejs na Wyspy Zachodnich Ludzi}

Rejs promem obfitował w piękne widoki: zarówno samotnych, niezamieszkałych wysepek przed Heimaey (największą wyspą), jak też klifów i skalnych ścian przy samym wejściu do portu. Podobnie byliśmy oczarowani \href{http://www.tjalda.is/en/herjolfsdalur/}{kempingiem}. Po pierwsze, jest on położony w~niecce u podnóża gór, które otaczają go z trzech stron, więc właściwie nie wieje. Po drugie, namioty rozbija się na pięknej trawie, pośród porośniętych mchem skałek. Po trzecie, kemping ten posiada salon z aneksem kuchennym, który --- jak zwykle --- jest schludny i~funkcjonalny (na wyposażeniu m.in. kuchenka mikrofalowa i grzejniki). Z początku ucieszyliśmy się, bo nie było nigdzie widać recepcji. Czyżby nocleg w tak pięknych okolicznościach przyrody miał się odbyć ,,za frajer''? Otóż nie. Po dłuższej chwili pojawiła się dziewczyna robiąca za poborcę podatków, która zainkasowała za ten luksus po 1300 kr od łebka. Ech\textellipsis

Pod wieczór znów wszyscy byli ,,wycięci'' po intensywnym, pełnym wrażeń dniu --- Put pisał list do domu, Kasia biła rekordy w 2048, Karolina starała się zaplanować operację prania (wbrew pozorom, pod koniec dnia, to rzecz wymagająca!), a Paweł\textellipsis Paweł spisywał wszystkie te małe i duże wydarzenia ,,ku pamięci potomnych'' :)

Nocą wybrałem się na mini-wycieczkę górską. Wystarczy podejść jakieś 20 minut, by z grani Eggjar móc podziwiać leżące u stóp miasto. Maskonury chyba poszły już spać. Szkoda.

\img{./photos/x-s-2014-07-25_22-44-06__52.jpg}{heimaey_camping}{Kemping --- śródgórska, malownicza oaza spokoju}

\img{./photos/haimaey_topo.jpg}{heimaey_topo}{Fragment mapy wyspy Haimaey z zaznaczonym kempingiem.}

	\chapter*{26.07. --- Wyspy Zachodnich Ludzi}

Rano deszcz i pochmurne niebo --- standard. Szczęśliwie koło 9:30 coś się poprawia, gdzieniegdzie przeziera błękit i promienie słońca. W oczekiwaniu na pełne wypogodzenie robimy pranie, reperujemy sprzęt…

\img{./photos/x-s-2014-07-26_14-00-40__54.jpg}{heimaey_sunshine}{Istna rzadkość na Islandii --- słońce!}

W południe ruszamy na południowy kraniec wyspy (Stórhöfði), by pooglądać maskonury (ang. \emph{puffins} --- nie mylić z \emph{muffins}). Zmęczyliśmy się odrobinę pokonując strome serpentyny na ostatnim odcinku, lecz wysiłek definitywnie się opłacił. Z klifów maskonury widać było jak na dłoni --- i to nie pojedyncze sztuki, a całą kolonię! Urządzamy więc swoiste safari fotograficzne, a potem --- jako że niebo zrobiło się bezchmurne --- leżymy plackiem wsłuchując się w szum fal…

Wracaliśmy już na kemping, gdy nagle Cebulka stwierdziła, że ,,chyba coś dziwnego dzieje się jej z bagażnikiem''. Faktycznie, wypadnięcie paru śrubek i skrzywienie jednego ze wsporników można określić mianem ,,czegoś dziwnego'' ;) Nie ma co ukrywać, nastroje zrobiły się minorowe --- bez sprawnego bagażnika nie ma co myśleć o kontynuowaniu wyprawy! Lecz wtedy przypomniałem sobie, jak to podczas jednej ze swoich wypraw ratowałem swój bagażnik… Pojechaliśmy więc do ,,wioski'' i  w pierwszym lepszym domu poprosiliśmy o młotek --- wystarczyło parę klepnięć i… bagażnik był jak nowy. Tak, to definitywnie jedna z zalet bagażników aluminiowych. Śrubki pożyczyliśmy z roweru Karoliny, która miała zamontowany tylko jeden kosz na bidon --- dwie sztuki śrubek były więc zbędne. Uff… możemy kontynuować podróż!

Ponieważ czasu do odpłynięcia promu było jeszcze sporo, wyprawiliśmy się na okoliczne szczyty celem oglądania jeszcze większej liczby maskonurów. Te ptaszki w ogóle się nas nie boją --- z początku skradamy się ostrożnie, zamierając po każdym kroku, by żadnego nie spłoszyć, lecz nawet po podejściu na odległość 1 metra one wciąż siedzą jak siedziały! Co nas zszokowało, to fakt że na kempingu widzieliśmy tylu ,,full-pro'' turystów --- w porządnych butach górskich, z kijkami i z plecakami --- a na tych górskich ścieżkach (raptem 20-30 min od kempingu) nie spotykamy ani jednego. Choć w sumie… tym lepiej dla nas!

\img{./photos/x-s-2014-07-26_17-50-56__60.jpg}{puffin_hunter}{Karolina i maskonur}

Postanowiliśmy opuścić wyspę ostatnim promem. Zajeżdżamy więc do kasy na jakieś 20 minut przed jego odpłynięciem, chcemy kupić bilety, a tu… lipa! Nie ma wolnych miejsc! Możemy co najwyżej zapisać się na listę oczekujących (i liczyć na to, że ktoś z osób z rezerwacją się nie pojawi)… Przeżywamy chwile grozy --- zwłaszcza, że nie jesteśmy pierwszymi na tejże liście oczekujących --- i zastanawiamy się już: Co to będzie? Co to będzie? No bo jak się nie uda zaokrętować, to czeka nas kolejny nocleg na wyspie… W końcu jednak miejsce dla nas się znalazło --- hura!

\hint{Jeśli to możliwe --- zawczasu zarezerwuj online miejsce na promie. Przychodząc do terminalu nawet na 30 minut przed odpłynięciem promu może okazać się, że wszystkie zostały już wyprzedane (bądź właśnie zarezerwowane) wcześniej.}

\curiosity{Od spotkanego w drodze do portu polskiego inżyniera-stoczniowca (\emph{,,Znają mnie w każdej tawernie na Islandii i stawiają kawę czy herbatę!''}) dowiedzieliśmy się ciekawostki: Na Heimaey po erupcji z 1973 r. na nowopowstałe pola lawy zrzucano ze specjalnie przystosowanego samolotu łubin --- jedną z nielicznych roślin odpornych na islandzki klimat i tak surową ,,glebę''. Tenże łubin po latach miał dać zaczątek faktycznej glebie. Dziś podłoże jest już na tyle żyzne, że gdzieniegdzie rosną nawet niewielkie drzewa iglaste. \newline \processifversion{PDF}{Więcej o wysiłkach Islandczyków podejmowanych w celu rekultywacji (nie tylko Wysp Zachodnich Ludzi) można poczytać w dokumencie pod adresem  \url{http://www.land.is/english/images/pdf-documents/healing_the_landL.pdf}} \processifversion{HTML}{Więcej o wysiłkach Islandczyków podejmowanych w celu rekultywacji (nie tylko Wysp Zachodnich Ludzi) można poczytać \href{http://www.land.is/english/images/pdf-documents/healing_the_landL.pdf}{tutaj}}}.

\img{./photos/x-s-2014-07-26_23-05-25__115.jpg}{ferry_to_mainland}{Heimaey w promieniach zachodzącego słońca}

Wieczorem podjechaliśmy jeszcze wspólnie pod Seljalandsfoss, a potem rozdzieliliśmy się --- Put, Kasia i Karolina pojechali na nocleg, a ja uparłem się, że chcę wybrać się na (nocną) wycieczkę ze Skógar do Þorsmörk przez przełęcz Fimmvörðuháls.
	\chapter{27.07.}

\section{Wypad do Þorsmörk}

%% TODO: zdjęcia

Cała ta eskapada odbywała się na wariackich papierach - zaniżone odległości, optymistyczne szacowanie czasu przejazdów, wyjście z założenia “wszystko przebiegnie po mojej myśli”... Pierwsze zaskoczenie przyszło, gdy po 20 km wciąż nie widziałem żadnego skrętu na Skógar. Po 25 km - już mocno zaniepokojony - pytam o drogę w jakiejś “chacie”. Mieszkająca tam dziewczyna łamaną angielszczyzną instruuje mnie, że muszę pojechać jeszcze parę kilometrów dalej (czyli droga wydłużyła się o 10 km). Dodaje też, że wraz z siostrą też tam jutro idą - może się spotkamy.

Jakoś - już po nocy - dotarłem do Skógar, lecz tam znów zbłądziłem. Zrobiłem więc jakiemuś losowemu Islandczylowi “backdoor” na werandę - i otrzymałem kolejną porcję wskazówek: wróć się, pojedź za stare obory, tam będzie furtka i żwirowa droga biegnąca ostro w górę - jedź za nią.

Sam początek drogi był tragiczny: taie kamerdolce, że nawet pchać rower było ciężko! Później było już trochę lepiej, lecz wciąż na przemian - trochę jazdy 6 km/h (bo drogę zalegają kamienie wielkości pięści) - i trochę pchania. Szczęśliwie niby środek nocy (1:00), ale słońce do końca nie zaszło, więc od biedy nie muszę nawet korzystać z czołówki - z pewnym wysiłkiem, ale dostrzegam wszystkie przeszkody leżące na drodze.

W połowie drogi do schroniska Baldvinsskáli spotkała mnie niemiła niespodzianka - dalszą drogę przegradzała lodowcowa rzeka (niby był bród, ale bardziej dla monster-trucków niż dla rowerów). Usiadłem zrezygnowany, zastanawiając się co dalej - ryzykować samotną przeprawę przez lodowaty nurt czy dać sobie spokój i wrócić do Skógar? Chcąc kupić sobie trochę czasu poszedłem za potrzebą i… dostrzegłem nieopodal przerzuconą przez potok kładkę dla turystów pieszych! Tuż za nią stała tablica z olbrzymią mapą terenu (i rozrysowanymi szkalami turystycznymi), której nie omieszkałem sfotografować. Nie dysponowałem bowiem fizycznie jakąkolwiek mapą papierową, tylko tym co zapamiętałem z mapy samochodowej 1:300 000.

Równo o 4:00 dojechałem pod pierwsze schronisko, zostawiłem rower i ruszyłem niebieskim szlakiem do Þorsmörk. W plecako-worku miałem zapas żywności i wody, do tego kalesony, rękawiczki, opaskę, polar, kurtkę… Choć w Skógar było dość ciepło, to 1000 m różnicy wzniesień i bliskość lodowców sprawia, że w nocy w górze temperatury oscylują koło 5 °C. O to, czy znajdę drogę wśród tych pustkowi przestałem się martwić, gdy tylko zobaczyłem eleganckie, solidne żółte słupy wbite w regularnych odstępach (wcześniej droga wyznaczona była rachitycznymi palikami, których końcówka pomalowana była na czerwono lub niebiesko - w zależności od koloru szlaku).

Poranne widoki były obłędne - czyste niebo, podnoszące się z dolin mgły, skrzące się w ostrym słońcu płaty śniegu, formacje z pumeksu, “pustynie” z popiołu… {zdjęcia}

Koło 6:30 zawitałem na pole biwakowe Básar - znajdujące się u wrót Þorsmörk - gdzie mogłem wreszcie nabrać wody (i skorzystać z kulturalnej toalety;) Ponieważ 0,5 l napoju w bidonie okazało się ilością niewystarczającą, więc wygrzebałem z pojemnika na butelki PET jedną “sprawną” sztukę - naprawdę, pomiędzy Básarem a Baldvinsskáli właściwie nie ma miejsca, gdzie można by uzupełnić wodę! A to ponad 700 m podejścia na dystansie ponad 10 km.

Ze względu na skrajnie mało czasu (umówiłem się z resztą o 11:00 pod Skógafossem) powrót przebiegał w błyskawicznym tempie. Znalazłem jednak chwilę, by zahaczyć o drugie schronisko - na samym Fimmvörðuháls. Ta chatka ma klimat i jeśli myślisz o pieszej wędrówce po tamtej okolicy (i stać cię na zapłacenie 5000 kr za noc) - gorąco je polecam! Schroniskiem zarządza \href{http://www.utivist.is/english}{Utivist}, takie islandzkie PTTK.

Na koniec spotkało mnie jeszcze jedno rozczarowanie, bo zjazd odbywał się z niewiele większą prędkością niż wyjazd. Ach… gdyby tak mieć opony jak mijające mnie monstertrucki… albo gdybym nie wiózł na bagażniku całego dobytku i nie obawiał się o uszkodzenie bagażnika albo złapanie gumy… W połowie drogi spotkałem dziewczyny, z którymi rozmawiałem w nocy - jak się okazało, były to Polki pracujące w wakacje na w Skógar, w branży turystycznej. Z ciekawych rzeczy, które opowiadały, to że szef funduje im atrakcje w stylu: wypad na lodowiec, bilety autobusowe z Þorsmörk (niby 40 km, a kosztują ze 100 zł), rejs na Wyspy Zachodnich Ludzi...

\hint{Dokładna relacja z 2-dniowej wyprawy (nie, nie mojej:) szlakiem ze Skógar do Þorsmörk znajduje się \href{http://adrian-harvey.com/2012/07/21/the-fimmvorduhals-diary/}{tutaj}, natomiast \href{http://www.volcanohuts.com/fimmvorduhals}{tu} można poczytać opis w wersji skróconej.}

\section{Pustkowia po raz pierwszy}

W Vík przygotowaliśmy pierwszy podczas naszej wyprawy obiad polowy - pierwszy, bo w poprzednich dniach po prostu nie było sprzyjających warunków do gotowania na świeżym powietrzu! Tym razem przycupnęliśmy we wnęce w ścianie marketu i tak - osłonięci od wiatru - spokojnie gotowaliśmy. Ludzie patrzyli się na to zaintrygowani, lecz nikt nie próbował nas stamtąd wyrzucać :)

Droga za Vík to coś tak nurzącego… Pola, pola, aż po horyzont pola łubinu…Żadnej drogi w bok, żadnego zabudowania, tylko od czasu do czasu miejsce postojowe.

\img{./photos/x-s-2014-07-27_18-43-10__49.jpg}{vik_fields}{Znużeni jeźdźcy gdzieś w szczerym polu za Vík}

Nocleg zaplanowaliśmy na kempingu w Hrífunes. Jeszcze jadąc po \road{209} wszystko wyglądało w porządku, przy drodze stał elegancki znak z symbolem kempingu… Problem pojawił się, gdy 50 metrów dalej ścieżkę przegradzał sznurek, na którym zawieszono kawałek dykty z napisem “LOKAД (isl. zamknięte). Wizja lokalna w zapyziałym budynku sanitariatów wykazała, że kemping przestał funkcjonować dobrych parę lat wcześniej. Nie było jednak sensu jechać gdziekolwiek indziej, więc rozbiliśmy się tuż przy wspomnianej tabliczc (akurat trafiła się tam mała polanka, w sam raz na dwa namioty).

W okolicy śpią też bracia Rosjanie (którzy też nastawiali się na nocleg na kempingu), co stało się przyczynkiem do mało wybrednych żartów: “Pamiętajcie, żeby uważać na gaz!”, “Pilnujcie dobrze rowerów… i zegarków!” bądź też “Ciekawe, czy nas zaanektują…”
	\chapter*{28.07.}

\section*{Obiad w Kirkjubæfarklaustur}

Od rana pogoda fatalna --- permanentna mżawka, a momentami faktycznie pada. Gdy jest tak chłodno i wilgotno człowiek zużywa znacznie więcej energii energii na ogrzanie siebie i --- siłą rzeczy --- szybciej robi się głodny. Trudno się więc dziwić, że gdy tylko zrobiliśmy zakupy w superkarkecie w Kirkjubæfarklaustur zapragnęliśmy zjeść ciepły obiad. Z początku projekt zdawał się nierealny, bo żadne miejsce w pobliżu nie miało potencjału na stanie się naszą kuchnią polową. I tu jak z nieba spadła nam pani z informacji turystycznej która stwierdziła, że ,,nie ma problemu, byście się rozgościli w jednym z naszych pokoi biurowych na piętrze''. Przyznam, że w takich warunkach --- wśród stosów segregatorów i drukarek --- jeszcze nigdy nie zdarzyło mi się gotować obiadu!

\img{./photos/x-s-2014-07-28_15-03-53__50.jpg}{ti_dinner}{Obiad w biurze informacji turystycznej Kirkjubæfarklaustur - któż by się spodziewał takich luksusów?}

\section*{Pustkowia po raz drugi}

Obiad zjedzony --- dzień zaliczony. Pora ruszać dalej w drogę. Ta jak zwykle nie była przesadnie urozmaicona --- po raz kolejny już dziś jedziemy prostymi jak strzała odcinkami. Dosłownie! Odcinki są tak proste, że gdy po 10 czy 15 km nagle pojawia się zakręt, to prowadzący grupę, niczym GPS, komunikuje reszcie \emph{,,za 300 metrów lekko skręć w lewo''}!

Z mijanych osobliwości: gość w koparce, który gładzi łyżką żwir wokół swojej maszyny. Komentarz Puta --- \emph{,,To taki ich program walki z bezrobociem?''} --- bezcenny.

\img{./photos/x-s-2014-07-28_22-06-42__95.jpg}{sandur}{Sandury. Miejsce postojowe. Ławka i… DRZEWO?}

Droga może nie byłaby taka zła, gdyby nie mgła i silny frontowy wiatr. Zgoda, zdążyliśmy się już przyzwyczaić, że ,,na Islandii wieje zawsze przeciwnie do kierunku jazdy'', lecz że wieje w twarz nawetr po skręcie o niemal 180° (gdy zjechaliśmy z \road{1} w stronę kempingu)?! To już zakrawa na ponury żart.

%% TODO: grafika "na Islandii zawsze wieje przeciwnie do kierunku jazdy"

Kemping w Skatafell można podsumować krótko --- zdzierstwo! Za nocleg liczą sobie 1000 kr, prysznic to wydatek kolejnych 500 kr, ładowanie komórki --- 200 kr, a do tego nie ma nawet gdzie wysuszyć rzeczy (bo czy drewniana buda ze sznurkami, lecz bez żadnego ogrzewania, liczy się jako suszarnia?).

	\chapter*{29.07. --- Jak Islandia to musi być lód!}

Z rana wybraliśmy się oglądać sztandarową atrakcję Skaftafell --- wodospad Svartifoss, którego ściany przypominają olbrzymie bazaltowe organy. Potem podeszliśmy jeszcze kawałek wyżej, na punkt widokowy Sjónarsker, z którego widać było sandury w całej ich okazałości.

Podczas tego spaceru Kasia opowiedziała nam o ciekawych zajęciach, w jakich uczestniczyła w Oslo podczas Erasmusa. Chodziło o cyklu wykładów poświęconych zjawisku imigracji w Europie w ujęciu społecznym. Między innymi poruszana była na nich kwestia praworządności. Otóż nacje Europy dzielą się na ,,czerwone'' i ,,niebieskie''. Niebiescy generalnie ufają władzy, przestrzegają prawa, są przykładnymi obywatelami, czerwoni --- wprost przeciwnie. O ile niebiescy bardzo dobrze prosperują w okresie ,,pokoju'', o tyle w~sytuacjach kryzysowych zwyczajnie głupieją, gdyż brakuje im szczegółowych instrukcji ,,co robić na wypadek kryzysu''. U czerwonych rzecz ma się dokładnie na odwrót --- na co dzień ich krętactwo psuje system, lecz właśnie ten brak poszanowania dla reguł i kombinatorstwo sprawia, że doskonale odnajdują się w trudnych chwilach. Polacy oczywiście są ,,czerwoni'' :)

\img{./photos/x-s-2014-07-29_13-44-46__150.jpg}{sandurs_from_above}{Sandury, sandury\textellipsis aż po horyzont sandury! }

Tak więc, jako przedstawiciele ,,czerwonej nacji'', dokonaliśmy aneksji większości pralni na cele suszarni, a do tego --- wbrew zakazowi --- podładowaliśmy komórki w tamtejszych gniazdkach. Aż dziw, że nikt inny nie wpadł na taki pomysł! Może to właśnie ta różnica w~mentalności? Mniejsza z tym, ważne że wreszcie rzeczy są suche!

Z kempingu wyruszamy w dobrych nastrojach, przy w miarę stabilnej pogodzie. Kręcimy kilometry, kręcimy, aż zajeżdżamy do Fjallsárlón. Była to pierwsza widziana przez nas laguna lodowcowa i zgodnie stwierdziliśmy, że to dość urokliwe miejsce. Szybko jednak pojawiły się głosy, że to jednak nie jest \emph{ta} laguna, którą oryginalnie planowaliśmy zobaczyć. Woda jakaś taka mętna, fok brak i coś mało ludzi\textellipsis

I faktycznie, do słynnej laguny Jökulsárlón należało jeszcze podjechać jeszcze kolejne 10~km. Tam, w promieniach zachodzącego słońca, mogliśmy podziwiać wszystko to, co pokazywały foldery reklamowe: sunące majestatycznie górki lodowe, polujące foki i mewy oraz\textellipsis tłumy turystów.

\img{./photos/x-s-2014-07-29_20-02-07__176.jpg}{lesser_lagune}{Mała laguna lodowcowa\textellipsis}

\img{./photos/x-s-2014-07-29_21-09-00__114.jpg}{greater_lagune}{\textellipsis i jej większa krewna!}

Pod wieczór wiedzieliśmy już, że nie dojedziemy na zaplanowany kemping nieopodal Vagnsstaðir. Szybka kontrola zapasów wykazała, że brakuje nam kluczowego składnika porannej paszy --- mleka. Zdecydowaliśmy się dokonać jego interwencyjnego zakupu w przydrożnej restauracji Hali Country Hotel. O, żebyście widzieli minę ludzi z obsługi, gdy zobaczyli dwóch chłopaków wchodzących w obcisłych wdziankach i usłyszeli, że chcemy kupić karton mleka! Choć podyktowana przez szefową restauracji cena była czarnorynkowa --- 400~kr za litr --- to każdy z nas w duchu cieszył się, że jutro z rana nie zajdzie konieczność konsumowania owsianej mamałygi na wodzie.

Ostatecznie zanocowaliśmy na pięknym, miękkim ,,mechowisku'' raptem cztery kilometry za wspomnianą restauracją. Gdyby tylko tak nie wiało\textellipsis Bo znów przy silniejszych podmuchach nasz ,,męski'' namiot przypomina czapkę smerfa!
	\chapter*{30.07.}

Po raz pierwszy budzi nas piękne słońce (a do tego właściwie brak wiatru)! Nie na długo…

“Co ci Islandia da przed zakrętem, zaraz za zakrętem odbierze.” - oto nowe powiedzenie ukute przez Puta, które jak ulał pasuje do islandzkch warunków. Gdy tylko wyjechaliśmy za najbliższy cypel, w twarz uderzył nas wiatr tak silny, że raz Karolina autentycznie niemal wylądowała w rowie. Nasza prędkość - na prostym, równym odcinku - spadła do 8 km/h, a miejscami ledwo byliśmy w stanie wyjechać na choćby niewielkie wzniesienia! Za to jakieś 15 km przed Höfn wreszcie skręciliśmy tak, że zaczęło wiać w plecy i dzięki temu kawałek przed wjazdem do miasta mogliśmy spokojnie rozpędzić się na prostych do 50 km/h. (Policjanci z drogówki mieli spory ubaw obserwując nasze wyczyny :)

\img{./photos/x-s-2014-07-30_08-48-02__59.jpg}{moss_camping}{Słońce po raz drugi!}

\hint{Przy silnym wietrze należy szczególnie uważać na mostach (zarówno przy wjeździe jak i zjeździe), przy przejeżdżaniu przez niewielkie otwarte przestrzenie w stylu wiadukt oraz gdy mijają cię większe pojazdy (choćby duży jeep). Dlaczego? Początkowo walczysz z wiatrem. Potem zaczyna mijać cię tir - zasłania ci wiatr, a ty zjeżdżasz w stronę środka jezdni. Potem kończy cię mijać i wtedy ponownie uderza w ciebie wiatr - lądujesz w rowie. Na moście bywa śmieszniej, bo odbijający się od wysokich barierek i murków wiatr lubi tworzyć swoisty tunel aerodynamiczny…}

W Nettó w Höfn dokonujemy po raz pierwszy “królewskich zakupów” - potrzebujemy bowiem jedzenia na 3 dni, a przy okazji nie krzywdujemy sobie - w koszyku ląduje fura ciastek, owoce w puszce w dużych ilościach, sok pomarańczowy… Do pełni szczęścia brakuje tylko zacisznego miejsca na zrobienie obiadu. Po chwili znajdujemy takowe - tuż obok tylnego wejścia do supermarketu stoi solidny drewniany ogrodowy ławo-stół. Na drugim daniu się nie skończyło, w ramach deseru konsumujemy Marijki z Szoko-szoko!

\img{./photos/x-s-2014-07-30_16-10-44__60.jpg}{xtra_dinner}{Produkty marki X-tra podstawą zdrowej diety :)}

Kawałek za Höfn niespodzianka - tunel, jeden z 3 czy 4 na całej Islandii. Rozpoczyna się nerwowe szukanie kamizelek, które co po niektórzy wrzucili na samo dno sakwy. Generalnie tunel jest krótki i dość dobrze oświetlony, lecz wiadomo - przezorny zawsze ubezpieczony.

Kasia: “Co się może zdarzyć na ostatnich kilometrach przed kempingiem?” Otóż na ostatnich kilometrach może znowu powiać w mordę z prędkością dochodzącą do 15 m/s - zjeżdżamy z dość stromej górki ledwo dokręcając (na przełożeniach 1-3 bądź 1-4) do 8 km/h.

Zaraz za mostem prowadzącym do Stafafell widzimy znak “kemping 7 km” i strzałkę na drogę szutrową. Ponieważ nie ogarnęliśmy, że jeszcze jeden kemping znajduje się za 2 km (tuż przy \road{1}), więc chcąc niechcąc ruszamy w podróż po żwirze. Po 1 km jazdy mamy już serdecznie dość wertepów i po prostu rozbijamy się na minipolance przy drodze.

Wieczorem wreszcie znaleźliśmy chwilę czasu i ochoty, by pograć w karty. Nic ambitnego - makao. Ale ile emocji wzbudza samo ustalanie wspólnej wersji zasad!

\img{./photos/x-s-2014-07-30_23-04-25__117.jpg}{camping_stafafell}{…i po raz kolejny kemping w pięknych okolicznościach przyrody!}

	\chapter*{31.07.}

\section*{Na grani}

Jeszcze będąc w Höfn zawitałem do tamtejszej informacji turystycznej z pytaniem: co warto zobaczyć w okolicy? Pan chwilę poskrobał się w głowę, mój pomysł wyprawy w górę \road{F980} skwitował ,,szaleństwo, droga zalana'', w końcu zasugerował spacer do wąwozu Hvannagil (ten region Islandii nazywa się Lónsöræfi). Był nawet na tyle miły, że od razu wydrukował mi mapę szlaku i instrukcje jak odnaleźć jego początek.

Pierwszą próbę zobaczenia wspomnianego wąwozu podjąłem wczoraj wieczorem, lecz trochę pobłądziłem i ostatecznie dałem sobie spokój. Dziś skoro świt postanowiłem spróbować raz jeszcze, tym razem idąc najpierw w górę rzeki Jökulsá í Lóni.

Pierwszą niespodzianką było to, że spacer odbywał się dnem kanionu. Fakt, niby płynęło tam parę strumieni, lecz żaden nie był na tyle głęboki czy rwący by nie dało się go przekroczyć idąc po kamieniach lub przeskakując z jednego brzegu na drugi. Późniejsza wspinaczka na ścianę kanionu również nie nastręczała większych problemów i dlatego w ledwie 1-1,5 godziny byłem już za połową trasy. Popatrzyłem na okoliczne szczyty --- piękne, wyraźnie górujące nad okolicą, aż proszą się by z ich wierzchołków podziwiać ziemię u stóp. Spojrzałem na zegarek --- była 7:30. Reszta towarzystwa obudzi się pewnie dopiero za ponad dwie godziny. Niewiele myśląc powziąłem decyzję: \emph{,,A, zdobędę ten najbliższy!''} Dopiero w Krakowie udało mi się gdzieś wyszperać informację, że ta kupa kamieni, na którą przez blisko 40 minut wdrapywałem się na czworaka, ma swoją nazwę --- Grakinnartindar. Wędrówka była wyczerpująca, bo każdy krok w górę wiązał się z równoczesnym zsunięciem pół kroku w dół --- takie to było rumowisko. Gdzieniegdzie trafił się kamień większy niż pięść i wtedy można było na chwilę przystanąć i odsapnąć, póki i on nie zaczynał się osuwać razem ze mną. Niemniej gdy wreszcie osiągnąłem wierzchołek, to rozległa panorama na ośnieżone łańcuchy górskie, na rozczapierzone ,,palce'' u ujścia Jökulsá í Lóni i na wąwóz Hvannagil zrekompensowały mi wszelkie trudy (z dwukrotnym przekraczaniem kanionu ,,na dziko'' włącznie). Miłym akcentem była możliwość --- po raz pierwszy w życiu --- naprawdę długiego zbiegania ze szczytu ,,na azymut'' i to zupełnie bez patrzenia pod nogi. Te kamyki (właściwie tłuczeń wysypywany pod tory kolejowe) były tak luźno związane z podłożem, że bieg przypominał ,,płynięcie''. Ryzyko skręcenia kostki: znikome.

\img{./photos/x-s-2014-07-31_10-44-48__131.jpg}{winner_triumph}{Tryumf zwycięzcy}

\curiosity{To, że ten szczyt nazywał się akurat Grakinnartindar dowiedziałem się z portalu \href{http://www.islamicfinder.org/prayerDetail.php?country=iceland\&city=Grakinnartindar\&id=4471\&lang=}{Islamic Finder}, który podaje godziny modlitw w danym miejscu na ziemi. Porównałem współrzędne tam podane z tym co pokazywały Google Mapsy --- zgadza się!}

\hint{
Krótki opis trasy z ,,przewodnika'' otrzymanego w Höfn: Standardowa trasa trwa ok. 4 godz. Zaczyna się w Stafafell, koło pensjonatu i kościoła --- żwirowa droga prowadzi w górę wzgórza do małego kurnika. Potem należy podążać za ścieżką wydeptaną przez owce na szczyt wzgórza, przekroczyć płot i znów iść po owczej ścieżce --- aż do kanionu. Powrót brzegiem rzeki Jökulsá í Lóni. \newline \processifversion{PDF}{Mapę okolic Stafafell znajdziesz tutaj: \url{http://www.stafafell.is/uploads/8/3/3/1/8331287/5674623_orig.jpg}}
\processifversion{HTML}{Mapę okolic Stafafell znajdziesz \href{http://www.stafafell.is/uploads/8/3/3/1/8331287/5674623_orig.jpg}{tutaj}}
}.

%TODO: tytuł sekcji
\section*{xxx}

Dziewczyny były na tyle odważne (albo zdesperowane), że umyły się w tej mętnej polodowcowej rzece. Niby to nie żadne ścieki tylko drobinki piasku i ziemi, ale i tak moja reakcja była dość jednoznaczna --- \emph{Yy… fuuuj…} ;-)

Od pewnego Polaka, przypadkowo spotkanego w Nettó w Höfn, dowiedzieliśmy się o istnieniu ,,sklepu na CPN-ie w Djúpivogurze''. Zależało nam na zakupach, więc dokładaliśmy starań, by zdążyć tam na jakąś rozsądną porę. Jakież było nasze zdziwienie, gdy skoro tylko zajechaliśmy tam na 15:40, pani ekspedientka powitała nas od razu tekstem: \emph{,,Róbcie proszę zakupy szybko, bo zaraz zamykamy.''} Że co proszę? Miejscowość liczy 350 dusz, a sieciowy supermarket Samkaup zamykają w dzień powszedni o 16:00? Naprawdę?

\img{./photos/x-s-2014-07-31_19-57-52__224.jpg}{dead_deer}{Hm… rzeźba ogrodowa w islandzkim stylu?}

\img{./photos/x-s-2014-07-31_21-33-03__148.jpg}{berufjorthur}{Majestatyczne szczyty okalające Berufjörður}

No nic, zakupy zrobione, zaczęliśmy szukać miejsca na przygotowanie obiadu. Po dłuższej naradzie wybraliśmy obiad ,,na krzywy ryj'' w kuchni na lokalnym kempingu. Zachęciła nas do tego informacja, że jego właściciele rezydują w hotelu oddalonym o pół kilometra, zatem ryzyko że ktoś nas wyrzuci jest znikome. Tam właśnie, w tej przytulnej kuchni, poznaliśmy naszych późniejszych nieodłącznych towarzyszy na rowerowej trasie: Polkę z Poznania (też wraca do Berlina w dniu 18. sierpnia) oraz Krisa i Nelly --- parę Holendrów których już jutro powinniśmy mijać w drodze do Egilsstaðir .

Ostatnie 10 km drogi to znów wygwizdów, a nam --- po sutym obiedzie --- niezbyt chce się pedałować… A, no i jeszcze w pewnym momencie skończył się asfalt na \road{1}, co dodatkowo psuło nam przyjemność z jazdy z pełnym żołądkiem.

\img{./photos/x-s-2014-07-31_22-48-21__226.jpg}{malbik_endar}{Cóż, nawet na \road{1} czasem ,,Malbik Endar''…}
%TODO: wytłumaczyć w stopce - koniec asfaltu

Wieczorem doświadczamy standardowego problemu z kempingami zaznaczonymi na mapie --- mianowicie ich braku. Niby na niemieckiej mapie mamy przy skręcie na Öxi jak byk zaznaczony czerwony namiocik, lecz zagadnięty o to tubylec w sile wieku stwierdził: \emph{,,Kemping? Mieszkam tu od dziecka i nigdy żadnego kempingu nie było!''} Potem poradził albo jechać dalej 10 km po \road{1} albo cofnąć się z 5 km. My oczywiście zrobiliśmy po swojemu i zanocowaliśmy na dziko.
	
	\chapter*{01.08. --- Wypluwamy płuca na Öxi}

Ta noc była zimna jak nigdy --- nie dość, że zawiewało z gór, to jeszcze ciągnęła wilgoć od zatoki, a momentami temperatura spadała poniżej zera. Skutkiem tego o poranku, wbrew zapowiedziom z dnia poprzedniego, większość wycieczki nie umyła się w (czystej!) górskiej rzece \wink

\hint{Choćby pogoda była paskudna, minimum ,,higieny rowerzysty'' trzeba zachować --- umyć pachwiny, pachy, stopy\textellipsis Owszem, nie zawsze to przyjemne, lecz wierzcie mi --- zapocone odparzenia i~czyraki są znacznie gorsze! Aha, zęby można myć w namiocie i~pluć przez wejście \wink}

Coś zawiodło, bo zaplanowana na 8:00 pobudka nie doszła do skutku z powodu ,,oporu materii''. Wyruszyliśmy dopiero koło 10:30. Na szczęście nic to, bo do przejechania mamy ledwie 65~km.

Wyjazd na Öxi --- oto nasze główne dzisiejsze zadanie bojowe. Trasa liczy sobie 19~km po drobniutkim szutrze, a po drodze trzeba zaliczyć parę podjazdów o nachyleniu 17\% (i~do tego jeszcze wiele mniejszych). Z początku wyglądało to groźnie: ,,17\%?! Toż to prawie pionowa ściana!'' Nic bardziej mylnego. Owszem, na jednej ściance trzeba było podprowadzić rower, lecz generalnie bardzo dobry stan nawierzchni (niemal jak asfalt) umożliwiał sprawne pokonanie większości podjazdów.

\img{./photos/x-s-2014-08-01_12-46-26__149.jpg}{oxi_17_perc}{Podjazd 17\%? Pff\textellipsis}

\pagebreak

W związku z nadchodzącą ulewą zaraz na krzyżówce z \road{1} urządzamy sobie piknik pod płachtą --- tak tak, ta wożona dotychczas bez celu płachta budowlana wreszcie znajduje praktyczne zastosowanie! Cztery osoby bez problemu mogą się pod nią wygodnie schować, co jeszcze nie raz będziemy wykorzystywać\textellipsis

Gdzieś w którejś z zamieszczonych w internecie relacji można było wyczytać, że ,,zjazd z Öxi do Egilsstaðir to sama poezja --- 50~km zjazdu''. Błąd. Po drodze liczne podjazdy, momentami konkretne (choć może nie 15 czy 17\%), a że akurat wiatr wieje --- standardowo --- w twarz, to trzeba dokręcać, aby utrzymać prędkość. Od czasu do czasu trochę pada. Mimo to nie narzekamy na pogodę, bo w porównaniu z warunkami na przełęczy było cudnie. Öxi niby ma marne 532~m~n.p.m., lecz tak pizgało i było tak chłodno, że smarki niemal zamarzały. Na nagrodę przyszło nam czekać aż do dojazu na rogatki miasta --- słońce zaświeciło tak intensywnie, że można się było opalać. Bez zaciskania zębów z zimna.

\img{./photos/x-s-2014-08-01_14-36-57__233.jpg}{oxi_kasia}{Kasia dzielnie walczy ze stokiem}

\img{./photos/x-s-2014-08-01_16-07-31__155.jpg}{sheet_lunch}{Pierwszy podczas wyjazdu posiłek pod płachtą}

Na kemping zajeżdżamy dopiero o 17:50 i niezwłocznie przystępujemy do realizacji planu pod tytułem ,,dzień turysty'': ciepły prysznic, pranie, gotowanie na spokojnie, przegryzane orzeszkami wieczorne piwo (Viking Light --- czy to wciąż jeszcze można nazwać piwem? Smakuje jak lekko gazowana woda z delikatną nutką chmielu!)\textellipsis Nawet nie wspominam już o prądzie, a więc także i wi-fi.

Gdy idziemy spać (tj. koło północy) Kasia staje przed namiotem, przygląda się mu uważnie i zdziwiona pyta: \emph{,,Ej, dlaczego on taki sztywny?!''} Podchodzi bliżej, maca, patrzy --- a na jej palcach roztapiają się płatki lodu. Z powodu przymrozka jakaś para porzuciła swój namiot i umościła się na materacu w łazience, tuż przy grzejniku. Czyżby obawiali się zamarznąć?

\hint{Na Islandii nigdy nie należy czynić żadnych założeń odnośnie pogody. A już szczególnie nie należy próbować przewidywać kierunku wiatru --- lepiej przyjąć, że zawsze wieje w twarz. Co za tym idzie, wszelkie planowanie bądź ekstrapolacja czasu przejazdu mija się z celem\textellipsis}

(Luźna impresja.) Krajobrazy islandzkie są skrajnie różne od tego, co znamy z ,,Europy kontynentalnej'' --- to, że łatwiej tu o spotkanie z owcą niż z człowiekiem, to żadna nowość (patrz: liczne górskie regiony Walii czy holenderskie wybrzeże Morza Północnego), ale brak miejscowości przy drodze (a w zasadzie --- czegokolwiek!) to już zupełnie inna bajka! Słowo daję, wszelkie zabudowania na islandzkiej prowincji oddalone są od \road{1} co najmniej o 1 km, co dodatkowo potęguje wrażenie jazdy przez pustkowia.

\hint{Na islandzkich kempingach w zasadzie nie trzeba pilnować swoich rzeczy. Podobnie nie trzeba przypinać rowerów idąc do sklepu bądź w góry na wycieczkę. Istne szaleństwo! Według mnie wytłumaczenie tego fenomenu może być tylko jedno --- miejscowych prawie nie ma (a nawet jak są, to nie kradną --- taka kultura), a~przyjezdni to albo nadziani emeryci z Niemiec, Francji czy USA, albo wagabundzi, którzy mają swój kodeks honorowy i nie rabują innych podróżnych :)}

%\img{./photos/x-s-2014-08-01_21-44-01__65.jpg}{laundry_party}{Na zakończenie dnia --- impreza w pralni.}
	\chapter*{02.08. --- Dzień Turysty}

\hint{To bardzo ważne, by raz na czas --- po solidnym, długotrwałym wysiłku --- zrobić sobie dzień odpoczynku. Wpływa to korzystnie nie tylko nie ciało, ale także na “ducha” --- poprawia nastrój w drużynie.}

Cisnęliśmy bez przerwy od 11 dni, z czego przez ostatnie --- ostro pod wiatr, nocując w miejscach gdzie hulał lodowaty wiatr, czasem bez możliwości odpowiedniego umycia się. Widać było, że forma i nastroje zaczynają siadać, częściej pojawiały się spięcia i konflikty, zaistniało ryzyko “zmęczenia materiału”. Dlatego też dzisiejszy dzień postanowiliśmy niemal w całości spędzić w Egilsstaðir --- tak na luzie.

Wreszcie znalazł się czas, by na spokojnie zrobić pranie i zakupy, by pogawędzić z innymi turystami-rowerzystami (w tym z naszymi Holendrami), by napisać kartki do rodziny i znajomych, trochę nic-nie-robić i by zjeść obiad bez pośpiechu. Dla chętnych znalazła się dodatkowa atrakcja w postaci basenu --- co prawda bez gorącej wody (z wyjątkiem dwóch “niecek”), ale za to z trzema pustymi torami. Ach… nic tak nie rozluźnia mięśni pleców i nóg jak solidna porcja żabki i kraula!

\hint{Za wstęp na islandzki basen płaci się zazwyczaj około 500-600 kr, wejście jest bez limitu czasu (można siedzieć od rana do wieczora;) Nie ma wymogu zakładania czepka.}

Zachciało nam się przejechać dziś jeszcze takie symboliczne 30 km --- głównie po to, by nie płacić za kolejną noc na kempingu. Przed wyjazdem każde z nas wzięło jeszcze ostatni prysznic w ciepłej wodzie i o 18:30, przy dość ładnej pogodzie, ruszamy w dalszą trasę.

Pod Nettó, tuż obok kempingu, spotkaliśmy ponownie Panią Polkę. Tym razem miała ona ze sobą pełny ekwipunkek --- na 29-calowym potworze zawisły wyładowane do granic możliwości sakwy (przednie i tylne), solidna torba na kierownicę (na olbrzymią lustrzankę), na bagażniku leżał sobie statyw tak solidny, że pewnie i kamerę telewizyjną by utrzymał… Trzeba przyznać, że wzbudzało to respekt! Tylko jej opowieści nie brzmiały zbyt wiarygodnie, bo gdy ktoś twierdzi że przejeżdżał przez Öxi ledwie godzinę po tobie przy pięknej słonecznej pogodzie, a ty męczyłęś kilometry w lekkiej, niemal marznącej mżawce, to jak tu takiemu komuś wierzyć?!

\img{./photos/x-s-2014-08-02_12-14-29__66.jpg}{wool_market}{Standardowy asortyment typowego islandzkiego supermarketu}

Aż do okolic wiaduktu nad kanionem, którym płynie rzeka Jökulsá á Brú, \road{1} przypomina jazdę kolejką górską w wesołym miasteczku --- to pnie się kawałek w górę, to trochę opada, sumarycznie jednak zdobywa się wysokość. Za to ostatni zjadz --- na wiadukt --- jest długi i piękny, to taka nagroda. Później droga wznosi się już bardzo łagodnie --- tak łagodnie, że Put autentycznie myślał, że jest wręcz po równym ;)

Obozowisko założyliśmy kawałek za miejscem, gdzie od \road{923} odbija jakaś nienazwana szutrówka. Dosłownie dwa metry od drogi znaleźliśmy dość równe trawiaste poletko, gdzie szpilki wchodzą w podłoże jak w masło --- ale się trzymają fest!

Rzut oka na licznik --- 50 km. Ot, taki sobie dzień turysty!

	\chapter*{03.08. --- Interior}

Droga do Brú jest jeszcze całkiem znośna --- owszem, sporo jeżdżenia góra-dół, ale nawierznia niczego sobie (nawet asfalt mają po wioskach!)

Gdy zajechaliśmy do ,,centrum'' Brú i stanęliśmy pod drogowskazami, aż przetarliśmy oczy ze zdumienia --- jak byk stoi ,,Askja 88''. Czyżby mapa nas okłamała? Łudzimy się jeszcze, że może to odległość na szczyt Askji --- a ponieważ my jedziemy na kemping Dreki, u jej podnóża, to akurat wyjdzie 78 km.

Zaraz na pierwszym szutrowym podjeździe dopadają nas deszcze konwekcyjne --- tym razem wiemy już, co robić. Znów plandeka idzie w ruch, znów wykorzystujemy tą chwilę na przygotowanie i konsumpcję kanapek.

Po paru kilometrach napotkaliśmy pierwszy bród. Przy jego przekraczaniu mieliśmy jeszcze sporo radości --- zwłaszcza że był na tyle płytki, że od biedy można by przejechać przez niego rowerem.

Zaraz potem wjechaliśmy w strefę permanentnych opadów. Podobno właśnie tu, na ciągnącym się na prawo i na lewo łańcuchu górskim, spada większość deszczu z ciągnących znad wschodniego wybrzeża chmur i dalej (nad interior) napływa już w miarę suche powietrze. Niby kropi, ale równocześnie promienie słońca bajecznie rozświetlają niewielkie kamyczki zalegające przydrożne tereny, tworząc wrażenie jechania przez srebrzyste morze! Szkoda, że strome podjazdy utrudniają spokojną kontemplację tych widoków ;-)

\img{./photos/x-s-2014-08-03_16-23-04__68.jpg}{rainy_moon_rocks}{Słońce, deszczyk, srebrzyste kamyczki…}

Jeden z brodów, na które natrafiliśmy na \road{F910} po zjeździe z przełęczy, nie był zaznaczony na naszej --- dość dokładnej --- niemieckiej mapie. Na mapie rowerowej ,,Cycling Iceland'' wręcz żaden z trzech brodów nie figurował! Akurat trafiliśmy na taki stan wody, że wystarczyło założyć klapki i jako tako dało się przeprowadzić rower, ale przy odrobinie wyższym stanie wody ryzykowałoby się zalaniem piasty. Natomiast to, co napotkaliśmy na odcinku za skrzyżowaniem z \road{F905}, to już były brody pełną gębą --- woda po kolana, dość silny nurt, w obu przypadkach nie obyło się bez demontażu bagażu.

\img{./photos/x-s-2014-08-03_18-53-50__165.jpg}{ford_riding}{Bród czasem można pokonać ,,w bród''…}
\img{./photos/x-s-2014-08-03_21-01-14__259.jpg}{ford_walking}{lecz bywa też, że trzeba się ,,spieszyć'' ;)}

\hint{Bród najlepiej przekraczać jego dolnym skrajem (czyli po tej stronie, w dół której płynie nurt), gdyż tam jest najpłycej. Przy większości brodów stoją tabliczki ilustrujące to w odpowiedni sposób.}

W pewnym momencie widzimy, że Kasia została wyraźnie z tyłu, że rozmawia z jakimiś losowymi ludźmi jadącymi samochodem i że coś sobie podają z ręki do ręki. O co chodzi?! Historia wyglądała tak: Kasia jedzie sobie spokojnie, w pewnym momencie czuje że jakoś tak lżej się jej jedzie, ale się tym nie przejmuje --- a wręcz cieszy! Jadący za nią samochód trąbi, lecz Kasia tego nie słyszy. Po chwili orientuje się, skąd to nagłe przyspieszenie --- zgubiła rzeczy z paki! Samochód za nią nadal trąbi, Kasia myśli sobie --- \emph{,,O wy ch*** głupie!''} Gdy wreszcie samochód dogonił Kasię, okazało się że jej zgubione rzeczy są w środku! Ci podróżni specjalnie zawrócili, żeby dowieźć zguby, a jakby tego było mało --- zaproponowali, że śmieci wezmą ze sobą :)

Niemal co parę kilometrów powtarza się mniej więcej taka wymiana zdań --- ktoś z ekipy pyta \emph{Dojedziemy w jeden dzień do Askji?}, na co Paweł pewnym głosem odpowiada \emph{Tak!}.

\img{./photos/x-s-2014-08-03_17-01-48__71.jpg}{road_to_askja}{Droga do Askji}

Gdy słońce chyliło się już ku zachodowi, a my wciąż przemierzaliśmy pustynię i nijak nie widać było końca naszej dzisiejszej wędrówki, zapragnęliśmy wreszcie zjeść coś konkretnego. Tylko gdzie tu gotować, kiedy wokół nas tylko piasek? Tu ponownie swą użyteczność wykazała płachta, którą obłożyliśmy oparte o duży kamień rowery tak, że całość tworzyła coś w rodzaju plażowego parawanu z okapem. Były to warunki partyzanckie i Put-kucharz wiele się natrudził, by wciąż na nowo podpalać gasnący palnik, lecz obiad przyszedł w samą porę (bo jak wiadomo --- głodny Polak to zły Polak;) Po takim posiłku znów mieliśmy siły (i odrobinę więcej ochoty), by jechać dalej.

Olbrzymie wrażenie zrobiła na nas księżycowa sceneria po przekroczeniu mostu na rzece Kreppa --- spękane płyty zastygłej magmy, dużo pumeksu, żwirek i wijąca się wśród wulkanicznych skałek droga. Nic dziwnego, że teren ten został objęty ochroną. A do tego jeszcze widok słońca zachodzącego za Herðubreið, którego szczyt nakryty był jakby kapeluszem chmur --- no po prostu magiczne!

Właśnie --- na terenie rezerwatu nie wolno się rozbijać. Czy jednak mieliśmy wybór, gdy o 23:00 wciąż znajdowaliśmy się w przysłowiowym (!) lesie? Nie zważając na zakazy, z duszą na ramieniu, założyliśmy obozowisko za dużymi skałami --- tak, by patrolujący okolicę \emph{rangersi} (czyli strażnicy parku) nie dostrzegli nas zbyt łatwo. Z tego samego powodu przed pójściem spać zatarłem jeszcze nasze ślady klapkiem Puta…

\img{./photos/x-s-2014-08-04_00-51-09__268.jpg}{herthubreith}{Herðubreið o zmroku}
\img{./photos/x-s-2014-08-04_08-37-36__75.jpg}{camping_national_park}{Breaking the law! (niestety)}

\hint{Płachta budowlana może służyć jako warstwa izolacyjno-ochronna od podłoża (m.in. żeby drobne kamyczki nie podziurawiły podłogi namiotu). Należy jednak zwrócić uwagę, by była ona w całości schowana pod namiotem! (Niezastosowanie się do tego grozi --- w przypadku deszczu --- powodzią w namiocie!)}

	\chapter*{04.08. --- Askja}

Tej nocy nikt nie przespał dobrze --- każdy raczej jak mysz pod miotłą --- a to wszystko za sprawą gozy, jaką napawała nas perspektywa płacenia kary za nocleg w miejscu niedozwolonym. Dręczyły nas koszmary senne, właśnie z rangersami w roli głównej --- mnie na przykład śniło się, że mandat wyniósł po 150 € na łebka, a cała nasza akcja z rozbijaniem się została uwieczniona na monitoringu. Niby rangersi byli Polakami i pewnie by nam to puścili płazem --- potraktowaliby nas jako wyprawę geologów --- lecz ich szefostwo już widziało nagrania, więc nie ma zmiłuj. Fabuła snów pozostałych uczestników naszej eskapady była mocno zbliżona. Tak więc co by się już więcej nie stresować, zwijamy się “wcześnie” rano (koło 8:00), odjeżdżamy paręset metrów od “miejsca zbrodni” i dopiero tam rozkładamy się ze śniadaniem.

Głównym naszym problemem dnia dzisiejszego jest brak wody pitnej. Od momentu przekroczenia ostatnich brodów nie mijaliśmy żadnego strumienia --- tak będzie już aż do Dreki.

\hint{Podczas jazdy przez interior wykorzystuj każdą okazję, by uzupełnić zapas wody! Wiele z rzek i strumieni zaznaczonych na mapie wypływa spod lodowców i jest na tyle mętna, że picie jej polega bardziej na piaskowaniu zębów…}

\img{./photos/x-s-2014-08-03_19-52-17__253.jpg}{interior_roads_1}{Codzienność w interiorze to "tarka"…}
\img{./photos/x-s-2014-08-04_12-53-21__269.jpg}{interior_roads_2}{…i piach}

Jazda dziś przypomina jakiś ponury żart, trochę w stylu \href{http://pl.wikipedia.org/wiki/Paradoksy_Zenona_z_Elei#Achilles_i_.C5.BC.C3.B3.C5.82w.5B2.5D}{paradoksu z Achillesem i żółwiem} --- tyle że tu pytanie brzmi “czy rowerzyści dojadą (dziś) na kemping?” Początko jedziemy dość dobrą drogą, z prędkością 12 km/h, do przejechania mamy 40 km więc snujemy już wizje, jak to za powiedzmy 4 godziny zajeżdżamy na kemping. (Tak, wiem że to naiwne, ale umysł niejednego rowerzysty przeprowadza często takie optymistyczne szacowanie.) Cóż, po paru kilometrach dobra droga kończy się, a zaczyna więcej żwirku. Nie jest źle, 9 km/h, czyli za 4 godziny będziemy. Na 15 km przed kempingiem (czyli na 3-4 km przed skrzyżowaniem z \road{F88}) zaczynają się takie wydmy, że nie ma bata --- właściwie non stop musimy pchać rowery. Nasza prędkość spada do 3-4 km/h, a więc wciąż na dotarcie do kempingu potrzebujemy koło 4 godzin… Ku naszej uldze, od skrzyżowania zaczynałą się swoista “autostrada”, którą w 40 minut dotarliśmy do celu. Ale to wcześniejsze asymptotyczne zmierzanie do celu było takie demotywujące!

\img{./photos/x-s-2014-08-04_14-45-35__271.jpg}{askja_vehicle}{Ech… czemuż nie założyliśmy i my takich opon?}
\img{./photos/x-s-2014-08-05_15-25-57__283.jpg}{impassible}{Impassible is nothing.}

Na skrzyżowaniu wita nas ręcznie wykonana tabliczka, że \road{F88} jest \emph{“Impassable for small jeeps due to high water level”}. Że co? Niedługo później sprawa się wyjaśniła: dozorca w Dreki powiedział nam, że lodowcowa rzeka zalała drogę na odcinku ok. 500 m tak, że woda sięga do kolan --- więc raczej nie przejedziemy. Na chwilę obecną nie martwimy się tym przesadnie, nasz plan wygląda następująco: jedziemy jutro do Herðubreiðarlindir i zobaczymy, co dalej --- albo miniemy rzekę “na dziko” (przez pustynię) albo będziemy żebrać u jakiś Niemców poruszających się po interiorze w takich księżycowych pojazdach o zabranie naszych rzeczy albo --- w ostateczności --- poczekamy na autobus do Mývatn o 15:30, który powinien przewieźć nasze bagaże.

O 15:00, po blisko sześciu godzinach jazdy i pchania, wreszcie osiągamy kemping Dreki. Jawi się on nam jako oaza: sanitariaty, olbrzymia i dobrze wyposażona wspólna kuchnia z jadalnią, możliwość rozbicia się w “cieniu wiatrowym” budynku mieszkalnego… Niestety, nie ma prądu, a z ciepłą wodą też bywają problemy --- choć prysznic kosztuje standardowe 500 kr, to lejącą się wodę można określić bardziej jako letnią niż ciepłą, co empirycznie sprawdziła na sobie Karolina.

Gdy już się rozbiliśmy, zjedliśmy obiad i odpowiednio ubraliśmy, wyruszyliśmy na pieszą wycieczkę nad jezioro Viti. Cel: kąpiel w gorących źródłach (stąd połowę naszego worko-plecaka zajmują stroje kąpielowe, ręczniki i klapki). Pogoda nie rokuje: mgła, deszcz, a w wyższych partiach --- padający śnieg. Dodatkowo zejście nad Öskjuvatn utrudniają płaty śliskiego śniegu (a tylko ja mam buty górskie)… Wszystko to sprawiło, że po dojściu w okolicę Bathsraun samotnie pobiegłem na króciutki zwiad, który niestety nie potwierdził bliskości jeziora Viti. W związku z tym oraz z uwagi na pogarszającą się pogodę zarządziliśmy odwrót. W sumie… może to i dobrze? Słyszeliśmy, że ledwie tydzień wcześniej potężna lawina błotna sprawiła, że poziom zwierciadła wody w Öskjuvatn podniósł się o parę metrów, przez co chłodna woda przelała się do Viti --- nie dość, że źródło przestało być chwilowo gorące, to jeszcze ścieżka została uszkodzona.

%% TODO: umieścić to w tekście
\curiosity{Nazwa “Bathsraun” ma interesującą etymologię --- sprawdź \href{http://davemcgarvie.wordpress.com/2012/04/}{skąd się wzięła.}}

\img{./photos/x-s-2014-08-04_20-35-49__186.jpg}{dreki}{Dreki --- oaza na bezkrestnym pustkowiu}
\img{./photos/x-s-IMG_20140804_203225178.jpg}{way_to_viti}{Wyprawa nad jeziorko Viti --- typowe islandzkie lato}

Po zejściu do Dreki urządzamy sobie istną ucztę --- chleb z konserwą i darmową musztardą, puree ziemniaczane, ciasteczka imbirowe moczone w herbacie… W międzyczasie przy właściwie wszystkich pozostałych stolikach odbywa się konsumpcja mniej lub bardziej procentowego alkoholu --- ci lekko już zawiani emeryci z Niemiec i Francji dość osobliwie patrzą na nas, gdy nożem wygrzebujemy z puszki mięso.

Dużą radość sprawiło nam przysłuchiwanie się dialogowi Pana Ruska z Panem Anglikiem. Dialog zainicjował Pan Rusek, który widząc łyżko-widelco-nóż spytał z charakterystycznym rosyjskim akcentem: \emph{“What is this? I don’t understand… Why not normal things?”} Aż parsknęliśmy śmiechem, gdyż nasze wcześniejsze i późniejsze doświadczenia faktycznie wykazały wyższość tradycyjnych, metalowych sztućców nad plastikowymi niezbędnikami --- szczególnie w przypadku widelca.

Put miał awarię kwaśnego mleka w kartonie, na szczęście uszkodzenie kartonu było niewielkie, także gęste kwaśne mleko nie zalało wszystkiego. Nie obyło się jednak bez mycia sakwy i części rzeczy…

\hint{Wszelkie płyny (jak sok bądź mleko) warto wozić w plastikowych butelkach. Lepiej stracić trochę cieczy podczas operacji przelewania niż póżniej wąchać zakiśnięte mleko…}

\processifversion{PDF}{
\hint{Mapę okolic Askji znajdziesz na stronie internetowej parku narodowego Vatnajökull: \url{http://www.vatnajokulsthjodgardur.is/media/fixlandia/N\&A_100k_EN_20131022.jpg}}
}

\processifversion{HTML}{
\hint{Mapę okolic Askji znajdziesz na \href{http://www.vatnajokulsthjodgardur.is/media/fixlandia/N\&A_100k_EN_20131022.jpg}{stronie internetowej} parku narodowego Vatnajökull.}
}
	\chapter*{05.08.}

Pierwsze, co dziś o poranku słyszymy z namiotu dziewczyn, to soczystą wiązankę. Cytować nie będę, ale przytoczę sens, parafrazując za prof. Miodkiem: \emph{,,Bardzo mnie irytują te powstałe na podłodze namiotu kałuże, które sprawiły iż moja mata samopompująca oraz śpiwór nadmiernie zawilgły. Nie nastraja mnie to pozytywnie do dalszej podróży''}. Co spowodowało ten kataklizm? Właśnie to, że nieumiejętnie rozłożyliśmy folię i cała woda z tropiku spływała pod namiot (jak czytelnik może się domyślać, poprzednia porada powstała post factum).

To --- w połączeniu z faktem odkrycia przez Karolinę, że wczoraj rozwalił jej się bagażnik (w wyniku jazdy po wertepach nastąpiło zmęczenie materiału i dwa wsporniki ,,poszły w pizdu'') --- wprowadziło nas w nastrój ogólnej rezygnacji, który trwał dobre dwie godziny.

Wreszcie jakoś wzięliśmy się w garść i przystąpiliśmy do naprawy zniszczeń. Bagażnik udało się złożyć do kupy z pomocą --- a jakże --- taśmy izolacyjnej i opasek zaciskowych (jak ustaliliśmy po dłuższej chwili: po śląsku --- trytytek, po małopolsku --- zipów). Niestety, z mokrymi matami i śpiworami nie możemy chwilowo zrobić czegokolwiek konstruktywnego, bo na polu mży, a w budynkach nie ma ogrzewania.

Holendrzy po raz kolejny nas wyprzedzili --- oni wyruszyli o 9:30, a my o 12:00. Cicha rywalizacja wciąż trwa…

Po wczorajszym dniu wiedzieliśmy już, że pierwsze 13 km drogi (do skrzyżowania) to będzie poezja. Szkoda, że ta autostrada kończyła się niewiele dalej. Dalej, aż do Herðubreiðarlindir, wytrzęsła nam tyłki swoista ,,tarka'' (wiecie, takie jakby żwirowe garby w regularnych odstępach paru centymetrów). Właśnie ta fatalna nawierzchnia spowodowała straty w dwóch kolejnych bagażnikach: ja zgubiłem nakrętkę od wspornika, a Put --- śrubkę (również od wspornika). Dobrze, że mieliśmy po jednej nadprogramowej, bo inaczej bylibyśmy w przysłowiowej ,,czarnej dupie''…

\hint{Zadbaj, by w rowerze mieć standardowe śruby 5 mm z łebkiem na imbus oraz nakrętki samozaciskowe (nie odkręcą się same). Weź też ze sobą na zapas po parę sztuk zarówno śrub jak i nakrętek.}

\img{./photos/x-s-2014-08-05_20-57-16__190.jpg}{rasper}{Przed nami tarka…}
\img{./photos/x-s-2014-08-05_20-57-32__192.jpg}{askja_void}{…a wokół nas --- jak zwykle --- nic.}

Nie tylko droga, ale też i samo Herðubreiðarlindir zostało zalane --- w szczególności na drodze, na wjeździe i wyjeździe, pojawiły się dwa dodatkowe ,,brody'' do przekroczenia. Te niespodziewane przeszkody terenowe sprawiły, że nie w głowie było nam brać dalszej drogi ,,z marszu'' --- zapragnęliśmy najpierw zjeść godziwy obiad. Udałem się zatem do pani dozorczyni spytać się o dalszą drogę i o to, \emph{,,w którym miejscu na kempingu mieści się kuchnia''}. W odpowiedzi usłyszałem między innymi, że \emph{,,skorzystanie z kuchni kosztuje 500 kr.}'' No ale jak to, pytam, za sam fakt posiedzenia w ciepłej kuchni (mówiłem wyraźnie: mamy własne palniki, naczynia)?! Właśnie tak! Skwitowałem to, że spytam się reszty, co oni na to. A reszta mówi: \emph{Trudno. Zapłacimy. Byleby obiad był…''}. Wracam więc do dozorczyni, wręczam banknot 1000 kr i czekam na resztę. A pani mówi: ,,Jeszcze 1000 kr''. Oniemiałem. No bo jak to, będą nas kasować za każdą osobę? Wyjąłem zatem z ręki dozorczyni wręczony uprzednio banknot i wygłosiłem oświadczenie o tym, jak to polskich studentów nie stać na płacenie 2000 kr za luksus zrobienia sobie obiadu. Wtedy dozorczyni stwierdziła, że w porządku --- 1000 kr za całą grupę starczy. Przystałem na jej propozycję, lecz niesmak pozostał…

Po obiedzie --- całkiem przypadkiem --- załapaliśmy się na autobus do Mývatn. Niby wiedzieliśmy, że takowy kursuje, ale jakoś pomyliły nam się godziny odjazdu z Herðubreiðarlindir (sądziliśmy, że przyjedzie dopiero za godzinę). Do środka zmieściły się tylko dziewczyny (z rowerami i swoimi sakwami) oraz część moich i Puta bagaży. W środku siedzieli znajomi z Askji: Anglicy ,,od nerek'' (ich wyprawa miała charakter charytatywny, zbierali bodaj na przeszczep nerki dla kogoś) i Pan Rusek. \emph{,,Leszcze!''} --- pomyśleliśmy. Plan był taki, że autobus przewiezie dziewczyny przez najtrudniejszy odcinek (czyli zalaną drogę i dwa brody), a zaraz potem wysadzi. Ta operacja nie powinna kosztować ani korony! Tak też się stało, wszystko poszło zgodnie z planem: bezproblemowo i za frajer. Osobliwie zachował się starszy Anglik, który wysiadł potem na chwilę razem z dziewczynami --- nachylił się nad Karoliną i szepnął jej do ucha: \emph{,,Remember, it’s warmer inside the bus!''}

W międzyczasie ja i Put ruszyliśmy na autonogach w dalszą drogę. Wbrew apokaliptycznym wizjom roztaczanym przez ludzi z Dreki zalana droga była łatwa do obejścia --- wystarczyło odbić 20 metrów i przejść bokiem po lekko namokniętym gruncie. Podobnie było z brodami --- ich sforsowanie nie wymagało większego wysiłku niż pokonywanie tych na \road{F910}. Dziewczyny czekały na nas w sumie może ze 20 minut, lecz było to dla nich wystarczająco długo, by rozbić namiot ;-P

Ciśniemy dalej, w lekkim deszczu i chłodzie, do oporu, po drodze pokonując jeszcze jeden solidny bród. W sumie jednak końcówkę jechało się całkiem znośnie, bo ani razu nie było piasku. Droga w większości dość ubita --- może to przez padający deszcz, który ,,związał'' ten piasek?

\img{./photos/x-s-2014-08-05_20-01-34__79.jpg}{last_ford}{Ostatni bród w tej edycji interioru.}
\img{./photos/x-s-2014-08-06_10-44-32__81.jpg}{}{Jak się nie ma, co się lubi…}

Nocleg wypadł znów --- nie inaczej --- pośród pustyni, 19 km od \road{1}. Największym problem okazało się wybranie miejsca na rozbicie namiotów. Chodziło o to, żeby były one osłonięte od wiatru oraz żeby na ziemi leżało jak najmniej kamyków i kamyczków. Aż chce się powiedzieć: \emph{,,Wszędzie tyle samo! Czy to takie ważne?!''} Hm… to po prostu wychodzi z nas zmęczenie --- 77 km po wertepach, w ciężkich warunkach robi swoje.

Powiem za siebie: Wieczorem byłem tak wytyrany, że w 5 minut po rozłożeniu namiotu i przebraniu się --- zasnąłem. Nie udało mi się wytrwać do rozpoczęcia fazy przygotowania obiado-kolacji, miałem za to plan by zaraz z rana pocisnąć do Mývatn i dokonać solidnych zakupów w tamtejszym Samkaupie. Problem polegał na tym, że na obiado-kolację miał być kuskus, który to ja posiadałem w swoich zapasach. Put wołał mnie później z namiotu dziewczyn (podobno głośno) --- nie dawałem znaku życia, szukał torebek z kaszą --- lecz nie znalazł. Reszta nie doczekała się więc kuskusu i ostatecznie towarzystwo dojadało dwoma chlebami z mielonką.

	\chapter*{06.08.}

Mój poranek wyglądał następująco: 6:15 --- pobudka, szybka pasza i zwijanie namiotu w lekkim deszczu; 7:15 --- wyjazd, godzina jazdy po \road{F88} (przy świetnych warunkach drogowych) i potem gonienie resztką sił do Reykjahliðu na zakupy w Samkaupie. Cel został osiągnięty o 10:00. Jedyna trudność podróży ,,tam'' to podjazd o nachyleniu 10\%, długi na około 1~km, na dosłownie ostatnich kilometrach przed miasteczkiem.

Po trzech dniach spędzonych w interiorze, przekroczywszy próg sklepu poczułem się wprost przytłoczony bogactwem dóbr spożywczych, które były na wyciągnięcie ręki. Ponieważ jednak żołądek pilnie, ale to bardzo pilnie, domagał się pokarmu, więc na pierwszy ogień poszedł chleb i ,,szoko''\footnote{Tani krem orzechowy.}. Dopiero gdy zjadłem pół bochenka chleba z połową kubka kremu czekoladowego, nabrałem sił niezbędnych do zrobienia głównych zakupów. W międzyczasie postanowiłem jeszcze trochę podsuszyć przemoczone ubrania --- ściągnąłem buty turystyczne i starałem się ustawić je tak, by stały jak najbliżej wentylatora od lodówki. Tu spotkała mnie miła niespodzianka, bo gdy tylko pani ekspedientka zobaczyła moje zabiegi, ulitowała się nad biednym rowerzystą i zaprosiła do kotłowni. A tam mogłem nie tylko suszyć rzeczy, ale też podładować komórkę i w spokoju zjeść. Po prostu tyle wygrać! Właśnie miałem ostatecznie już opuszczać sklep, gdy natknąłem się na owych Anglików z Askji. Widząc mnie z samego rana, dość rześkiego, wykazywali już więcej respektu dla osiągnięć naszej ekipy!

W drodze powrotnej przekąpałem się w wypływającym z Krafli cieplutkim strumieniu, który przepływał przez łąkę raptem parę metrów od \road{1}. Po prostu po trzech dniach bez prysznica, kiedy ręka lepi się do wszystkiego, wszelkie obiekcje w stylu ,,co sobie ludzie pomyślą?'' przestają się liczyć --- no zwyczajnie trzeba się porządnie wyszorować! Islandczyków widok mnie kąpiącego się ,,w stroju Adama'' przesadnie nie ruszał, za to przejeżdżający Niemcy byli tym wręcz wyraźnie rozbawieni i nawet przyjaźnie machali! (Hm\textellipsis może przyjechali z NRD-ówka, gdzie naturyzm był oznaką sprzeciwu wobec władzy?)

Zebraliśmy się z powrotem do kupy na niewielkim kempingu w Grimmstaðir, tuż obok\textellipsis namiotu Holendrów! O tak, oni rozplanowali noclegi zdecydowanie lepiej niż my: w Brú, potem w Dreki i wreszcie we wspomnianym Grimmstaðir (już koło \road{1}). Dzięki temu ani razu noc nie zastała ich w pośrodku niczego\textellipsis. Niestety, samych Holendrów nie zastaliśmy --- pewnie pojechali zwiedzać okolicę.

Stojący nieopodal niewielki sanitariat umożliwiał skorzystanie z toalety jak ludzie (i za darmo, bo to było bardziej pole biwakowe ,,przy chałupie''), a że słoneczko wyszło zza chmur, to pojawiła się też kusząca możliwość zrobienia prania i od razu wysuszenia go. Proces ,,konserwacji sprzętu i siebie'' trwał dobre dwie godziny i polegał między innymi na zjadaniu różnych dobrych rzeczy, świeżo przywiezionych ze sklepu. Dzięki temu potem wszyscy w dobrych nastrojach i z wysokim morale ruszamy w stronę Dettifossu. Tuż przed odjazdem zostawiliśmy Holendrom małe słodkie pozdrowienia ,,od ekipy z Polski'' --- dwie duże żelki wraz z karteczką, wszystko zapakowane w woreczek i przyklejone do ich plastikowego super-stołeczka podróżnego.

Droga do Dettifossu to większy dramat niż interior --- cały czas ,,tarka'', liczne dziury i sporo podjazdów. To jakiś cud, że nasze bagażniki jeszcze żyją! W połowie dystansu do wodospadów spotykamy naszych Holendrów --- ot, taki miły akcent.

\img{./photos/x-s-2014-08-06_21-22-05__198.jpg}{dettifoss_and_cookies}{Cookies'y --- doskonale uświetniają osiągnięcie Dettifossu}

Na parkingu przy Dettifossie jakiś Kanadyjczyk stojący koło dużego jeepa zagadnął nas: \emph{,,Ej, nie macie czasem pożyczyć pompki?''} Mówię, że owszem mamy i być może nawet napompuje nią koło do 0,1~atmosfery (bardzo starałem się, by zabrzmiało to ironicznie). O on na to: \emph{,,O! Super!''} No i faktycznie, pożyczył ode mnie tę pompkę. Siedział potem przez pół godziny pompując na zmianę raz prawą raz lewą ręką aż\textellipsis napompował! W sumie potem Put wytłumaczył mi, że oponę nabija się do około 2~atmosfer (znacznie mniej niż w rowerze), więc problemem nie jest ciśnienie tylko objętość powietrza, którą należy wtłoczyć. Niemniej byłem pod wrażeniem wytrwałości tego człowieka. Ale że żaden z licznych turystów-samochodziarzy na parkingu mu nie pożyczył pompki?!

Prognozując dalszą trasę, po raz kolejny i ja i Put popełniamy ten sam błąd, przykładając polską miarę do islandzkich warunków. Otóż droga wzdłuż polskich rzek pnie się spokojnie dnem doliny. Tu natomiast rzeka płynie sobie głębokim kanionem, a droga idzie dokładnie przez sam środek okolicznych szczytów. Tak! Niby znajdujemy się 25~km od morza, a te pagórki mają po 300~metrów wysokości!

Szczęśliwie złożyło się, że od Detiffosu do Ásbyrgi nie ma już tarki. Tyle, że miejscami drogę pokrywa solidna warstwa drobnego szutru, co na zjazdach mocno daje się we znaki --- rower zwyczajnie ,,pływa''.

\img{./photos/x-s-2014-08-06_23-46-58__308.jpg}{gravel}{Nagroda za wytrwałość --- szkoda, że droga szutrowa.}

Początkowo chcemy zanocować na kempingu na końcu doliny Ásbyrgi --- wydaje nam się, że tam będzie naprawdę urokliwie. Błąd! Kemping ten nie istnieje, niemiecka mapa znów nas oszukała! Wróciliśmy się więc na kemping u wylotu doliny i o 23:00 zabraliśmy się za rozbijanie namiotów, a potem --- za przygotowanie i pałaszowanie ciepłej kolacji (kuskusu z warzywami z puszki, poprawionego kromkami z szoko)!

Ten kemping jest bodaj najdroższym dotychczas napotkanym --- opłata za osobę wynosi~1400 kr, lecz za prysznic należy zapłacić dodatkowe 500~kr. Nie ma ani kuchni ani \emph{common roomu}, dobrze że jest chociaż dość sprawnie działająca suszarnia (coś w rodzaju szafy, w podłodze której umieszczono dmuchawy zimnego powietrza --- na rano pranie suche). Słowem --- to miejsce przeznaczone jest typowo dla turystów w kamperach (podobnie jak \href{http://www.vatnajokulsthjodgardur.is/english/plan-your-visit/camping/}{pozostałe kempingi} na terenie parku narodowego Vatnajökull).

Z całej naszej czwórki tylko Kasia wskoczyła pod prysznic, stwierdzając: \emph{,,Po interiorze mi się należy!''}. Karolina stwierdziła, że już się myła wczoraj  w Dreki, ja też dałem sobie spokój --- w końcu rano się kąpałem --- za to Put dokonywał ekwilibrystyki nad umywalką w ubikacji dla niepełnosprawnych \wink

\hint{Toaleta dla niepełnosprawnych to wymarzone miejsce dla rowerzysty --- ma i muszlę i umywalkę oraz jest zamykana, a co za tym idzie można się spokojnie podmyć (a czasem też podsuszyć ubrania pod suszarką do rąk\textellipsis --- bywało, że właśnie w toalecie dla niepełnosprawnych znajdował się jedyny działający egzemplarz).}

	\chapter*{07.08.}

Z samego rana poszliśmy na spacer na otaczające dolinę klify - główną \href{http://www.visithusavik.com/attractions/asbyrgi-canyon/}{atrakcję} okolicy. Żadnych trudności terenowych, niewielki dystans, pogoda (jeszcze) w miarę stabilna - ot… taka przechadzka na rozprostowanie nóg.

Na śniadanie po raz pierwszy zjedliśmy (łącznie, na grupę): 1 kg paszy zalanej 1 l mleka i 1 l surmjolka - czyli średnio po 2000 kalorii na osobę:) (Należy zaznaczyć, że zaczynaliśmy od ½ tej porcji i ledwie taką ilość przejadaliśmy - teraz to tak akurat na styk.)

\img{./photos/x-s-2014-08-07_10-26-01__82.jpg}{asbyrgi_camping}{Nocleg u stup klifów.}
\img{./photos/x-s-2014-08-07_15-19-08__209.jpg}{asbyrgi_breaking_the_law}{Znów łamiemy prawo :(}

Z Ásbyrgi wyruszamy przy wietrze w plecy - szok i niedowierzanie. Radość trwa krótko, bo po 25 km, gdy tylko zatrzymaliśmy się na posiłek, zaczęło padać :(

Coś na rozluźnienie. Scenka wydarzyła się na koniec solidnego podjazdu. Ledwo zipiemy, pot się leje, wieje nam w mordę.

%% TODO: dialog
ja: No i co tak ludzie nam zdjęcia robią? Nawet się nie uśmiechną! Dobrego słowa nie powiedzą! Tylko ten znudzony wyraz twarzy…
Put: Eee.. dobrze że robią…
ja: No to niech chociaż jakieś ciastko w zamian dadzą!
Put: Czyli chciałbyś być taką dziwką rowerową? Sprzedawałbyś swoje podjazdy za ciastka?

\section*{Húsavík}

Specjalnie zajechaliśmy odpowiednio wcześniej do Húsavíku, by Put i Kasia mogli załapać się na “whale watching”. Mimo intensywnej agitacji Karolina i ja nie zdecydowaliśmy się na dołączenie - Karolina widziała już z bliska delfiny, gdy pływała żaglówką po Morzu Północnym, a mnie do rejsu nie zachęcała pogoda (zimno, pochmurno, wietrznie). Co więcej, żadnej gwarancji zobaczenia wieloryba nie ma - pani w informacji powiedziała, że dwa poranne rejsy nic nie widziały, kolejny był już ciut lepszy ale też bez szału. “Natomiast na pewno mogę zagwarantować, że będzie chłodno!” - dodała z uśmiechem. (Dla ustalenia uwagi, rejs trwa około 3 godziny).

\img{./photos/x-s-2014-08-07_19-54-58__84.jpg}{whalewatching_clothing}{Kasia w pełnym rynsztunku}

Ja i Karolina pożegnaliśmy więc “wielorybników” i pojechaliśmy zakładać obozowisko. Kemping kosztował 1200 kr / os. + 500 kr za prysznic; był wyposażony w suszarnię z prawdziwego zdarzenia i małą kuchnię turystyczną z paroma palnikami i paroma stołkami. Gdy już się rozgościliśmy i przebraliśmy, ruszyliśmy na miasto “na rybę” - to znaczy głównie chodziło nam o to, by po ponad 2 tygodniach tułaczki wreszcie zjeść coś normalnego. Po obejściu wszystkich (czterech) portowych knajp i wnikliwym porównaniu menu oraz cen wybraliśmy restaurację \href{https://www.facebook.com/naustid}{Naustið}. Musieliśmy dobrą chwilę czekać na wolny stolik - taki był ruch - lecz przynajmniej wreszcie czekaliśmy na coś w cieple! Zamówiliśmy zupę rybną oraz “2x szaszłyk rybny na 1 talrzu” - coś pysznego! W zupie pływała duża ilość tłuściutkich krewetek, a całość była dobrze doprawiona. Podobnie szaszłyczane rybie mięso było bardzo delikatne, a w ramach przybrania występowały: małe ziemniaczki, sałata, feta… Ach, to było jak podróż do innego, lepszego świata - za oknem siąpi i hula wiatr, a my grzejemy się siedząc przy stoliku tuż obok kaloryfera i zajadamy się takimi smakołykami… I tylko żal było się tak szybko zmywać (bo ani się oglądnęliśmy, a zrobiła się 22:00 - godzina zamykania lokalu). W drodze powrotnej zahaczyliśmy jeszcze o grill-bar przy stacji benzynowej (nie, bynajmniej nie po to, by psuć sobie smak, tylko żeby rozmienić pieniądze) i… to było coś koszmarnego. No normalnie traumatyczne doświadczenie - wejście do takiego czegoś po wizycie w przytulnej knajpce? Gdzie przy świetle świetlówek zmarnowani ludzie męczą hamburgery i frytki? Brr…

Ok, jednak trochę zepsuliśmy sobie smak szaszłyków i zupy rybnej, bo po droga powrotna na kemping tak nas zmęczyła, że konieczne było dojedzenie chlebem z szoko…

Wielorybnicy wrócili o 23:30, nie zobaczywszy ani jednej taakiej ryby. Na pocieszenie wrzucili za to na fb swoje sweet-selfie-focie w gustownych sztormiakach zawodowego rybaka morskiego ;)

PS. Dziś na podjazdach Kasia pokazała pełną klasę i prawdziwego ducha walki - gratulacje!

	\chapter*{08.08.}

Od rana pada. A może tylko mży? Albo kropi? Ostatatnimi czasy dyskusja o poprawnej terminologii w odniesieniu do stopnia nasilenia opadów atmosferycznych pochłania nas bez reszty --- lansowane przez większość uszeregowanie według stopnia intensywności to (począwszy od najlżejszych opadów): mży, kropi, pada. Nasze spory wynikały z niechęci do sięgnięcia po \href{http://sjp.pwn.pl/}{SJP PWN}\footnote{Słownik Języka Polskiego wydany przez Polskie Wydawnictwo Naukowe.}. Gdybyśmy jednak do niego zajrzeli, to szybko dowiedzielibyśmy się, że ,,mżyć'' to \emph{,,o deszczu: padać gęsto drobniutkimi kropelkami''}, ,,kropić'' to \emph{,,o deszczu: padać [po prostu] drobnymi kroplami''}, a ,,padać'' oznacza zwyczajnie \emph{,,o zjawiskach atmosferycznych: spadać na ziemię w postaci wody, śniegu lub lodu''}. Tak czy inaczej opad wody ma --- z różną intensywnością --- trwać przez cały dzień.

Pierwszym punktem programu na dziś była sprzedaż biletów na oglądanie wielorybów. Put i Kasia dokładali starań, by wcisnąć je napotkanym w centrum Húsavíku turystom, lecz szło im to opornie. Może to kwestia różnic kulturowych? Weźmy na przykład takiego Hiszpana --- już chce kupić, ale jeszcze coś nie daje mu spokoju. Idzie więc do kasy i pyta się \emph{,,Przepraszam, czy to będzie w porządku, jeśli kupię niewykorzystane bilety od tej dwójki, co stoi przed biurem?''}. No i co mu ma kasjerka odpowiedzieć?! \emph{,,Tak, oczywiście --- to nasi sprawdzeni partnerzy handlowi.''}? Dobrze, że wkrótce przyplątali się bracia Słowacy, którzy doskonale orientowali się w istocie trasakcji. Po krótkim targu (poprosili o zniżkę 1000~kr) stali się szczęśliwymi posiadaczami naszych biletów.

\hint{Bilety na oglądanie wielorybów zostają ,,skasowane'' w systemie dopiero wtedy, gdy dany rejs faktycznie zobaczy wieloryby. Teoretycznie można próbować oglądać do skutku\textellipsis}

Jeszcze w Húsavíku rozdzieliliśmy się, gdyż moją idée fixe była chęć odwiedzenia \href{http://www.visithusavik.com/attractions/the-turf-house-museum/}{muzeum w Grenjaðarstaður}. Od dwóch tygodni głowiłem się ,,jak właściwie ludzie funkcjonowali dawniej (przed erą elektryczności i geotermii) w islandzkim klimacie?'', a tu tuż pod nosem znalazłem ekspozycję poświęconą właśnie temu zagadnieniu! Inni nie podzielali tej fascynacji (bądź też pęd do wiedzy przegrał z deszczem), stąd eskapada była samotna.

W Grenjaðarstaður moje serce od razu podbił kustusz muzeum, który widać że był kustoszem z powołania: rzucił garścią ciekawostek, wyczerpująco odpowiadał na różne moje pytania, pozwolił się wysuszyć\textellipsis Sama ekspozycja niby typowa --- a to stoją sobie rzędem meble, a to leżą na stole jedne obok drugich narzędzia stolarskie --- lecz jej sercem były porządnie zrobione książeczki-przewodniki, w których wytłumaczono przeznaczenie poszczególnych sprzętów (i to używając poprawnej angielszczyzny!). Po zwiedzeniu tego domu-muzeum mogę powiedzieć krótko: życie Islandczyków diametralnie różniło się od tego, co znamy z Europy Środkowo-Wschodniej (czy ogólnie Europy kontynentalnej), dlatego gorąco polecam każdemu zawitanie do Grenjaðarstaður i za marne 600~kr poznanie odrobiny islandzkiej historii. (PS. Dobrej jakości gorąca herbata w cenie biletu!)

Kawałek za skrzyżowaniem \road{87} z \road{853} kończy się asfalt, a oprócz znaku ,,malbik endar'' dodatkowo stoi ,,Blindhæð'' i tabliczka 16~km. To tak żeby w zawoalowany sposób powiedzieć rowerzyście: ,,Słuchaj, przez najbliższe 16~km czeka cię jeżdżenie góra-dół --- wjazd na każdy możliwy pagórek!'' Pieprzony rollercoaster\textellipsis Pokonywanie tego w deszczu --- masakra.

\img{./photos/x-s-2014-08-08_15-56-24__85.jpg}{rollercoaster}{Deszcz, dziury, pod górę. Swojsko.}

Trochę zabalowałem w muzeum i potem całą dalszą drogę gnałem na złamanie karku --- na tyle, na ile pozwalała marna jakość nawierzchni i fatalna pogoda. Z resztą grupy zrównałem się dosłownie na przedmieściach Reykjahliðu. Pierwszym miejscem, do którego skierowaliśmy się po przybyciu do miasteczka, był sklep spożywczy. Tam podjęliśmy ważką \mbox{(damsko-)}męską decyzję: \emph{,,Pier******, dziś nie zwiedzamy, nie jedziemy dalej --- idziemy na najbliższy kemping z kuchnią i common roomem i tam czekamy jutra. Ewentualnie pomoczymy jeszcze tyłki w basenie, ale tylko i wyłącznie jeśli będzie z gorącą wodą!}
%TODO: może jakoś ocenzurować <pierdolimy>?

Pierwszy kemping, na który się udaliśmy, to ten naprzeciw Samkaupa. Na nim właśnie rozbili się Holendrzy. Nie wzbudził naszej sympatii --- gdy zobaczyliśmy, że za kuchnio-namiot (gdzie zimno jak w psiarni) \linebreak i prysznic musielibyśmy płacić po 1500~kr od łebka, to postanowiliśmy zrobić ten ostatni wysiłek i poszukać jeszcze czegoś lepszego. Cóż\textellipsis osta\-tecznie trafiliśmy wcale nie lepiej --- na kemping Hlíð.

Na kempingu Hlíð za kuchnię i salę ,,kominkową'' służy konstrukcja wykonana z plastiku falistego rozpostartego na drewnianym stelażu, przykryta dachem od namiotu piwnego --- całość ustawiona jest po prostu na asfaltowym parkingu. Środkiem tejże ,,kuchni'' płynie rzeka, a pod ,,sufitem'' gwiżdże wiatr. Nie mamy już jednak ani siły, ani motywacji, by jechać gdziekolwiek indziej --- ociekamy wodą, a nasz ubiór można wykręcać. No, może z wyjątkiem Puta, którego cienka kurteczka i spodnie przeciwdeszczowe mimo paru sezonów wciąż jeszcze w miarę dobrze zatrzymują wodę (tak że on ma tylko od środka wilgoć, a nie kałuże). Do listy zażaleń możemy jeszcze dodać brak suszarni (znaczy się niby są suszarki bębnowe, ale to nie dla nas), brak grzejników (żadnych!) oraz niedziałające suszarki do rąk\textellipsis

Na obiad Kasia zaszalała i przyrządziła spaghetti z twarożkiem i szpinakiem --- świetna odmiana po konserwach mięsnych czy tunfiskach z makaronem bądź ryżem i warzywami z puszki! Ten lekkostrawny, zdrowy posiłek kończymy dopychaniem się solidną porcją chleba z szynką oraz dżemem. Dopiero pojedzeni w ten sposób zabieramy się za rozbijanie namiotów.

% {pogoda taka, że przypomniał mi się serial ,,Pacyfik'' (który odcinek? + link :> | 1x4, 14:40 albo 22:35 albo 28:40-29:30)}

\medskip

Podczas obiadu padły ważkie słowa:

\vspace{-1em}

\epigraph{,,Jeśli za rok powiem, że chcę jechać rowerem gdzieś na północ od Krakowa\textellipsis nie, od Warszawy\textellipsis albo nie --- po prostu na północ od Polski, to przypomnijcie mi, że jednak nie chcę!''}{--- \textup{Put}}

\vspace{-2em}

\epigraph{,,Islandia? Chętnie jeszcze raz, ale na pewno nie rowerem!''}{--- \textup{Kasia}}

\vspace{-1em}

\noindent Oczywiście i ja i Karolina mamy bardzo zbliżony pogląd na te kwestie.

\img{./photos/x-s-2014-08-08_22-33-14__86.jpg}{tent_party}{Miało być wyjście na basen, wyszło --- ,,piżama'' party.}

	\chapter*{09.08. --- Wokół Myvatn}

Rano budzi nas słońce --- tak nas zaskoczyło, że aż nie wiemy co z tym fantem zrobić! ;-) Masowo suszymy ubrania z wczoraj i konserujemy przerdzewiałe łańcuchy.

Zawitaliśmy do informacji turystycznej koło Samkaupa, by zasięgnąć języka ,,co warto zobaczyć''. Część atrakcji odpadła w przedbiegach: ze względu na aktywność tektoniczną temperatura wody w grocie Grjótagjá (kręcono w niej fragment jednego z epizodów Gry o Tron) znów podniosła się do 60 °C (kąpiel grozi nie tyle ,,ugotowaniem jajek'', co poważnym poparzeniem bądź wręcz śmiercią), Krafla znajduje się za daleko (i za wysoko…), rzeczy podobne do Hverarönd już widzieliśmy… Ostatecznie postanawiamy zwiedzić tylko to, co jest łatwo osiągalne z drogi naokoło jeziora.

Pierwszy punkt programu: formacje skalne w Dimmuborgir. Jest przyzwoicie, choć bez szału. W sumie natrafiliśmy na dwa ładne miejsca --- ,,oczko'' w skale i tzw. Kościół. Jak to dobrze, że za wstęp nie trzeba płacić, bo wtedy byłoby nam żal wydanych pieniędzy ;)

\img{./photos/x-s-2014-08-09_12-33-43__88.jpg}{dimmuborgir}{Dimmuborgir}

Kolejną atrakcję stanowi rezerwat Höfði --- ze znajdującego się w jego sercu pagórka roztaczają się piękne widoki na wysepki na Mývatn. Choć w sumie… dla Islandczyków największą frajdą musi być możliwość oglądania takiego skupiska prawdziwych, wysokich drzew!

Ostatnie miejsce, które zwiedzamy, to pseudo-kratery w Skútustaðir. Moim skromnym zdaniem o wiele, wiele lepiej prezentują się one na zdjęciach z lotu ptaka, no ale skoro już tu zajechaliśmy… Podczas tego spaceru spotkaliśmy dwie intrygujące osoby. Pierwszą była starsza Amerykanka (prawdopodobnie z prowincji), która słysząc naszą mowę zawołała do męża: \emph{,,Bob, look! Polish people!''} Bob ani na chwilę nie przerwał robienia zdjęć, nie skwitował tego choćby jednym słowem, więc Amerykanka zaczęła tyradę (do siebie samej) o tym, jak to chciałaby już być w domu i że lot na Islandię taki długi… Drugą osobą był Polak z Mazur mieszkający od 7 lat nad Mývatn. Rozbroiło nas jego stwierdzenie: \emph{,,Mieszkam tu już tyle lat, a codziennie zauważam coś nowego!''} Co takiego można tu nowego zobaczyć --- nowy kamyk przy drodze?

\img{./photos/x-s-2014-08-09_16-03-32__215.jpg}{myvatn}{Myvatn}
\img{./photos/x-s-2014-08-09_16-10-39__318.jpg}{forest}{Na spacerze w bodaj jedynym (!) lesie na Islandii.}

Karolina ma problemy z tylną przerzutką --- właściwie nie może wrzucić normalnie 6-8, jedyne co pozostaje to ,,ręczna skrzynia biegów'', czyli ciągnięcie za linkę na rurze od ramy. Interesujące, że objawy nasiliły się, gdy po interiorze Karolina usunęła brud z przerzutki. Wcześniej, mimo zapiaszczenia, działały względnie dobrze. Aż ciśnie się na usta znana mądrość ludowa: \emph{Częste mycie skraca życie!}

Tuż za pierwszym podjazdem, nad jeziorem Másvatn --- zdrada! Dopadła nas krótka, lecz tak intensywna ulewa, że nim zdążyliśmy się oporządzić --- już mamy wszystko mokre: buty, spodnie, polary… Oczywiście jak zwykle Put przezornie założył co trzeba na samiutkim początku, gdy tylko poczuł pierwsze krople. Nasza trójka spróbowała ucieczki przed deszczem na drugi brzeg jeziora, co okazało się  fatalne w skutkach. Zrezygnowani, ponownie puściliśmy w ruch płachtę i --- siadłszy w przydrożnym rowie --- zaczęliśmy konsumować ciastka w oczekiwaniu na dalszy rozwój wypadków (pogodowych). Tym razem prognoza pogody się nie sprawdziła --- miało padać dopiero pod wieczór, a tu 16:00 i już woda!

Na drodze do Akureyri naliczyliśmy łącznie trzy podjazdy, wszystkie długie, równo nachylone, asfaltowe… Jazda po czymś takim to sama przyjemność, bo wiadomo że na szczycie każdego podjazdu czeka nagroda w postaci pięknego, równie długiego zjazdu! Podobnie jazda doliną koło Goðafossu nie wymęczyła nas ani trochę, gdyż droga biegła w zasadzie po równym.

Lecz cóż może być większą nagrodą dla utrudzonych rowerzystów niż możliwość podziwiania pięknego zachodu słońca nad fiordem Eyjafjörður? Załapaliśmy się na niego dosłownie w ostatniej chwili --- gdyby wspinaczka na przełęcz zajęła nam choćby 5 minut więcej, to nic byśmy nie zobaczyli! Daliśmy przykład kierowcom samochodów --- najpierw zatrzymała się niemiecka furgonetka, potem jacyś Francuzi, dalej dwa auta z Polski… Wszyscy z opuszczonymi szybami i wystawionymi aparatami fotograficznymi, uwieczniają opuszczającą się za chmury i wierzchołki gór pomarańczową kulę.

\img{./photos/x-s-2014-08-09_19-12-20__89.jpg}{gothafoss_dinner}{Obiad przy Goðafossie}
\img{./photos/x-s-2014-08-10_00-02-44__327.jpg}{eyjafjorthur_sunset}{Zachód słońca nad Eyjafjörður}
%TODO: zadbać, by te zdjęcia faktycznie pojawiły się pod tym akapitem

Niezbyt chce nam się jechać dziś do samego Akureyri, więc rozbijamy się na miedzy, jakieś 15 km od miasta. Sądząc po śladach koło strumienia nie byliśmy pierwszymi, którzy wpadli na ten pomysł :-D

Rozbijając namioty żartujemy sobie z Putem: ,,A może by rozbić namiot tak, jak Pan Rusek polecał koło źródeł --- tuż nad strumieniem, żeby rano było łatwo załatwić ,,dwójkę''? Dziewczyny mogłyby sobie tak ustawić namiot, by mieć tam wejście ,,od kuchni''…''
%TODO: wytłumaczyć, o co chodzi z <dwójką>

\hint{Żarty żartami, ale z wiadomych względów naprawdę warto zawczasu jasno ustalić podział strumienia na odcinki: toaleta,  łazienka i kuchnia.}

Ogólne spostrzeżenie po dniu dzisiejszym: stajemy się zblazowani, jeżeli chodzi o atrakcje turystyczne Islandii. Po blisko trzech tygodniach jeżdżenia po tej wyspie coraz ciężej o coś naprawdę nowego…

	\chapter*{10.08.}

Początek dnia można streścić krótko - długie spanie, późne śniadanie, późny wyjazd ;-) Wiadomo, taki typowy niedzielny poranek, wszystko się odbywa dość sennie… Dlatego aż przecieraliśmy oczy ze zdumienia, gdy o 11:00 zobaczyliśmy, jak “naszą” miedzą sunie w naszą stronę olbrzymi traktor. Myślimy sobie - pewnie zaraz skręci gdzieś w bok. A on nie - jedzie dalej prosto na nas. Gdy maszyna znalazła się już w odlełości ledwie 100 m od nas, rozpoczęło się gorączkowe pakowanie rzeczy i zwijanie namiotów. Bo a nuż gospodarz będzie chciał nas pogodnić widłami… albo traktorem? Nic z tych rzeczy! Kierowcą traktora okazał się sympatyczny Pan Islandczyk, który sprawiał wrażenie rozbawionego całą sytuacją. Płynną angielszczyzną zaczął podpytywać skąd jesteśmy i jak nam się udała wyprawa, spokojnie poczekał aż przestawimy namioty i w ogóle tylko brakowało, by zaprosił nas na obiad do siebie.

Po tych porannych przygodach ruszyliśmy sprawnie do Akureyri, by jak najwcześniej zajechać do świątyni konsumpcji i mekki wszystkich islandzkich turystów - do Bonusa!

\hint{Plan Akuyreri można dostać na stacjach benzynowych. Bonus znajduje się przy \road{1} na północy miasta, nie trzeba krążyć.}

W Bonusie zakupiliśmy liczne dobra luksusowe (salami, czekoladę, ananasy, masło, jogurty), z części których przyrządziliśmy następnie extra paszę de luxe w ilości “do porzygu”. Głodni byliśmy, to nikt sobie nie krzywdował, w efekcie czego chwilę później ledwo-ledwo wtaczamy się na pagórki przy wylocie z miasta…

\img{./photos/x-s-2014-08-10_13-54-57__90.jpg}{feeder_de_luxe}{Pasza de luxe - na jogurcie, z bananem…}

Dzisiejsza droga jak ze snu: ładne, równe doliny, długie podjazdy i konkretne zjazdy - żadnego jeżdżenia po pagórkach! Szkoda tylko, że panuje aż tak duże natężenie ruchu - zupełnie, jak na polskich drogach wojewódzkich. (Aha, lepiej pokonywać tą trasę od strony Akureyri, bo wtedy na zjazdach nie ma serpentyn!) Dobitnym dowodem tego, że droga nie była trudna niech będzie fakt, że w połowie zjazdu zatrzymaliśmy się na kanapki (z masłem i salami - mniam!). Warto zaznaczyć, że po raz pierwszy (chyba?) tak pojedliśmy w południe koło Bonusa, że na dobrą sprawę kanapki i ciastka jedliśmy w drodze bardziej z łakomstwa niż z autentycznej potrzeby zasypania organizmu kolejną porcją kalorii!

Na zjeździe z przełęczy po raz trzeci widzę tych samych wrocławian (dwa samochody na blachach DW) - wcześniej mijali mnie na \road{85} i przy zjeździe z zachodem słońca. Ot, uroki jeżdżenia po Islandii, gdzie właściwie wszyscy poruszają się po tej samej drodze i oglądają te same miejsca! Z pozostałymi znajomymi ciężko: Holendrów brak, Pani Polki (raczej) już nie spotkamy (bo nie zamierzała robić \road{F35}), a Pan Rusek objeżdża Islandię w przeciwnym kierunku.

Zmiany pogody w górskich regionach Islandii zaskakują. Z doliny gdzie niebo zasnute było szczelnie gęstymi chmurami zjeżdżamy prosto do kolejnej, gdzie już praży ostre słońce. Fakt, że deszczu niet i wiatr nie-w-twarz upoważnia nas do określenia całości mianem “naprawdę pięknej islandzkiej pogody”! Lecz czym byłaby podróż bez przygód? Ostatnie 15 km potwornie męczymy, bo jednak wiatr odkręcił. Wieje tak mocno, że po drodze robimy odpoczynek - autentycznie musimy nabrać sił do dalszej walki z wiatrem!

W Varmahlíð mieliśmy do wyboru dwa kempingi - jeden tuż przy wjeździe, drugi ponad kilometr dalej. Choć generalnie nogi nam już z tyłka wychodziły, zdobyliśmy się na ten ostatni wysiłek, a to z bardzo prostej przyczyny - średnią mieliśmy ochotę, by przy którymś podmuchu wiatru nasze namioty odleciały. Tak, pod tym względem Kempingu Tjödum i Skagarfirði był zdecydowanie faworytem - zaciszny, położony wśród drzew, zrobiony na wzrór europejski (czyli:boksy na namioty/samochody wytyczone szpalerami drzew). Cena przyzwoita: 1100 kr/os. + 100 kr namiot + 250 kr prysznic, a “w cenie” ciepła łazienka (która od biedy może robić za kuchnię) oraz grzejące na maks liczne kaloryfery - wreszcie robiąc pranie można mieć pewność, że na rano będzie ono suche.

Po raz pierwszy widzimy na Islandii księżyc! Nie, by najmniej nie dlatego, że chodzimy spać z kurami - po prostu po raz pierwszy wieczorem na niebie niemal nie ma chmur!

	\chapter*{11.08. --- Interior: drugie starcie}

\curiosity{Czy wiesz, że butelkowaną islandzką kranówę\textellipsis --- o przepraszam, wodę źródlaną --- można nabyć w licznych punktach sprzedaży detalicznej w USA? (Tekst promocyjny z etykiety \href{http://icelandspring.com/}{icelandspring.com}.)}

Ostatnio powstał swoisty standardowy zestaw tematów do posiłków: Ile kilometrów mamy do przejechania w poszczególnych dniach? Ile jeszcze noclegów i gdzie one wypadają? Ile śniadań i obiadów czeka nas przed najbliższym sklepem? Co zjemy na który posiłek (zwłaszcza na obiad, bo trzeba wybrać typ zapychacza i wkładki mięsnej)? Zgoda, to ważne sprawy i należy je omówić, lecz ilość czasu którą obecnie na to poświęcamy jest olbrzymia. Chyba zaczęliśmy to traktować jako zapełniacz czasu.

Kolejny dzień, kiedy chociaż z rana świeci słońce. Och, jak taka pogoda pozytywnie nastraja do pedałowania! Choć w sumie\textellipsis poranek taki piękny, a do przejechania tylko jakieś 70~km, więc po cóż się rwać? Wszystko toczy się iście emeryckim tempem, z leżeniem na trawie, opalaniem się oraz skakaniem na dużej dmuchanej trampolinie włącznie. Wyjeżdżamy na trasę późno jak nigdy (jeżeli nie liczyć dnia turysty), bo około 13:00, lecz wszyscy jesteśmy zadowoleni z tych miło spędzonych chwil.

\img{./photos/x-s-2014-08-11_15-13-41__330.jpg}{vermahlith_sun}{Miła odmiana od deszczu.}

Zawitaliśmy jeszcze do supermarketu, by kupić ,,chleb i mleko'', co szybko przerodziło się w szał zakupów. Gdy wreszcie udało się wszystkie puszki z owocami i paczki ciastek poutykać po sakwach, te znów są konkretnie wypchane \smile

Droga \road{733} oraz \road{F35} (na odcinku do Áfangi) to cud-miód-poezja: solidnie utwardzona (chyba jakimś środkiem wiążącym podłoże), łagodnie nachylona (począwszy od okolic elektrowni wodnej Blanda nie ma już stromych podjazdów) i\textellipsis tylko momentami trochę dziurawa (ale tak do przeżycia).

\img{./photos/x-s-2014-08-11_17-49-02__336.jpg}{into_interior}{Ponownie wkraczamy w interior}


Po odbiciu z \road{733}, tuż za mostem, zapragnęliśmy przyrządzić kolejny polowy obiad. Doświadczenia z Askji zaprocentowały --- znów konstruujemy wiatrołap. Ponieważ tym razem świeci słońce i nigdzie nam się nie spieszy, więc znalazł się czas, by Kasia przeszła pełnowymiarowy kurs otwierania konserw przy użyciu scyzoryka. Po obiedzie zgodnie stwierdzamy, że coraz ciężej uchwycić moment przejścia między uczuciem głodu, nasycenia i przejedzenia. Zazwyczaj objawia się to tym, że najpierw jemy i jemy, a po takim posiłku ledwo dajemy radę kręcić pedałami --- każdy stromszy podjazd grozi zwrotką.

\img{./photos/x-s-2014-08-11_21-47-43__230.jpg}{road}{Ech, ta droga\textellipsis}

Jazda w kompletnej ciszy to nowe, osobliwe doświadczenie. Nie mam na myśli tego, że tylko pedałujemy, nie odzywając się do siebie nawzajem. Nie, to nie tak! Samochodów brak, ludzi brak, owadów brak, flauta\textellipsis By\-wało, że gdy przystawaliśmy --- gdy milkł chrzęst nienasmarowanych łańcuchów --- to aż zaczynało dzwonić w uszach!

Późniejszym popołudniem było tak upalnie i bezwietrznie, że nawet Kasia dołączyła do grona osób podróżujących w krótkim rękawku! Taki stan nie utrzymał się jednak długo, a~dokładniej --- utrzymał się przez całe 10 minut. Zresztą gdy później Put i Karolina polowali na szczycie wzniesienia przed Áfangi na zachód słońca, podmuchy wiatru były tak mocne i lodowate, że Kasia i ja skapitulowaliśmy (innymi słowy nie dotrwaliśmy do zachodu) i~udaliśmy się do pobliskiego ,,kurortu'' w celu ogrzania zgrabiałych palców.

Áfangi nieco nas rozczarowało. Niby cena za nocleg przyzwoita, bo po 1200~kr od łebka (a prysznic i jacuzzi w cenie!), lecz mimo to szału nie ma. Zdania ,,czy zostajemy'' są podzielone, zatem decyzję podejmujemy\textellipsis rzucając monetą. Choć z rzutu wyszło, że nie zostajemy, to i tak ostatecznym argumentem było to, że jedyny fragment trawiasty zdatny do rozbicia się znajdował się\textellipsis tuż obok stajni! Odjechaliśmy więc kilometr dalej i rozbiliśmy się parę metrów od \road{F35}, na względnie płaskim i nie-kamienistym skrawku ziemi.

\img{./photos/x-s-2014-08-11_23-25-53__350.jpg}{afangi_sunset}{Uroki jazdy o zmroku.}

\pagebreak

%\img{./photos/x-s-2014-08-12_00-52-07__354.jpg}{afangi_campsite}{Wzgardziliśmy noclegiem tuż obok stajni w Áfangi.}

	\chapter*{12.08. --- Hveravellir}

Pobudka o 10:30 --- tym samym ustanowiony został nowy rekord! Dziś jednak była to nawet okoliczność korzystna, gdyż koło 8:00 całe niebo zasnuwały szczelnie chmury. A tak, gdy wyruszaliśmy o 12:30, mogliśmy już podziwiać pierwsze nieśmiałe prześwity w tym całunie.

Czekająca nas trasa to czysta rekreacja --- niecałe 40 km, z czego pierwsze 30 km po takiej nawierzchni jak dzień wcześniej. Potem mało przyjemny podjazd po kamykach i jeszcze gdzieś króciutki odcinek po kamerdolcach. Można spokojnie podziwiać krajobrazy, a te są niecodzienne. Jedziemy w końcu przez płaskowyż Kjölur, zatem jak okiem sięgnąć rozciągają się gładkie łączki i niewielkie pustynie i tylko gdzieś na horyzoncie ciemnieją zęby łańcuchów górskich oraz błyszczą się w słońcu czapy lodowców.

Zalewie po trzech godzinach docieramy do celu. Błyskawicznie przygotowujemy obiad, a zaraz potem  zaczynamy grzanie tyłków w ,,basenie'' z gorącą wodą. Moczenie trwało do wieczora --- siedzieliśmy tak długo, że nawet załapaliśmy się na wschód księżyca…

W Hveravellir po raz kolejny okazało się, że flaga pełni funkcję towarzyską. To właśnie dzięki niej poznałem Piotrka, z ,,zawodu'' słuchacza Akademii Marynarki Wojennej w Gdyni, a z zamiłowania --- obieżyświata. Rozmawialiśmy o wielu interesujących rzeczach, wymienialiśmy się licznymi praktycznymi radami odnośnie wypraw, lecz nie będę ich przytaczał w tym miejscu (zebrałem je w stosownym dziale). Wspomnę tyko o tym jak to się stało, że go tam spotkaliśmy: otóż Piotrek przemierzał stopem Islandię i akurat nocleg wypadł mu tu, w Hveravellir. Zagadnął właściciela kempingu, czy mógłby coś pomóc ,,w obejściu'' w zamian za wikt i opierunek, a ten dał mu fuchę w postaci wymiany rur od gorącej wody. Nazajutrz spytał jeszcze Piotrka, czy nie chciałby jeszcze ich zutylizować, na co ten przystał. Po dwóch dniach Piotrek otrzymał propozycję pomocy jeszcze przez tydzień, w zamian za 500 €. W ten oto sposób zdobył fundusze na pokrycie niemal całej swojej islandzkiej eskapady!

\img{./photos/x-s-2014-08-13_12-13-18__234.jpg}{hveravellir_crowded}{Hveravellir okazało się być popularnym miejscem…}
\img{./photos/x-s-2014-08-13_12-42-05__235.jpg}{hveravellir_natural}{Dwie główne atrakcje (no… może poza ,,jacuzzi'')}

Podczas wieczornej narady klamka zapadła --- zmieniamy plany i zamiast prosto do Gullfossu, jedziemy jutro całą grupą do kempingu u podnóża Kerlingarfjöll. Przez moment rozważaliśmy jeszcze taki wariant: ja i Karolina jedziemy dziś wieczorem do Kerlingarfjöll i tam śpimy, jutro z rana idziemy w góry i spotykamy się znów w komplecie na rozdrożu. Lecz skoro wszyscy przystali na opcję z Kerlingarfjöll, to nie będzie konieczne takie kombinowanie! Jedyny problematyczny aspekt to zaopatrzenie w żywność --- nie przewidzieliśmy dodatkowego dnia w interiorze, a po drodze sklepów brak. Co najwyżej można zjeść coś w restauracjach, ale to średnio nam się uśmiecha ze względu na duuużą marżę.

Dzisiejszy dzień był najluźniejszy i według mnie najpiękniejszy podczas całego naszego wyjazdu --- mam na myśli zarówno pogodę, jak i widoki oraz atrakcje na trasie. Nie, nie przygody, a prawdziwe atrakcje --- choćby widok stada stu koni pędzonych po ,,stepie''.

Dziś każde z nas jechało na dobrą sprawę ,,na własną rękę'', swoim tempem, nierzadko w odległości kilometra od wcześniejszej osoby --- ale to też jest potrzebne! Ciężko robić przez tak długi czas wszystko kolektywnie. Podejrzewam, że każdy potrzebuje od czasu do czasu odpoczynku od innych, zwłaszcza że od trzech tygodni funkcjonujemy właściwie w tym samym ,,sosie''. Oczywiście trzeba podkreślić fakt, że warunki atmosferyczne i trudność trasy na to pozwalały --- nie istniało ryzyko, że ta ostatnia osoba ,,zaginie w akcji''.

Po basenach, tak przed spaniem, Karolina i Put szarpnęli się na piwo (1000 kr za puszkę to solidny wydatek!) --- nie ma się jednak czemu dziwić, skoro byliśmy w jednym z nielicznych miejsc, w którym dało się kupić prawdziwe piwo, a nie jakieś siki…

Po raz pierwszy idziemy spać ,,na głodnego''. Nie dopychamy się na dobranoc chlebem, bo z obliczeń wyszło nam że wtedy pojutrze zabraknie nam go w trasie. Tylko Put stwierdził, że niesamowicie go przysysa i dlatego zamówił w barze coś do jedzenia i gorącą czekoladę.

	\chapter*{13.08.}

Poranek pierońsko chłodny, mimo polara i rękawiczek budzę się o 6:00 z zimna. (Zresztą pozostali też mimo wielu warstw i zawijania się w śpiwór wielokrotnie budziła się w nocy - Kasia nawet ubierała się w półśnie, z sukcesem!) Niebo zasnute jednolitą warstwą grubych chmur, ale silny północny wiatr jakby zaczynał je powoli rozganiać.

Nie chciało mi się wracać do namiotu (bo wiem, że już nie zasnę), więc wybrałem się na spacer zieloną trasą turystyczną (najkrótszą z trzech możliwości, coś 3 km), lecz przyznam że trochę się nią rozczarowałem - właściwie żadnych nowych krajobrazów, form skalnych albo roślin… Parę szczelin w ziemi, z których bucha para, ale poza tym szału ni ma. Według mnie największą atrakcją przyrodniczą Hveravellir jest - oprócz gorącego źródła - buchający gęstymi kłębami pary stożek. Zobaczenie go w zupełności wystarcza do zaliczenia programu zwiedzania.

Ciekawostka. Główny problem życia w Hveravellir? - Brak zimnej wody! Czy jesteście w stanie uwierzyć, że są w górach na Islandii miejsca, gdzie wodę trzeba chłodzić w lodówce? Serio, ze względu na podziemne gorące cieki wodne z kurka z "zimną" wodą leci coś, co ma temperaturę koło 20 stopni :-D

Dziś nawierzchnia na drodze ma standard ze dwie klasy niższy niż w poprzednich dniach - bardzo dużo żwiru (na odcinku około 10 km), miejscami kamienie wielkości pięści, dobrze że chociaż dość płasko, bo inaczej byłby dramat.

\img{./photos/x-s-2014-08-12_15-55-28__361.jpg}{on_the_road_again}{On the road again!}

Sprzęt zaczyna się sypać… Wyrwał mi się fragment “pleców” jednej z sakw - akurat razem z hakiem, na którym wisi sakwa. Coś, co od biedy mogę określić mianem “naprawa” zostało wykonane z użyciem parcianego pasa. Nie zabrałem sprzączki ani klamry, więc cuduję by jakoś związać to wszystko na tyle solidnie, by sakwa nie latała dalej na wszystkie strony.

Droga od \road{35} do Asgarður okazała się być całkiem znośna. Obawialiśmy się, że czeka nas co najmniej 4 km pchania roweru, a tu: brodu brak, chamskiego żwiru [niemal brak, tylko jeden naprawdę ostry podjazd… Właśnie ten ostatni podjazd był naprawdę mylący - wydawało się, że by dojechać do kempingu trzeba będzie wdrapywać się jeszcze drugie tyle, a tu po zjeździe na niewielkie wypłaszczenie okazuje się, że "o, to już tu!" :-)

W pewnym momencie na trasie straciłem możliwość zmieniania przedniej przerzutki - była ustawiona na 2 i nie wchodziła ani na 1 ani na 3. "Ani chybi kwestia linki bądź sprężyny" - myślę, lecz uważniejsze oględziny wskazały faktyczną przyczynę. Okazało się że do mechanizmu nawpadało mi drobnych kamyków z drogi (łącznie trzy sztuki) i to one fizycznie blokowały wszelki ruch pantografu.

Na dziurze tuż za ostatnim mostem przed Asgarður straciłem na dziurze dwie szprychy w tylnim kole - szczęśliwie urwały się tylko łepki nypli, także naprawa tych zniszczeń zajmie może 15-20 minut. Inaczej byłaby zabawa ze zdejmowaniem kasety i przeplataniem szprychy przez tarczę hamulca, a ja głupi zapomniałem wziąć śrubokręt do torxów bądź wymienić te śrubki na takie pod imbusy! Cóż… na chwilę obecną koniec "rumakowania" :)

\img{./photos/x-s-2014-08-13_16-26-43__241.jpg}{german_polish_horses}{Dwa rumaki - niemiecki i polski}

Zaraz po dojeździe na kemping - obiad, a potem - w góry! Postanowiliśmy przejść się pętlą wokół doliny Hveradalir, która nosi miano "prawdopodobnie nakpiękniejszego obszaru geotermalnego na Islandii". Jeśli starczy nam czasu i ochoty, wdrapiemy się jeszcze na najwyższy szczt w okolicy, Snækollur (1477 m n.p.m.)

W drodze napotkaliśmy liczną młodzieżową grupę turystów górskich z USA. Któryś z nich spytał nas "skąd my?", a gdy odparliśmy że "Poland", zaraz zawołał "Dżoana! Here are folks from Poland!". Joanna okazała się być córką polskich emigrantów i tak się ucieszyła, że może rozmawiać po polsku, że przegadaliśmy dobre 10 minut… Czego ciekawego się dowiedzieliśmy? Ano na przykład że idą tydzień z buta przez Kjölur, z całym ekwipunkiem na grzbiecie, a smaku dodaje fakt, że około 5 z 30 osób nigdy nie biwakowało (choćby przez jeden dzień)! Szaleńcy…

\img{./photos/x-s-2014-08-13_21-18-39__253.jpg}{near_hveradalir}{W okolicach Hveradalir (w tle Snækollur)}
\img{./photos/x-s-2014-08-13_20-47-40__249.jpg}{hveradalir_moss}{Mech. Kolory oryginalne.}

Osiągnięcie rejonu Hveradalir okazało się trudnym zadaniem, szczególnie ze względu na fatalne znakowanie szlaku (części tyczek - schowanych gdzieś za skałkami - w ogóle nie było widać)! Ponieważ wiemy już, że na Snækollur nie zajdziemy, a ja pragnąłem popodziwiać trochę ładnych panoram, więc u podnóża szczytu Mænir rozdzielamy się. Po raz kolejny na czworaka przeprowadzam atak szczytowy po tłuczniu, ponownie podczas powrotu stosuję standardową kombinację zsuwania się i zbiegania. Czy było warto? No, tym razem mam wątpliwości. Z jednej strony owszem, widoki niczego sobie (nawet gdzieniegdzie jakby ocean był widoczny), z drugiej - gdy zobaczyłem na aparacie Puta bajkowe zdjęcia z Hveradalir, to faktycznie tam było pięknie i inaczej niż w innych obszarach geotermalnych… Dnem snują się opary, ściany doliny porastają jaskrawozielone mchy, kolor skał to pełny gradient, a w ramach bonusu do widoków można jeszcze pomoczyć nogi w ciepłych strumieniach.

W ośrodku wczasowym Kerlingarfjöll już trzeci sezon pracuje Polka ze Szklarskiej Poręby (“wyhaczyła” mnie - a jakże - dzięki fladze na rowerze:) Ona zadowolona, Islandczycy zadowoleni, więc tym razem przyjechała ze swoim chłopakiem (on też do pracy). W sumie w takim miejscu to można popracować ze dwa tygodnie, w wolnych chwilach robiąc wycieczki po okolicy. Szlaków od groma, spokojnie da się z nich sklecić 4 ładne pętle.

\img{./photos/x-s-2014-08-13_22-13-19__396.jpg}{hveradalir_panorama}{Hveradalir w pełnej krasie.}
\img{./photos/x-s-2014-08-13_19-49-15__97.jpg}{hveradalir_feet}{Ulga dla zmęczonych stóp zapewniona.}

Krótki raport meteo. Kolejny dzień, gdy właściwie przez cały czas świeciło słońce, a wiatr wiał w plecy. Zaczynam wierzyć w magię Kjöluru! (Niestety, zimne podmuchy nie zawsze pozwalają na jazdę w krótkim rękawie.)

Coś się dziś przy kolacji rozmarzyliśmy i stwierdziliśmy, że "kiedyś, gdy będziemy bogaci, przyjedziemy na Islandię solidnymi jeepami, będziemy spać po chatach, jeździć na koniach, pić piwo po 25 zł za puszkę 0,5 l i ogólnie korzystać z życia!" Na chwilę obecną jeździmy rowerami, nocami marzniemy w namiotach i liczymy każdy grosz.

Chwilę po 21:00 jesteśmy już w komplecie w bazie. Na króciutkiej naradzie sztabowej podejmujemy decyzję o odjechaniu parę kolometrów w stronę \road{35}, w sam raz na tyle by nie płacić za kemping. Pech chciał, że teren na którym się rozbijamy (koło pierwszych wodospadów, licząc od strony Kerlingarfjöll) tylko z drogi wyglądał na przyjazny. W rzeczywistości momentami było to raczej bagienko, skutkiem czego buty znów nadają się do suszenia.

\hint{W miejscach szczególnie narażonych na "ciągnięcie od ziemi" sprawdza się patent z folią NRC - wystrczy wyłożyć nią podłogę namiotu i od razu robi się cieplej!}

	\chapter*{14.08.}

Wczoraj pewien Niemiec na motorze, spotkany na skrzyżowaniu przed Asgarður, solidnie nas nastraszył: \emph{,,Pierwsze 10-15 km od skrzyżowania to będzie jakiś dramat --- żwir i kamienie przez cały czas. Ja na motorze jakoś dałem radę przejechać, ale nie wiem, czy wam też się uda.''} Każde z nas wyobraża już sobie mozolne brnięcie przez morze kamyków i ile to godzin będziemy w związku z tym ,,w plecy''. Jedziemy jednak, jedziemy, kilometry mijają, a tu nic! Jedzie się całkiem znośnie, żadnego zsiadania z roweru, może trafiły się dwa krótkie stromsze podjazdy, na których trzeba by trochę się skoncentrować na szukaniu dogodnego przejazdu. Kawałek przed mostem przy Hvítárvatn zaczyna się dobra, twarda nawierzchnia, a typowy asfalt --- na 13-15 km przed Gullfossem.

\img{./photos/x-s-2014-08-14_15-06-56__260.jpg}{rock_break}{Postój przy skałce-wiatrochronie --- jedynej w promieniu wielu kilometrów}
\img{./photos/x-s-2014-08-14_15-08-38__407.jpg}{lone_rider}{Kasia jako Samotny Jeździec}

Trapią nas dziś liczne usterki. Karolinie pękł trzeci już wspornik bagażnika oraz zrobiła się dziura w sakwie (z przetarcia). Podobną dziurę w sakwie zyskał też Put. Natomiast mnie przetarta się tylna opona --- przetarła całkiem konkretnie, można palec włożyć w dziurę. Oczywiście od razu złapałem przez to kapcia. Od mijających nas Niemców wyżebrałem łatki na oponę, lecz wcześniej już ,,załatałem'' dziurę kawałkiem brezentu (a potem jeszcze zmieniłem oponę tylną z przednią, bo przednia się aż tak bardzo nie starła) --- wierzę, że jadąc wystarczająco ostrożnie jakoś doturlam się do Keflaviku…

\hint{Jeśli nie chcesz wozić ze sobą zapasowej opony, zainwestuj w łatki na oponę --- na przykład \href{http://www.competitivecyclist.com/park-tool-emergency-tire-boot-set-tb-2c}{TB-2C} firmy Park Tool. Taka łatka przypomina dużą, grubą, powlekaną naklejkę i doskonale sprawdza się na tymczasowe (!) naprawy.}

Gdy tylko przejechaliśmy po pierwszych metrach nawierzchni bitumicznej, z radości aż urządziliśmy ,,powitanie asfaltu''. Po czterech dniach jazdy bezdrożami nagły brak ciągłych podskoków i ,,uciekania'' tylnego koła był czymś wprost niesamowitym!

Powrót do cywilizacji to nie tylko asfalt, ale i bliskość sklepów. W związku z tym mogliśmy sobie pozwolić na autentyczne świętowanie i tak nasza kolacja okazała się być trzydaniowa. Jako danie główne występowały parówki z ryżem, potem wparował kuskus z warzywami, a na deser --- herbatniki imbirowe z herbatą oraz daktyle. W ten sposób pozbyliśmy się właściwie dokumentnie całej żywności.

\img{./photos/x-s-2014-08-14_16-12-19__98.jpg}{tyre_repair}{Na ostatnich kilometrach zawsze się coś spieprzy…}
\img{./photos/x-s-2014-08-14_18-12-22__100.jpg}{welcome_asphalt}{Powitanie asfaltu.}
\img{./photos/x-s-2014-08-14_19-28-50__101.jpg}{icelandic_bus}{Autobus turystyczny --- wersja islandzka}
%TODO: wrzucić te zdjęcia w odpowiednie miejsca w tekście (np. powitanie asfaltu bezpośrednio pod jego opisem)
	\chapter {15.08.}

Spokojnie zwijamy sobie namioty, pakujemy sakwy, a tu widzimy, że drogą jedzie kolumna z 10 jednakowo ubranych rowerzystów z pełnym ekwipunkiem. Wyglądają tak... znajomo! Tak, mijaliśmy ich już koło Djúpivogur, tyle że wtedy oni jechali w przeciwnym kierunku. Nie daje nam spokoju, co to za impreza - rajd? kolonia? Wszystkiego dowiemy się wkrótce...

Rano kropi, potem pada, a momentami leje. Serwis pogodowy nie pozostawia złudzeń - taka aura utrzyma się do wieczora.

Zaraz po wjeździe do miejscowości Geysir, gdy tylko minąłem znak “teren zabudowany”, usłyszałem charakterystyczny syk gdzieś przy ziemi. No tak, brezentowa łatka też się przetarła (wraz z dętką) i teraz ja również przejeżdżając przez kałuże robię za gejzer...  Kiedy towarzystwo idzie oglądać tryskający gejzer, ja łatam oponę na zapleczu jakiś magazynów - i przy okazji natykam się na tych samych Niemców, którzy wczoraj dali mi łatkę :) Na wszelki wypadek tym razem już nie kombinuję z brezentem, nie dziaduję, tylko naklejam "naklejkę".

Powracająca od gejzera Trójka przynosi nowiny - owi pro-rowerzyści, to w rzeczywistości kanadyjska rodzina. Widząc zdziwione spojrzenia, kobieta wyglądająca na matkę szybko sprostowała: "Tak, jeździmy sobie z mężem i 9 moich dzieci." Wow! Brać taką chmarę dzieci, z czego najstarsze ma może 20 lat, a najmłodsze koło 10 na Islandię? Szacunek! Jeszcze gdy dodali, że ich objazdówka trwa łącznie dwa miesiące, a śpią właściwie codziennie na dziko, to nagle poczuliśmy się tacy mali. Bo my co drugi dzień nocujemy na kempingu, a i tak po 3 tygodniach mamy już dość.

Uwaga turystyczna. Nie nastawiajcie się, że gejzer to będzie jak fontanna wylewająca hektolitry wody albo hydrant z kreskówek, z którego tryska intensywny strumień wody. W rzeczywistości "erupcja" trwa 2-3 sekundy - na tyle krótko, że co po niektórzy nawet się nie orientują, że to już!

Nasz główny checkpoint na trasie to Reykholt - tam według mapy ma się znajdować spożywczak. Rzeczywistość nieco nas rozczarowała, bo zamiast Nettó albo chociaż Samkaupa zastajemy jedynie sklepik "na CPN-ie". Zgoda, kupiliśmy wszystko czego potrzebujemy po w miarę znośnych cenach, no ale nastawialiśmy się na szał zakupów... Za oknem pada, według prognozy deszcz przejdzie koło 18:00. A ponieważ w środku w miarę ciepło, więc siedzimy sobie ponad dwie godziny przy stolikach jedząc, korzystając z internetów i grając w karty. Podkręciliśmy nawet grzejnik z 1 na 3 - tak żeby szybciej się wysuszyć. Po jakiś 20 minutach przyszedł właściciel z krótkim pytaniem: Did you turn on the heat? - Yes. - That’s not a living room! On poszedł, my grzecznie skręciliśmy na 1, ale widzieliśmy że od tego momentu gość miał nas cały czas na oku.

\img{./photos/x-s-2014-08-15_16-24-58__105.jpg}{card_break}{Przymusowa przerwa w podróży}

W Kerið zatrzymaliśmy się, by oglądnąć sobie malowniczy krater. Tę atrakcję oglądamy będąc dokumentnie przemoczeni, trzęsąc się z zimna i szczając zębami, stąd w skrócie nasze nastawienie wygląda tak - "szybko zaliczyć, odfajkować"! Sam krater może i ładny, ale dlaczego za przejście dosłownie paru metrów i rzucenie okiem na jeziorko powulkaniczne każą płacić 350 kr?!

Chcieliśmy zanocować na kempingu przy krzyżówce \road{35} i \road{36} - widzianym na którejś mapie - ale oczywście kempingu brak. Mnie znów nawala przednie koło (co moment muszę dopompowywać), więc zatrzymujemy się w przydrożnym fast-food barze. Ja zmieniam dętki, a reszta zajada hot dogi i popija kawę.

\img{./photos/x-s-2014-08-14_22-19-41__266.jpg}{will_there_be_sun}{Będzie słońce?!}

Ponieważ jestem dokumentnie przemoczony, a temperatura nie rozpieszcza, więc by uniknąć zaziębienia się postanawiam pocisnąć samotnie na kemping do Þingvellir. Już, już niemal widzę zabudowania, gdy na ostatnim podjeździe (jakieś 6 km przed wioską) znów łapię kapcia. Z początku łudziłem się, że wystarczy trochę podpompować i jakoś zajadę, więc gdy przejeżdżający akurat drogą rangersi spytali "Do you need help?”, to odparłem “No, no problem”. Gdy jednak zobaczyłem jak szybko ucieka powietrze, zmieniłem zdanie i skorzystałem z zaoferowanej możliwość podwózki. Po raz pierwszy w takiej sytuacji nie mam poczucia, że poszedłem na łatwiznę, tylko że podjąłem słuszną decyzję.

Namioty rozstawiamy po nocy, koło 23:00. Mimo ciemności jakoś udało nam się znaleźć zaciszne miejsce u podnóża rosłych krzaków - istnieje cień szansy, że w nocy nie zdmuchnie nam naszych małych domków. Pozostała jeszcze kwestia gotowania obiadu. Po dramatycznych pierepałkach z palnikiem, który gasł co chwila, przyszedł czas na zrobienie herbaty. W tym momencie już skapitulowaliśmy i zostawiliśmy zapalony palnik na zewnątrz, a impreza przeniosła się pod prysznic. W oczekiwaniu na wrzątek Kasia zabawiała nas deklamując "Baśń o stalowym jeżu" :)

%% {cytaty dot. deszczu i wiatru (“jak Islandia zmienia perspektywę”) → fb}
	\chapter*{16.08.}

Podjęta z rana próba załatania dętki zakończyła się połowicznym sukcesem --- niby powietrze ucieka, lecz na tyle powoli, że wystarczy co godzinę odrobinę podpompować i jakoś można jechać. Ciężko mi zdiagnozować, co było przyczyną powstania tej nowej nieszczelności --- przetarte fragmenty opony? niemiecka łatka? mój podkład z brezentu? taśma izolacyjna?

Ponieważ do Reykjavíku mamy raptem 50 km, więc niespiesznym tempem ruszamy o 14:15. A właściwie to o 15:00, spod parkingu koło takiego uskoku tektonicznego. To kolejne miejsce, gdzie kręcono epizod Gry o Tron (a dokładniej scenę gdzie Petyr Baelish prowadzi Sansę Stark wąwozem do bramy do Eyrie.

Wczorajszy manewr polegający na dobiciu do Þingvellir okazał się strzałem w dziesiątkę --- dziś wieje silny, północny-północnozachodni wiatr, który momentami "kładzie" nas na bok na drodze. A to znów dzięki prognozie pogody prosto od vedur.is!

Zbliżając się do Mosfellsbær przeżywamy szok, bo oto widzimy latarnie “Reykjavik megalopolis” (oraz wiaty MPK!) ustawione niemal w szczerym polu, parę kilometrów przed pierwszymi poważniejszymi zabudowaniami. Nie ma co ukrywać, silnie kontrastuje to z naszymi dotychczasowymi doświadczeniami z islandzką "architekturą drogową".

\img{./photos/x-s-2014-08-16_17-32-37__106.jpg}{sandpit_dinner}{Gotowanie obiadu (niemal) w piaskownicy}

W Akureyri było śniadanie pod Bonusem, a tym razem jest obiad. OK, nie pod samym Bonusem, a na pobliskim placu zabaw --- niemniej też klimatycznie! Matki z dziećmi miały duży ubaw obserwując nas spożywających posiłek w takim otoczeniu.

Wieczorem wybraliśmy się na spacer nadmorską promenadą do gmachu Harpy --- to był obowiązkowy punkt programu, zwłaszcza dla studentki architektury Karoliny. Potem poszliśmy na piwo do knajpy --- kolejna rzecz, która chodziła za nami od wielu dni. Było to osobliwe doświadczenie, bo normalnie na piwo ze znajomymi chodzi się po to, żeby pogadać co u kogo słychać. A my co? Mamy się pytać nawzajem "Hej! Co ciekawego robiłeś/robiaś przez ostatni miesiąc?" Bez sensu!

Czuć już atmosferę powrotu, dodatkowo potęgowaną przez szok cywilizacyjny. Naprawdę, na wjeździe "jedynką" do Reykjaviku aż nie wiedzieliśmy, jak się zachować --- nagle znaleźliśmy się na dwupasmowej drodze, gdzie co moment są ronda, a obok nas śmigają jeden za drugim samochody i tiry.

\img{./photos/x-s-2014-08-16_21-21-32__108.jpg}{winners}{Zdobywcy.}
\img{./photos/x-s-2014-08-17_00-03-06__420.jpg}{harpa}{Harpa --- kawał ciekawej islandzkiej architektury}
	\chapter*{17.08.}

Tak ciężko jak dziś, to nam się jeszcze nigdy nie zbierało. Dolegujemy, przewracamy się z boku na bok, wymyślamy wymówki dlaczego nie chcemy wypełznąć z namiotów. Wszystko to z bardzo prostego powodu --- mentalnie wyjazd skończył się wczoraj, z chwilą wjechania na \road{1} przed Reykjavikiem. Skończyło się obcowanie z naturą i ,,przygoda'', biwak na kempingu gdzie oprócz nas obozuje z pół tysiąca osób nie ma klimatu, miejsce otoczenie również nie nastraja optymistycznie. Zresztą --- znów do przejechania mamy zaledwie 50 km, więc \emph{,,po co się spieszyć? co będziemy robić w Asbru?!''}

Wczesne popołudnie upływa leniwie na długich zakupach pamiątek w sklepach przy ulicy Laugavegur (to coś jak reykjawickie Krupówki). W sumie to dziewczyny kupowały, a ja z Putem siedzieliśmy na murku przed sklepem i ,,kontemplowaliśmy'' życie miejskie. Ktoś, kto by nas obserwował, mógłby z całą pewnością stwierdzić: ci dwaj goście mają \href{http://en.wikipedia.org/wiki/Thousand-yard_stare}{,,spojrzenie na 1000 jardów''} --- tym objawia się nagły kontakt ze zbyt dużą ilością bodźców i ludzi\textellipsis

\img{./photos/x-s-2014-08-17_16-12-32__421.jpg}{is_this_the_end}{A więc\textellipsis to już koniec?!}

%TODO dialog
Tak sobie siedzimy na murku, gdy nagle zauważamy trzech starszych panów kręcących się koło nas i co chwila rzucających nerwowe spojrzenia w naszą stronę --- widać naradzają się, ale co takiego knują? Wreszcie jeden z nich podchodzi i pyta: Where are you from? I don’t see any country for this flag! Odpowiadamy, że ,,from Poland'', a oni na to: ,,Ale to powinna być flaga niebiesko-biało-czerwona, prawda?'' Jego kompan protestuje --- kolory na odwrót! (Pewnie zrozumieli, standardowo, ,,Holland'' zamiast ,,Poland''.) To gdzie teraz? --- ,,Do domu, do Berlina, do Wa-wy\textellipsis'' --- Aha, a ile kilometrów przejechaliście? --- 2000 --- z Warszawy?! Ogólnie to straszne tępaki, aż zachodziliśmy w głowę gdzie takie się rodzą, lecz na do widzenia sami się przyznali: ,,Greetings from Canada!''

Po zakupach w butikach zajrzeliśmy na moment do katedry, potem na drugie śniadanie do Bonusa i\textellipsis znów ruszamy w trasę.

W połowie drogi do Ásbrú rozegrał się istny dramat. Zatrzymaliśmy się na postój, oparłem rower o stojącą na poboczu latarnię, a tu nagle --- potężny huk! Ewidentnie coś eksplodowało i to eksplodowało w moim rowerze! Podejrzanych nie było wielu --- to znów feralna dętka w przednim kole. Najwyraźniej niemiecka łatka przetarła się (w końcu przejechała już blisko 200 km!) i zalegający pobocze ostry żwirek przeciął dętkę. Mniejsza o przyczyny, najgorsze że tego nie ma jak normalnie załatać! Zrezygnowany zaczynam iść dalej pieszo, lecz to bez sensu. Próbuję więc jednak coś zaradzić i zaklejam dziurę kawałkiem innej dętki, a całość dodatkowo owijam dla wzmocnienia taśmą izolacyjną --- znów klapa, powietrze uszło po stu metrach. Tym razem podkopało to już moje morale na tyle, że na dobre zaczynam pieszą wędrówkę w stronę Ásbrú. Z resztą jestem umówiony tak, że zostawią mi coś do jedzenia, a ja im dam rzeczy obiadowe. Po 1,5 km poddaję się --- co innego prowadzić rower po mieście, a co innego wyładowany do granic, ważący ze 30 kilo grzmot! Kombinuję zatem, co począć. Wygrzebuję odrobinę nieszczelną dętkę (ostatnią z zapasu), kroję łatkę do opon na mniejsze kawałki i podejmuję ostatnią rozpaczliwą próbę naprawy opony. Tym razem próba ta zakończyła się (chyba) sukcesem i w miarę sprawnie docieram w okolice Keflavík, gdzie spotykam (a właściwie --- doganiam) resztę grupy.

Podsumowując, dziś jazda dała nam się solidnie we znaki. Nie chodzi już o awarie, tylko bardziej o ruch ,,jak na Alejach'', wiatr w twarz, permanentny podjazd (na szczęście niewiele nachylony), a przede wszystkim --- o uczucie ,,już wszystko się skończyło\textellipsis co my tu jeszcze robimy?!''

Znów gości nas Artur, co stanowi dla nas naprawdę duże ułatwienie w logistyce. Wieczorem robimy małe pogaduchy z nim i jego żoną-Iranką. Główną osią były wspomnienia z objazdu, lecz momentami rozmowa schodziła na inne tory --- na przykład na religię. Najbardziej rozbroiła mnie ta oto kwestia, wygłoszona przez żonę Artura: \emph{,,Most of the people in Europe say they believe in some Jesus, but hopefully I’ve met some who said they believe in God''.}

\img{./photos/x-s-2014-08-17_23-52-35__112.jpg}{exhaustion}{Wieczór. Nocleg. Ciepło i miękko.}
	\chapter*{18.08. --- To już jest koniec\textellipsis}

\textellipsis i tak oto nasz ,,tabor cygański'' wyrusza z Ásbrú --- z pieśnią ,,My, cyganie'' na ustach i z Putem, robiącym za samolot, na czele\textellipsis Szczęśliwie podróż spod akademika na lotnisko odbyła się bez niespodzianek i tak oto pętla została definitywnie zamknięta. Teraz ostatnie nerwy: czy uda się wszystko spakować? czy rowery wejdą do pudła? czy nie odwołają lotu z powodu erupcji wulkanu?

Gdy tak siedzieliśmy na lotnisku przepakowując się, przed wejście zajechała karetka pogotowia. Pierwsza myśl --- oho, wypadek! Ale nie, po chwili drzwi od przedziału transportowego otwarły się i stanął w nich ratownik z walizką w ręku. Druga myśl --- może szef jakiegoś szpitala jedzie na wczasy i karetka robi za taksówkę? W tej chwili za ratownikiem powoli zaczął gramolić się\textellipsis nasz znajomy Anglik! Coś nam się jednak nie zgadzało. Dopiero po chwili spostrzegliśmy, że nasz druh paraduje w kołnierzu ortopedycznym. Konsternacja. Co się stało, że gość którego widzieliśmy niecałe dwa tygodnie temu w dobrej formie, nagle wylądował w takim stanie? Ano wybrał się pewnego razu na wycieczkę bez sakw, gwałtownie zahamował i przeleciał przez kierownicę, nadwyrężając kręgosłup. Mówił, że z początku leżał parę minut wręcz bez czucia w rękach i nogach\textellipsis Teraz wraca do Wielkiej Brytanii, gdzie przejdzie półroczną rehabilitację. Rokowania są dobre.

\img{./photos/x-s-2014-08-18_10-14-20__114.jpg}{my_cyganie}{My, Cyganie\textellipsis}

\pagebreak

Berlińczycy mieli niezłą łamigłówkę, jak spakować cały swój bagaż do odprawy, a wszystko to z chęci zaoszczędzenia pieniędzy. Na drogę powrotną wykupili tylko dwa bagaże rejestrowe, a jest ich trójka. Ileż było prób ważenia, ile przymiarek z cyklu ,,co wziąć do bagażu podręcznego''! Te kombinacje zajęły tyle czasu, że w kolejce do odprawy towarzystwo ustawiło się dopiero pół godziny po rozpoczęciu odprawy.

Gdy zarówno bagaże jak i rowery przeszły kontrolę bezpieczeństwa odetchnęliśmy wszyscy z ulgą --- ilość spraw, które mogą ,,pójść nie tak'' została ograniczona do minimum. Jeszcze zjedliśmy pożegnalne drugie śniadanie na posadzce terminalu, zakupiliśmy trunki w~sklepie wolnocłowym i\textellipsis nadszedł czas pożegnań! Berlińczycy w swoją stronę, ja w swoją.

\img{./photos/x-s-2014-08-19_06-18-13__116.jpg}{on_the_airport}{\textellipsis i my koczownicy.}

Przez pierwsze pół godziny lotu wprost nie mogłem odkleić nosa od szyby. Widzieć w mgnieniu oka trasę, którą mozolnie pokonywaliśmy przez cały pierwszy tydzień --- bezcenne! Co więcej, patrząc z nieba, sandury czy lodowce robią jeszcze większe wrażenie!

Na warszawskim lotnisku Okęcie od razu poczułem, że wróciłem do Polski --- na odbiór bagażu nasz rejs czekał blisko półtorej godziny! Dramat. Gdy już wreszcie zajechałem na Dworzec Centralny, znów miałem sytuację rodem z \emph{Misia}\footnote{Kultowa komedia w reżyserii Stanisława Barei, ukazująca absurdy PRL-u.}. Pani w kasie IC mówi, że nie może mi sprzedać biletu dla osoby z rowerem, bo już nie ma miejscówek. Pytam się, czy jest w stanie sprzedać mi bilet tylko dla jednej osoby (bez roweru) --- tak, oczywiście, żaden problem. Tylko że na miejsca siedzące też nie ma miejscówek. Czyli ostatecznie w kasie kupiłem coś uprawniającego mnie do zajęcia miejsca w pociągu, a potem u konduktora zakupiłem bilet na rower (bez dopłaty, yey!)

Miło znów podziwiać polskie krajobrazy. Mimo chłodu nocy i mimo padającego momentami deszczu, stoję przy otwartym oknie i wpatruję się w gęste lasy, obserwuję mgły ścielące się po szuwarach\textellipsis

O 5:20 pociąg zatrzymał się na dworcu Kraków Główny. Jeszcze tylko po raz ostatni ,,okulbaczyć'' mego konia mechanicznego, przejechać ostatnie kilometry i\textellipsis \emph{Home, sweet home}! Wszędzie dobrze, ale w domu najlepiej! Dla mnie islandzka eskapada dobiegła końca.

%TODO zdjęcie z pociągu


\end{document}
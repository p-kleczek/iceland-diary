\chapter*{04.08. --- Askja}

Tej nocy nikt nie przespał dobrze --- każdy raczej jak mysz pod miotłą --- a to wszystko za sprawą grozy, jaką napawała nas perspektywa płacenia kary za nocleg w miejscu niedozwolonym. Dręczyły nas koszmary senne, właśnie z rangersami w roli głównej. Mnie na przykład śniło się, że mandat wyniósł po 150~€ na łebka, a cała nasza akcja z rozbijaniem się została uwieczniona na monitoringu. Niby rangersi byli Polakami i pewnie by nam to puścili płazem --- potraktowaliby nas jako wyprawę geologów --- lecz ich szefostwo już widziało nagrania, więc nie ma zmiłuj. Fabuła snów pozostałych uczestników naszej eskapady była mocno zbliżona. Tak więc, co by się już więcej nie stresować, zwijamy się ,,wcześnie'' rano (koło 8:00), odjeżdżamy paręset metrów od ,,miejsca zbrodni'' i dopiero tam rozkładamy się ze śniadaniem.

Głównym naszym problemem dnia dzisiejszego jest brak wody pitnej. Od momentu przekroczenia ostatnich brodów nie mijaliśmy żadnego strumienia z wodą zadatną do picia i nie miniemy już żadnego kolejnego aż do Dreki!

\hint{Podczas jazdy przez interior wykorzystuj każdą okazję, by uzupełnić zapas wody! Wiele z rzek i strumieni zaznaczonych na mapie wypływa spod lodowców, a woda jest na tyle mętna, że picie jej polega bardziej na piaskowaniu zębów\textellipsis}

\img{./photos/x-s-2014-08-03_19-52-17__253.jpg}{interior_roads_1}{Codzienność w interiorze to ,,tarka''\textellipsis}
\img{./photos/x-s-2014-08-04_12-53-21__269.jpg}{interior_roads_2}{\textellipsis i piach}

Jazda przypomina teraz jakiś ponury żart, trochę w stylu \href{http://pl.wikipedia.org/wiki/Paradoksy_Zenona_z_Elei#Achilles_i_.C5.BC.C3.B3.C5.82w.5B2.5D}{paradoksu z Achillesem i żółwiem} --- tyle że tu pytanie brzmi ,,czy rowerzyści dojadą (dziś) na kemping?'' Początkowo poruszamy się dość dobrą drogą, z prędkością 12~km/h, do celu pozostało nam 40~km więc snujemy już wizje, jak to za powiedzmy 4 godziny zajeżdżamy na kemping. (Tak, wiem że to naiwne, ale umysł niejednego rowerzysty przeprowadza często takie optymistyczne szacowanie.) Cóż, po paru kilometrach dobra droga kończy się, a zaczyna więcej żwirku. Nie jest źle, 9~km/h, czyli za 4~godziny będziemy. Na 15~km przed kempingiem (czyli na \mbox{3--4~km} przed skrzyżowaniem z \road{F88}) zaczynają się takie wydmy, że nie ma bata --- właściwie non stop musimy pchać rowery. Nasza prędkość spada do \mbox{3--4~km/h}, a więc wciąż na dotarcie do kempingu potrzebujemy około 4~godzin\textellipsis Ku naszej uldze, od skrzyżowania zaczynała się swoista ,,autostrada'', którą w 40~minut dotarliśmy do celu. Ale to wcześniejsze asymptotyczne zmierzanie do celu było takie demotywujące!

\img{./photos/x-s-2014-08-04_14-45-35__271.jpg}{askja_vehicle}{Ech\textellipsis czemuż nie założyliśmy i my takich opon?}
\img{./photos/x-s-2014-08-05_15-25-57__283.jpg}{impassible}{Impassible is nothing.}

Na skrzyżowaniu wita nas ręcznie wykonana tabliczka, że \road{F88} jest \emph{,,Impassible for small jeeps due to high water level''}. Że co? Niedługo później sprawa się wyjaśniła: dozorca w Dreki powiedział nam, że lodowcowa rzeka zalała drogę na odcinku około 500~metrów tak, że woda sięga do kolan --- raczej nie przejedziemy rowerem. Na chwilę obecną nie martwimy się tym przesadnie, nasz plan wygląda następująco: jedziemy jutro do Herðubreiðarlindir i zobaczymy, co dalej --- albo miniemy rzekę ,,na dziko'' (przez pustynię) albo będziemy żebrać u jakiś Niemców poruszających się po interiorze w takich księżycowych pojazdach o zabranie naszych rzeczy albo (w ostateczności) poczekamy na autobus do Mývatn o 15:30, który powinien przewieźć nasze bagaże.

O 15:00, po blisko sześciu godzinach jazdy i pchania, wreszcie osiągamy kemping Dreki. Jawi się on nam jako oaza: sanitariaty, olbrzymia i dobrze wyposażona wspólna kuchnia z jadalnią, możliwość rozbicia się w ,,cieniu wiatrowym'' budynku mieszkalnego\textellipsis Niestety, nie ma prądu, a z ciepłą wodą też bywają problemy --- choć prysznic kosztuje standardowe 500~kr, to lejącą się wodę można określić bardziej jako letnią niż ciepłą, co empirycznie sprawdziła na sobie Karolina.

Gdy już się rozbiliśmy, zjedliśmy obiad i odpowiednio ubraliśmy, wyruszyliśmy na pieszą wycieczkę nad jezioro Viti. Cel: kąpiel w gorących źródłach (stąd połowę naszego worko-plecaka zajmują stroje kąpielowe, ręczniki i klapki). Pogoda nie rokuje: mgła, deszcz, a w wyższych partiach --- padający śnieg. Dodatkowo zejście nad Öskjuvatn utrudniają płaty śliskiego śniegu (a tylko ja mam buty górskie). Wszystko to sprawiło, że po dojściu w okolicę Bathsraun samotnie pobiegłem na króciutki zwiad, który niestety nie potwierdził bliskości jeziora Viti. W związku z tym oraz z uwagi na pogarszającą się pogodę zarządziliśmy odwrót. W sumie\textellipsis może to i dobrze? Słyszeliśmy, że ledwie tydzień wcześniej potężna lawina błotna sprawiła, że poziom zwierciadła wody w Öskjuvatn podniósł się o parę metrów, przez co chłodna woda przelała się do Viti. W efekcie nie dość, że źródło przestało być chwilowo gorące, to jeszcze ścieżka została uszkodzona.

%TODO: umieścić to w tekście
\curiosity{Nazwa ,,Bathsraun'' ma interesującą etymologię --- sprawdź \href{http://davemcgarvie.wordpress.com/2012/04/}{skąd się wzięła.}}

\img{./photos/x-s-2014-08-04_20-35-49__186.jpg}{dreki}{Dreki --- oaza na bezkresnym pustkowiu}
\img{./photos/x-s-IMG_20140804_203225178.jpg}{way_to_viti}{Wyprawa nad jeziorko Viti --- typowe islandzkie lato}

Po zejściu do Dreki urządzamy sobie istną ucztę: chleb z konserwą i darmową musztardą, purée ziemniaczane, ciasteczka imbirowe moczone w herbacie\textellipsis W międzyczasie prawie przy wszystkich pozostałych stolikach odbywa się konsumpcja mniej lub bardziej procentowego alkoholu. Ci lekko już zawiani emeryci z Niemiec i Francji dość osobliwie patrzą na nas, gdy nożem wygrzebujemy z puszki mięso.

Dużą radość sprawiło nam przysłuchiwanie się dialogowi Pana Ruska z Panem Anglikiem. Dialog zainicjował Pan Rusek, który widząc łyżko-widelco-nóż spytał z charakterystycznym, znanym z zachodnich filmów, rosyjskim akcentem: \emph{,,What is this? I don’t understand\textellipsis Why not normal things?''} Aż parsknęliśmy śmiechem, gdyż nasze wcześniejsze i późniejsze doświadczenia faktycznie wykazały wyższość tradycyjnych, metalowych sztućców nad plastikowymi niezbędnikami --- szczególnie w przypadku widelca.

Put miał awarię kwaśnego mleka w kartonie. Na szczęście uszkodzenie kartonu było na tyle małe, że gęste kwaśne mleko nie zalało wszystkiego. Nie obyło się jednak bez mycia sakwy i części rzeczy.

\hint{Wszelkie płyny (jak sok bądź mleko) warto wozić w plastikowych butelkach. Lepiej stracić trochę cieczy podczas operacji przelewania niż póżniej wąchać zakiśnięte mleko\textellipsis}

\processifversion{PDF}{
\hint{Mapę okolic Askji znajdziesz na stronie internetowej parku narodowego Vatnajökull: \url{http://www.vatnajokulsthjodgardur.is/media/fixlandia/N\&A_100k_EN_20131022.jpg}}
}

\processifversion{HTML}{
\hint{Mapę okolic Askji znajdziesz na \href{http://www.vatnajokulsthjodgardur.is/media/fixlandia/N\&A_100k_EN_20131022.jpg}{stronie internetowej} parku narodowego Vatnajökull.}
}
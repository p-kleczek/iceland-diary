\chapter*{07.08.}

Z samego rana poszliśmy na spacer na otaczające dolinę klify --- główną \href{http://www.visithusavik.com/attractions/asbyrgi-canyon/}{atrakcję} okolicy. Żadnych trudności terenowych, niewielki dystans, pogoda (jeszcze) w miarę stabilna --- ot… taka przechadzka na rozprostowanie nóg.

Na śniadanie po raz pierwszy zjedliśmy (łącznie, na grupę): 1 kg paszy zalanej 1 l mleka i 1 l surmjolka --- czyli średnio po 2000 kalorii na osobę! (Należy w tym miejscu zaznaczyć, że zaczynaliśmy od ½ tej porcji, a i tak ledwie taką ilość przejadaliśmy --- teraz te kilolgramy tak akurat na styk.)

\img{./photos/x-s-2014-08-07_10-26-01__82.jpg}{asbyrgi_camping}{Nocleg u stup klifów.}
\img{./photos/x-s-2014-08-07_15-19-08__209.jpg}{asbyrgi_breaking_the_law}{Znów łamiemy prawo :(}

Z Ásbyrgi wyruszamy przy wietrze w plecy --- szok i niedowierzanie. Radość trwa krótko, bo po 25 km, gdy tylko zatrzymaliśmy się na posiłek, zaczęło padać :(

%TODO: dialog
Coś na rozluźnienie. Scenka wydarzyła się na koniec solidnego podjazdu. Ledwo zipiemy, pot się leje, wieje nam w mordę.
ja: No i co tak ludzie nam zdjęcia robią? Nawet się nie uśmiechną! Dobrego słowa nie powiedzą! Tylko ten znudzony wyraz twarzy…
Put: Eee.. dobrze że robią…
ja: No to niech chociaż jakieś ciastko w zamian dadzą!
Put: Czyli chciałbyś być taką dziwką rowerową? Sprzedawałbyś swoje podjazdy za ciastka?

\section*{Húsavík}

Specjalnie zajechaliśmy odpowiednio wcześniej do Húsavíku, by Put i Kasia mogli załapać się na \emph{whale watching}. Mimo intensywnej agitacji ze strony Puta, ani Karolina ani ja nie zdecydowaliśmy się na dołączenie. Powody? Karolina widziała już z bliska delfiny (a to prawie jak widzieć wieloryby), gdy pływała żaglówką po Morzu Północnym, mnie natomiast do rejsu nie zachęcała pogoda (zimno, pochmurno, wietrznie). Co więcej, żadnej gwarancji zobaczenia wieloryba nie ma --- pani w informacji powiedziała, że dwa poranne rejsy nic nie widziały, kolejny był już ciut lepszy ale też bez szału. \emph{,,Natomiast na pewno mogę zagwarantować, że będzie chłodno!''} --- dodała z uśmiechem. (Dla ustalenia uwagi, rejs trwa około trzy godziny).

\img{./photos/x-s-2014-08-07_19-54-58__84.jpg}{whalewatching_clothing}{Kasia w pełnym rynsztunku}

Ja i Karolina pożegnaliśmy więc ,,wielorybników'' i pojechaliśmy zakładać obozowisko. Kemping kosztował 1200 kr / os. + 500 kr za prysznic; był wyposażony w suszarnię z prawdziwego zdarzenia i małą kuchnię turystyczną z paroma palnikami i paroma stołkami. Gdy już się rozgościliśmy i przebraliśmy, ruszyliśmy na miasto ,,na rybę'' --- to znaczy głównie chodziło nam o to, by po ponad 2 tygodniach tułaczki wreszcie zjeść coś normalnego. Po obejściu wszystkich (czterech) portowych knajp i wnikliwym porównaniu menu oraz cen wybraliśmy restaurację \href{https://www.facebook.com/naustid}{Naustið}. Musieliśmy dobrą chwilę czekać na wolny stolik (taki był ruch), lecz przynajmniej wreszcie czekaliśmy na coś stojąc w ciepłym pomieszczeniu! Zamówiliśmy zupę rybną oraz \emph{,,2x szaszłyk rybny na 1 talerzu''} --- coś pysznego! W zupie pływała duża ilość tłuściutkich krewetek, a całość była dobrze doprawiona. Podobnie szaszłyczane rybie mięso było bardzo delikatne, a w ramach przybrania występowały: małe ziemniaczki, sałata, feta… Ach, to było jak podróż do innego, lepszego świata! Za oknem siąpi i hula wiatr, a my grzejemy się siedząc przy stoliku tuż obok kaloryfera i zajadamy się takimi smakołykami… I tylko żal było się tak szybko zmywać (bo ani się oglądnęliśmy, a zrobiła się 22:00 --- godzina zamykania lokalu). W drodze powrotnej zahaczyliśmy jeszcze o grill-bar przy stacji benzynowej (nie, bynajmniej nie po to, by psuć sobie smak, tylko żeby rozmienić pieniądze) i… to było coś koszmarnego. To było traumatyczne wręcz doświadczenie --- wejście do takiego czegoś po wizycie w przytulnej knajpce. Wejść do miejsca, gdzie przy świetle świetlówek zmarnowani ludzie męczą hamburgery i frytki. Brr…

%TODO: zdjęcie posiłku w barze rybnym

Cóż, i tak mimo wszystko trochę zepsuliśmy sobie smak szaszłyków i zupy rybnej, bo droga powrotna na kemping tak nas zmęczyła, że konieczne było dojedzenie chlebem z szoko…

Wielorybnicy wrócili o 23:30, nie zobaczywszy ani jednej ,,taakiej'' ryby. Na pocieszenie wrzucili za to na fb swoje sweet-selfie-focie w gustownych sztormiakach zawodowego rybaka morskiego ;)

PS. Dziś na podjazdach Kasia pokazała pełną klasę i prawdziwego ducha walki --- gratulacje!
